\section{Superficies hipersustentadoras}
Para os cálculos iniciais de desempenho, apresentados no \autoref{diagramarestricoes}, foi considerado que o coeficiente de sustentação máximo da asa era de 2.
Agora, com os cálculos refinados de aerodinâmica apresentados neste capítulo, o $C_L$ máximo da asa é, na verdade, de 1,62.
Isso significa que, para a decolagem e pouso, o valor desse coeficiente deve ser aumentado em pelo menos 0,4 durante pousos e decolagens para cumprir com os requisitos estipulados para a missão.
Como a geometria da asa foi otimizada para cruzeiro, ela não deve ser modificada para essa fase de voo.
A solução é utilizar uma asa de geometria variável, através do uso de superfícies hipersustentadoras: flaps e slats.

De acordo com os dados apresentados em \cite{gudmundsson}, é relativamente fácil conseguir esse aumento em $C_L$.
O uso de flaps simples (``\emph{plain flaps}'')  isoladamente já cumpriria com esse requisito.
O projeto aqui apresentado não se contentará com isso e procurará especificar o melhor sistema de hipersustentação possível, o que poderá ser usado para aumentar o $C_L$ máximo em iterações futuras e melhorar o desempenho da aeronave, possívelmente alterando seu ponto de projeto.

\subsection{Slats}
O slat a ser utilizado é um de três posições (pouso, decolagem e retraído).
A posição para pouso é escolhida para maximizar a sustentação e causa grande aumento de arrasto, o que torna o pouso mais controlado.
A de decolagem é projetada para maximizar o desempenho de subida.

De acordo com \cite{gudmundsson}, um flap desse tipo traz ganhos de
\begin{align}
    \Delta {C_l}_{\max} &\approx {0,84} \\
    \Delta {\alpha}_{\max} &\approx \ang{11} \\
    \Delta {C_d}_{\min} &\approx 80\cdot 10^{-4} \\
    \Delta {C_m} &\approx -{0,115}
\end{align}
A deflexão dos slats é de aproximadamente \ang{25}.

\subsection{Flaps}
Para aeronaves de transporte, o usual é a utilização de \emph{flaps fowler} com uma ou duas fendas (``\emph{slots}'').
Esse tipo de flap proporciona ganhos bastante elevados em $C_L$ e a opção de ganhos moderados ou grandes em arrasto, dependendo da configuração (decolagem ou pouso).
Para este projeto, o \emph{flap fowler} de uma fenda foi preferido devido à sua maior simplicidade construtiva, e a não haver necessidade dos ganhos de $C_L$ superiores que podem ser obtidos usando mais fendas.

De acordo com \cite{gudmundsson}, os ganhos esperados são de
\begin{align*}
    \Delta {C_l}_{\max}|_{\text{decolagem}} &\approx 1,16 \\
    \Delta {C_l}_{\max}|_{\text{pouso}} &\approx 1,92 \\
    \Delta {\alpha}_{\max} &\approx \ang{-7} \\
    \Delta {C_d}_{\min} &\approx 40\cdot 10^{-4} \\
    \Delta {C_m} &\approx -0,75
\end{align*}
A deflexão de decolagem é de \ang{15} e a de pouso é de \ang{40}.

\subsection{Novos coeficientes para a asa}
Os slats deverão ser instalados por toda a extensão da asa.
Os flaps obviamente não podem ser instalados no mesmo lugar dos ailerons, então só irão até essas
superfícies de controle.
Isso tem o benefício adicional de evitar o estol de ponta com as superfícies de hipersustentação defletidas, já que, sem os flaps, o ângulo de estol da ponta de asa será substancialmente maior que o da raiz.

Considerando que o aileron vai ocupar aproximadamente 20\% da envergadura, temos
que a área flapeada é dada por
\begin{equation}
    S_{\text{flap}} = 0,8 b c_r (1+\frac{\lambda}{0,8})/2 = 45.7\si{m^2}
\end{equation}

Fazendo a média ponderada nas áreas e ignorando o arrasto induzido para uma
primeira aproximação, os novos coeficientes esperados são dados por
\begin{equation}
    C_X = {C_X}_{\text{limpo}} + \sum_i{\Delta_i C_X \frac{{S_X}_i}{S}}
\end{equation}
e os seus valores são
\begin{align*}
    {C_L}_{\max}|_{\text{decolagem}} &\approx 3.4 \\
    {C_L}_{\max}|_{\text{pouso}} &\approx 4.1\\
    {\alpha}_{\max} &\approx \ang{22} \\
    {C_d}_{\min} &\approx 151\cdot 10^{-4} \\
    {C_M} &\approx -0,80
\end{align*}
