\chapter{Conclusão}
% \addcontentsline{toc}{chapter}{Conclusão}
\label{conclusao}

Com base nas análises de mercado realizadas, nos estudos comparativos das aeronaves e em estudos conceituais de aeronaves regionais com cerca de 50 passageiros , definiu-se algumas configurações para a aeronave que será projetada.

Em relação à ergonomia interna, definiu-se que a aeronave possuirá 16 fileiras com três assentos cada e uma fileira com dois assentos, totalizando 50 passageiros.
Visa-se diminuir o arrasto da aeronave com esta configuração.

Quanto à configuração externa, três possibilidades foram analisadas inicialmente, sendo a primeira uma aeronave turbo-hélice com asa alta, motores abaixo da asa e empenagem em T; a segunda uma aeronave turbofan com enflechamento negativo, asa baixa e motores na configuração \emph{pusher}; e a terceira uma configuração em \emph{canard}.
No entanto, conforme discutido, esta última foi considerada inviável e, portanto, descartada das análises futuras.
Logo, para as próximas etapas de projeto, serão analisadas duas possíveis configurações, buscando a aeronave mais competitiva e que melhor atenda a missão definida. 