\chapter{Aerodinâmica}
\section{Perfil aerodinâmico da asa}
\label{perfilasa}

O objetivo mais importante na escolha do perfil da asa foi a redução do arrasto, visando melhor desempenho e menor consumo em cruzeiro. Além disso, também foram considerados a sustentação máxima do perfil e a sua espessura. A sustentação máxima é essencial para o pouso e decolagem, mas ficou em segundo plano porque ela pode ser incrementada com o uso de superfícies hipersustentadoras---flaps e slats. Perfis mais espessos são desejáveis porque aumentam a eficiência estrutural da asa e o espaço disponível para combustível.

Com esses critérios em vista, é natural escolher um perfil de escoamento laminar (Natural Laminar Flow, NLF). Devido aos amplos dados experimentais disponíveis, a seleção se limitou à família NACA série 6. Os perfis selecionados após uma triagem inicial estão na tabela \ref{tbl:perfisNACASelecionados}.

\begin{table}[h]
  \label{tbl:perfisNACASelecionados}
  \caption{Perfis NACA selecionados}
  \centering
  \begin{tabular}{lcccc}
    \toprule
    Perfil &
    $ c_{l_{\max}} $ & $ c_{d_{\min}} $ & $ c_{m_{ac}} $ & t/c \\
    \midrule
    NACA 63\textsubscript{1}-412 & 1,77 &0,0044 & -0.075 & 0,12 \\
    NACA 64\textsubscript{2}-415 & 1,66 &0,0046 & -0.070 & 0,15 \\
    NACA 65\textsubscript{1}-412 & 1,64 &0,0038 & -0.070 & 0,12 \\
    NACA 66\textsubscript{3}-418 & 1,57 &0,0037 & -0.075 & 0,18 \\
    \bottomrule
  \end{tabular}
\end{table}

\begin{figure}[p]
    \begin{tikzpicture}[gnuplot]
%% generated with GNUPLOT 5.0p3 (Lua 5.1; terminal rev. 99, script rev. 100)
%% Wed 26 Sep 2018 01:38:03 AM -03
\gpmonochromelines
\path (0.000,0.000) rectangle (16.000,21.000);
\gpcolor{color=gp lt color border}
\node[gp node center] at (8.000,20.692) {\bf Coeficientes Aerodinâmicos dos Perfis Selecionados};
\gpcolor{color=gp lt color axes}
\gpsetlinetype{gp lt axes}
\gpsetdashtype{gp dt axes}
\gpsetlinewidth{0.50}
\draw[gp path] (1.504,14.779)--(7.447,14.779);
\gpcolor{color=gp lt color border}
\gpsetlinetype{gp lt border}
\gpsetdashtype{gp dt solid}
\gpsetlinewidth{1.00}
\draw[gp path] (1.504,14.779)--(1.684,14.779);
\draw[gp path] (7.447,14.779)--(7.267,14.779);
\node[gp node right] at (1.320,14.779) {$-1.5$};
\gpcolor{color=gp lt color axes}
\gpsetlinetype{gp lt axes}
\gpsetdashtype{gp dt axes}
\gpsetlinewidth{0.50}
\draw[gp path] (1.504,15.492)--(7.447,15.492);
\gpcolor{color=gp lt color border}
\gpsetlinetype{gp lt border}
\gpsetdashtype{gp dt solid}
\gpsetlinewidth{1.00}
\draw[gp path] (1.504,15.492)--(1.684,15.492);
\draw[gp path] (7.447,15.492)--(7.267,15.492);
\node[gp node right] at (1.320,15.492) {$-1$};
\gpcolor{color=gp lt color axes}
\gpsetlinetype{gp lt axes}
\gpsetdashtype{gp dt axes}
\gpsetlinewidth{0.50}
\draw[gp path] (1.504,16.204)--(7.447,16.204);
\gpcolor{color=gp lt color border}
\gpsetlinetype{gp lt border}
\gpsetdashtype{gp dt solid}
\gpsetlinewidth{1.00}
\draw[gp path] (1.504,16.204)--(1.684,16.204);
\draw[gp path] (7.447,16.204)--(7.267,16.204);
\node[gp node right] at (1.320,16.204) {$-0.5$};
\gpcolor{color=gp lt color axes}
\gpsetlinetype{gp lt axes}
\gpsetdashtype{gp dt axes}
\gpsetlinewidth{0.50}
\draw[gp path] (1.504,16.917)--(7.447,16.917);
\gpcolor{color=gp lt color border}
\gpsetlinetype{gp lt border}
\gpsetdashtype{gp dt solid}
\gpsetlinewidth{1.00}
\draw[gp path] (1.504,16.917)--(1.684,16.917);
\draw[gp path] (7.447,16.917)--(7.267,16.917);
\node[gp node right] at (1.320,16.917) {$0$};
\gpcolor{color=gp lt color axes}
\gpsetlinetype{gp lt axes}
\gpsetdashtype{gp dt axes}
\gpsetlinewidth{0.50}
\draw[gp path] (1.504,17.629)--(7.447,17.629);
\gpcolor{color=gp lt color border}
\gpsetlinetype{gp lt border}
\gpsetdashtype{gp dt solid}
\gpsetlinewidth{1.00}
\draw[gp path] (1.504,17.629)--(1.684,17.629);
\draw[gp path] (7.447,17.629)--(7.267,17.629);
\node[gp node right] at (1.320,17.629) {$0.5$};
\gpcolor{color=gp lt color axes}
\gpsetlinetype{gp lt axes}
\gpsetdashtype{gp dt axes}
\gpsetlinewidth{0.50}
\draw[gp path] (1.504,18.342)--(7.447,18.342);
\gpcolor{color=gp lt color border}
\gpsetlinetype{gp lt border}
\gpsetdashtype{gp dt solid}
\gpsetlinewidth{1.00}
\draw[gp path] (1.504,18.342)--(1.684,18.342);
\draw[gp path] (7.447,18.342)--(7.267,18.342);
\node[gp node right] at (1.320,18.342) {$1$};
\gpcolor{color=gp lt color axes}
\gpsetlinetype{gp lt axes}
\gpsetdashtype{gp dt axes}
\gpsetlinewidth{0.50}
\draw[gp path] (1.504,19.054)--(7.447,19.054);
\gpcolor{color=gp lt color border}
\gpsetlinetype{gp lt border}
\gpsetdashtype{gp dt solid}
\gpsetlinewidth{1.00}
\draw[gp path] (1.504,19.054)--(1.684,19.054);
\draw[gp path] (7.447,19.054)--(7.267,19.054);
\node[gp node right] at (1.320,19.054) {$1.5$};
\gpcolor{color=gp lt color axes}
\gpsetlinetype{gp lt axes}
\gpsetdashtype{gp dt axes}
\gpsetlinewidth{0.50}
\draw[gp path] (1.504,19.767)--(7.447,19.767);
\gpcolor{color=gp lt color border}
\gpsetlinetype{gp lt border}
\gpsetdashtype{gp dt solid}
\gpsetlinewidth{1.00}
\draw[gp path] (1.504,19.767)--(1.684,19.767);
\draw[gp path] (7.447,19.767)--(7.267,19.767);
\node[gp node right] at (1.320,19.767) {$2$};
\gpcolor{color=gp lt color axes}
\gpsetlinetype{gp lt axes}
\gpsetdashtype{gp dt axes}
\gpsetlinewidth{0.50}
\draw[gp path] (1.504,14.779)--(1.504,19.767);
\gpcolor{color=gp lt color border}
\gpsetlinetype{gp lt border}
\gpsetdashtype{gp dt solid}
\gpsetlinewidth{1.00}
\draw[gp path] (1.504,14.779)--(1.504,14.959);
\draw[gp path] (1.504,19.767)--(1.504,19.587);
\node[gp node center] at (1.504,14.471) {$-20$};
\gpcolor{color=gp lt color axes}
\gpsetlinetype{gp lt axes}
\gpsetdashtype{gp dt axes}
\gpsetlinewidth{0.50}
\draw[gp path] (2.247,14.779)--(2.247,19.767);
\gpcolor{color=gp lt color border}
\gpsetlinetype{gp lt border}
\gpsetdashtype{gp dt solid}
\gpsetlinewidth{1.00}
\draw[gp path] (2.247,14.779)--(2.247,14.959);
\draw[gp path] (2.247,19.767)--(2.247,19.587);
\node[gp node center] at (2.247,14.471) {$-15$};
\gpcolor{color=gp lt color axes}
\gpsetlinetype{gp lt axes}
\gpsetdashtype{gp dt axes}
\gpsetlinewidth{0.50}
\draw[gp path] (2.990,14.779)--(2.990,19.767);
\gpcolor{color=gp lt color border}
\gpsetlinetype{gp lt border}
\gpsetdashtype{gp dt solid}
\gpsetlinewidth{1.00}
\draw[gp path] (2.990,14.779)--(2.990,14.959);
\draw[gp path] (2.990,19.767)--(2.990,19.587);
\node[gp node center] at (2.990,14.471) {$-10$};
\gpcolor{color=gp lt color axes}
\gpsetlinetype{gp lt axes}
\gpsetdashtype{gp dt axes}
\gpsetlinewidth{0.50}
\draw[gp path] (3.733,14.779)--(3.733,19.767);
\gpcolor{color=gp lt color border}
\gpsetlinetype{gp lt border}
\gpsetdashtype{gp dt solid}
\gpsetlinewidth{1.00}
\draw[gp path] (3.733,14.779)--(3.733,14.959);
\draw[gp path] (3.733,19.767)--(3.733,19.587);
\node[gp node center] at (3.733,14.471) {$-5$};
\gpcolor{color=gp lt color axes}
\gpsetlinetype{gp lt axes}
\gpsetdashtype{gp dt axes}
\gpsetlinewidth{0.50}
\draw[gp path] (4.476,14.779)--(4.476,19.767);
\gpcolor{color=gp lt color border}
\gpsetlinetype{gp lt border}
\gpsetdashtype{gp dt solid}
\gpsetlinewidth{1.00}
\draw[gp path] (4.476,14.779)--(4.476,14.959);
\draw[gp path] (4.476,19.767)--(4.476,19.587);
\node[gp node center] at (4.476,14.471) {$0$};
\gpcolor{color=gp lt color axes}
\gpsetlinetype{gp lt axes}
\gpsetdashtype{gp dt axes}
\gpsetlinewidth{0.50}
\draw[gp path] (5.218,14.779)--(5.218,19.767);
\gpcolor{color=gp lt color border}
\gpsetlinetype{gp lt border}
\gpsetdashtype{gp dt solid}
\gpsetlinewidth{1.00}
\draw[gp path] (5.218,14.779)--(5.218,14.959);
\draw[gp path] (5.218,19.767)--(5.218,19.587);
\node[gp node center] at (5.218,14.471) {$5$};
\gpcolor{color=gp lt color axes}
\gpsetlinetype{gp lt axes}
\gpsetdashtype{gp dt axes}
\gpsetlinewidth{0.50}
\draw[gp path] (5.961,14.779)--(5.961,19.767);
\gpcolor{color=gp lt color border}
\gpsetlinetype{gp lt border}
\gpsetdashtype{gp dt solid}
\gpsetlinewidth{1.00}
\draw[gp path] (5.961,14.779)--(5.961,14.959);
\draw[gp path] (5.961,19.767)--(5.961,19.587);
\node[gp node center] at (5.961,14.471) {$10$};
\gpcolor{color=gp lt color axes}
\gpsetlinetype{gp lt axes}
\gpsetdashtype{gp dt axes}
\gpsetlinewidth{0.50}
\draw[gp path] (6.704,14.779)--(6.704,19.767);
\gpcolor{color=gp lt color border}
\gpsetlinetype{gp lt border}
\gpsetdashtype{gp dt solid}
\gpsetlinewidth{1.00}
\draw[gp path] (6.704,14.779)--(6.704,14.959);
\draw[gp path] (6.704,19.767)--(6.704,19.587);
\node[gp node center] at (6.704,14.471) {$15$};
\gpcolor{color=gp lt color axes}
\gpsetlinetype{gp lt axes}
\gpsetdashtype{gp dt axes}
\gpsetlinewidth{0.50}
\draw[gp path] (7.447,14.779)--(7.447,19.767);
\gpcolor{color=gp lt color border}
\gpsetlinetype{gp lt border}
\gpsetdashtype{gp dt solid}
\gpsetlinewidth{1.00}
\draw[gp path] (7.447,14.779)--(7.447,14.959);
\draw[gp path] (7.447,19.767)--(7.447,19.587);
\node[gp node center] at (7.447,14.471) {$20$};
\draw[gp path] (1.504,19.767)--(1.504,14.779)--(7.447,14.779)--(7.447,19.767)--cycle;
\node[gp node center,rotate=-270] at (0.246,17.273) {$C_l$};
\node[gp node center] at (4.475,14.009) {$\alpha [\si{deg}]$};
\node[gp node center] at (4.475,20.229) {$C_l \times \alpha$};
\draw[gp path] (1.504,16.012)--(1.511,15.997)--(1.519,15.983)--(1.526,15.969)--(1.549,15.931)%
  --(1.556,15.918)--(1.563,15.907)--(1.571,15.894)--(1.578,15.884)--(1.586,15.873)--(1.593,15.864)%
  --(1.601,15.853)--(1.608,15.844)--(1.615,15.835)--(1.623,15.826)--(1.630,15.818)--(1.638,15.809)%
  --(1.645,15.801)--(1.653,15.794)--(1.660,15.786)--(1.675,15.772)--(1.682,15.765)--(1.690,15.759)%
  --(1.697,15.752)--(1.705,15.746)--(1.712,15.740)--(1.719,15.734)--(1.727,15.729)--(1.734,15.723)%
  --(1.742,15.718)--(1.749,15.713)--(1.757,15.709)--(1.764,15.704)--(1.771,15.699)--(1.779,15.695)%
  --(1.794,15.687)--(1.801,15.683)--(1.809,15.680)--(1.816,15.676)--(1.823,15.673)--(1.831,15.669)%
  --(1.838,15.666)--(1.846,15.663)--(1.853,15.661)--(1.861,15.658)--(1.868,15.655)--(1.875,15.653)%
  --(1.883,15.651)--(1.890,15.649)--(1.898,15.647)--(1.905,15.645)--(1.920,15.641)--(1.927,15.640)%
  --(1.935,15.638)--(1.942,15.637)--(1.950,15.636)--(1.957,15.635)--(1.965,15.634)--(1.972,15.633)%
  --(1.979,15.633)--(1.987,15.632)--(1.994,15.631)--(2.002,15.631)--(2.009,15.630)--(2.017,15.630)%
  --(2.024,15.629)--(2.031,15.629)--(2.039,15.629)--(2.046,15.630)--(2.061,15.629)--(2.069,15.630)%
  --(2.076,15.631)--(2.083,15.632)--(2.091,15.633)--(2.098,15.633)--(2.106,15.634)--(2.113,15.635)%
  --(2.121,15.637)--(2.128,15.639)--(2.135,15.641)--(2.143,15.642)--(2.150,15.644)--(2.158,15.645)%
  --(2.165,15.646)--(2.173,15.648)--(2.180,15.650)--(2.187,15.652)--(2.195,15.653)--(2.202,15.655)%
  --(2.217,15.657)--(2.225,15.659)--(2.232,15.661)--(2.239,15.663)--(2.247,15.666)--(2.254,15.668)%
  --(2.262,15.670)--(2.269,15.672)--(2.277,15.674)--(2.284,15.676)--(2.291,15.679)--(2.299,15.681)%
  --(2.306,15.684)--(2.314,15.687)--(2.321,15.690)--(2.329,15.692)--(2.336,15.695)--(2.343,15.698)%
  --(2.351,15.700)--(2.358,15.703)--(2.366,15.706)--(2.373,15.709)--(2.388,15.716)--(2.395,15.720)%
  --(2.403,15.723)--(2.410,15.727)--(2.418,15.730)--(2.425,15.733)--(2.433,15.737)--(2.440,15.740)%
  --(2.447,15.744)--(2.455,15.748)--(2.462,15.753)--(2.470,15.757)--(2.477,15.761)--(2.485,15.765)%
  --(2.492,15.769)--(2.499,15.773)--(2.507,15.777)--(2.514,15.781)--(2.522,15.785)--(2.529,15.789)%
  --(2.537,15.793)--(2.544,15.798)--(2.551,15.802)--(2.559,15.807)--(2.566,15.812)--(2.574,15.816)%
  --(2.581,15.821)--(2.596,15.830)--(2.603,15.835)--(2.611,15.840)--(2.618,15.844)--(2.626,15.849)%
  --(2.633,15.854)--(2.641,15.859)--(2.648,15.864)--(2.655,15.867)--(2.663,15.870)--(2.670,15.873)%
  --(2.678,15.876)--(2.685,15.879)--(2.693,15.883)--(2.700,15.887)--(2.707,15.891)--(2.715,15.895)%
  --(2.722,15.899)--(2.730,15.904)--(2.737,15.909)--(2.745,15.913)--(2.752,15.918)--(2.759,15.924)%
  --(2.767,15.929)--(2.774,15.934)--(2.782,15.939)--(2.789,15.944)--(2.797,15.950)--(2.804,15.955)%
  --(2.811,15.960)--(2.826,15.971)--(2.834,15.976)--(2.841,15.981)--(2.849,15.987)--(2.856,15.992)%
  --(2.863,15.998)--(2.871,16.004)--(2.878,16.009)--(2.886,16.015)--(2.893,16.021)--(2.901,16.026)%
  --(2.908,16.032)--(2.915,16.037)--(2.923,16.043)--(2.930,16.048)--(2.938,16.054)--(2.945,16.059)%
  --(2.953,16.065)--(2.960,16.070)--(2.967,16.076)--(2.975,16.081)--(2.982,16.087)--(2.990,16.093)%
  --(2.997,16.099)--(3.005,16.104)--(3.012,16.110)--(3.019,16.116)--(3.027,16.121)--(3.034,16.127)%
  --(3.042,16.132)--(3.049,16.138)--(3.057,16.144)--(3.064,16.149)--(3.071,16.155)--(3.079,16.161)%
  --(3.086,16.166)--(3.094,16.172)--(3.101,16.178)--(3.116,16.190)--(3.123,16.196)--(3.131,16.202)%
  --(3.138,16.208)--(3.146,16.214)--(3.153,16.220)--(3.161,16.226)--(3.168,16.232)--(3.175,16.238)%
  --(3.183,16.244)--(3.190,16.249)--(3.198,16.254)--(3.205,16.259)--(3.213,16.265)--(3.220,16.270)%
  --(3.227,16.276)--(3.235,16.281)--(3.242,16.287)--(3.250,16.292)--(3.257,16.298)--(3.265,16.303)%
  --(3.272,16.309)--(3.279,16.315)--(3.287,16.321)--(3.294,16.327)--(3.302,16.333)--(3.309,16.338)%
  --(3.317,16.344)--(3.324,16.350)--(3.331,16.356)--(3.339,16.361)--(3.346,16.367)--(3.354,16.373)%
  --(3.361,16.379)--(3.369,16.385)--(3.376,16.391)--(3.383,16.397)--(3.391,16.402)--(3.398,16.408)%
  --(3.406,16.414)--(3.413,16.419)--(3.421,16.425)--(3.428,16.431)--(3.435,16.437)--(3.443,16.442)%
  --(3.450,16.448)--(3.458,16.454)--(3.465,16.459)--(3.473,16.465)--(3.487,16.476)--(3.495,16.481)%
  --(3.502,16.486)--(3.510,16.491)--(3.517,16.495)--(3.525,16.499)--(3.532,16.504)--(3.539,16.507)%
  --(3.547,16.510)--(3.554,16.514)--(3.562,16.517)--(3.569,16.522)--(3.577,16.527)--(3.584,16.533)%
  --(3.591,16.539)--(3.599,16.546)--(3.606,16.553)--(3.614,16.559)--(3.621,16.566)--(3.629,16.573)%
  --(3.636,16.580)--(3.643,16.587)--(3.651,16.594)--(3.658,16.601)--(3.666,16.608)--(3.673,16.615)%
  --(3.681,16.623)--(3.688,16.630)--(3.695,16.636)--(3.703,16.643)--(3.710,16.650)--(3.718,16.658)%
  --(3.725,16.665)--(3.733,16.672)--(3.740,16.679)--(3.747,16.687)--(3.755,16.694)--(3.762,16.702)%
  --(3.770,16.709)--(3.777,16.717)--(3.785,16.725)--(3.792,16.732)--(3.799,16.739)--(3.807,16.747)%
  --(3.814,16.754)--(3.822,16.762)--(3.829,16.769)--(3.837,16.777)--(3.844,16.785)--(3.851,16.792)%
  --(3.859,16.800)--(3.866,16.807)--(3.874,16.815)--(3.881,16.823)--(3.889,16.830)--(3.896,16.838)%
  --(3.903,16.845)--(3.911,16.853)--(3.918,16.861)--(3.926,16.868)--(3.933,16.875)--(3.941,16.883)%
  --(3.948,16.890)--(3.955,16.897)--(3.963,16.904)--(3.970,16.911)--(3.978,16.918)--(3.985,16.926)%
  --(3.993,16.932)--(4.000,16.940)--(4.007,16.946)--(4.015,16.953)--(4.022,16.960)--(4.030,16.967)%
  --(4.037,16.973)--(4.045,16.980)--(4.052,16.987)--(4.059,16.994)--(4.067,17.001)--(4.074,17.008)%
  --(4.089,17.022)--(4.097,17.030)--(4.104,17.037)--(4.111,17.044)--(4.119,17.052)--(4.126,17.059)%
  --(4.134,17.067)--(4.141,17.074)--(4.149,17.082)--(4.163,17.097)--(4.171,17.105)--(4.178,17.112)%
  --(4.186,17.120)--(4.193,17.127)--(4.201,17.135)--(4.208,17.143)--(4.215,17.150)--(4.223,17.156)%
  --(4.230,17.163)--(4.238,17.171)--(4.245,17.180)--(4.253,17.188)--(4.260,17.197)--(4.267,17.206)%
  --(4.275,17.215)--(4.282,17.225)--(4.290,17.234)--(4.297,17.242)--(4.305,17.251)--(4.312,17.260)%
  --(4.319,17.269)--(4.327,17.279)--(4.334,17.288)--(4.342,17.297)--(4.349,17.306)--(4.357,17.315)%
  --(4.364,17.324)--(4.371,17.333)--(4.379,17.342)--(4.386,17.351)--(4.394,17.360)--(4.401,17.369)%
  --(4.409,17.377)--(4.416,17.386)--(4.423,17.395)--(4.431,17.404)--(4.438,17.413)--(4.446,17.422)%
  --(4.453,17.430)--(4.461,17.439)--(4.468,17.448)--(4.476,17.457)--(4.483,17.466)--(4.490,17.475)%
  --(4.498,17.483)--(4.505,17.492)--(4.513,17.501)--(4.520,17.510)--(4.528,17.519)--(4.535,17.529)%
  --(4.542,17.538)--(4.550,17.547)--(4.557,17.556)--(4.565,17.565)--(4.572,17.574)--(4.580,17.583)%
  --(4.587,17.592)--(4.594,17.601)--(4.602,17.610)--(4.609,17.619)--(4.617,17.628)--(4.624,17.637)%
  --(4.632,17.646)--(4.639,17.655)--(4.646,17.664)--(4.654,17.673)--(4.661,17.682)--(4.669,17.691)%
  --(4.676,17.699)--(4.684,17.708)--(4.691,17.717)--(4.698,17.725)--(4.706,17.734)--(4.713,17.743)%
  --(4.721,17.751)--(4.728,17.760)--(4.736,17.768)--(4.743,17.777)--(4.750,17.786)--(4.758,17.795)%
  --(4.765,17.804)--(4.773,17.814)--(4.780,17.823)--(4.788,17.832)--(4.795,17.841)--(4.802,17.850)%
  --(4.810,17.859)--(4.817,17.868)--(4.825,17.876)--(4.832,17.885)--(4.840,17.894)--(4.847,17.902)%
  --(4.854,17.911)--(4.862,17.919)--(4.869,17.927)--(4.877,17.935)--(4.884,17.944)--(4.892,17.953)%
  --(4.899,17.962)--(4.906,17.970)--(4.914,17.976)--(4.921,17.984)--(4.929,17.989)--(4.936,17.990)%
  --(4.944,17.989)--(4.951,17.988)--(4.958,17.988)--(4.966,17.988)--(4.973,17.988)--(4.981,17.989)%
  --(4.988,17.990)--(4.996,17.990)--(5.003,17.992)--(5.010,17.992)--(5.018,17.991)--(5.025,17.989)%
  --(5.033,17.987)--(5.040,17.985)--(5.048,17.983)--(5.062,17.978)--(5.070,17.977)--(5.077,17.974)%
  --(5.085,17.972)--(5.092,17.970)--(5.100,17.968)--(5.107,17.967)--(5.114,17.965)--(5.129,17.962)%
  --(5.144,17.962)--(5.152,17.963)--(5.159,17.962)--(5.166,17.963)--(5.174,17.964)--(5.181,17.964)%
  --(5.189,17.965)--(5.196,17.965)--(5.204,17.967)--(5.211,17.967)--(5.218,17.968)--(5.226,17.969)%
  --(5.233,17.970)--(5.241,17.973)--(5.248,17.974)--(5.256,17.977)--(5.263,17.978)--(5.270,17.979)%
  --(5.278,17.983)--(5.285,17.984)--(5.293,17.986)--(5.300,17.988)--(5.308,17.989)--(5.315,17.992)%
  --(5.322,17.992)--(5.330,17.996)--(5.337,17.996)--(5.345,18.000)--(5.352,18.001)--(5.360,18.003)%
  --(5.367,18.005)--(5.374,18.007)--(5.382,18.011)--(5.389,18.013)--(5.397,18.017)--(5.404,18.020)%
  --(5.412,18.023)--(5.419,18.028)--(5.426,18.030)--(5.434,18.034)--(5.441,18.037)--(5.449,18.040)%
  --(5.456,18.044)--(5.464,18.047)--(5.471,18.052)--(5.478,18.055)--(5.486,18.060)--(5.493,18.065)%
  --(5.501,18.070)--(5.508,18.075)--(5.516,18.080)--(5.523,18.085)--(5.530,18.090)--(5.538,18.095)%
  --(5.545,18.100)--(5.553,18.105)--(5.560,18.111)--(5.568,18.116)--(5.575,18.122)--(5.582,18.127)%
  --(5.590,18.133)--(5.597,18.138)--(5.605,18.143)--(5.612,18.149)--(5.620,18.155)--(5.627,18.161)%
  --(5.634,18.167)--(5.642,18.172)--(5.649,18.177)--(5.657,18.182)--(5.664,18.187)--(5.672,18.192)%
  --(5.679,18.197)--(5.686,18.203)--(5.694,18.209)--(5.701,18.214)--(5.709,18.220)--(5.716,18.225)%
  --(5.724,18.231)--(5.731,18.236)--(5.738,18.242)--(5.746,18.247)--(5.753,18.252)--(5.761,18.258)%
  --(5.768,18.264)--(5.776,18.269)--(5.783,18.275)--(5.790,18.280)--(5.798,18.285)--(5.805,18.290)%
  --(5.813,18.295)--(5.820,18.300)--(5.828,18.305)--(5.835,18.310)--(5.842,18.315)--(5.850,18.321)%
  --(5.857,18.327)--(5.865,18.332)--(5.872,18.338)--(5.880,18.343)--(5.887,18.349)--(5.894,18.354)%
  --(5.902,18.360)--(5.909,18.365)--(5.917,18.370)--(5.924,18.376)--(5.932,18.380)--(5.939,18.385)%
  --(5.946,18.390)--(5.954,18.395)--(5.961,18.400)--(5.969,18.404)--(5.976,18.409)--(5.984,18.415)%
  --(5.991,18.420)--(5.998,18.425)--(6.006,18.431)--(6.013,18.436)--(6.021,18.442)--(6.028,18.447)%
  --(6.036,18.452)--(6.043,18.458)--(6.050,18.463)--(6.058,18.469)--(6.065,18.474)--(6.073,18.479)%
  --(6.080,18.484)--(6.088,18.488)--(6.095,18.493)--(6.102,18.498)--(6.110,18.503)--(6.117,18.508)%
  --(6.125,18.513)--(6.132,18.518)--(6.140,18.523)--(6.147,18.528)--(6.154,18.533)--(6.162,18.538)%
  --(6.169,18.543)--(6.177,18.548)--(6.184,18.553)--(6.192,18.558)--(6.199,18.563)--(6.206,18.568)%
  --(6.214,18.573)--(6.221,18.577)--(6.229,18.582)--(6.251,18.595)--(6.258,18.599)--(6.266,18.604)%
  --(6.273,18.608)--(6.281,18.612)--(6.288,18.617)--(6.296,18.621)--(6.303,18.625)--(6.310,18.629)%
  --(6.318,18.633)--(6.325,18.637)--(6.333,18.641)--(6.340,18.646)--(6.348,18.650)--(6.355,18.654)%
  --(6.362,18.658)--(6.370,18.663)--(6.377,18.667)--(6.385,18.672)--(6.392,18.676)--(6.400,18.681)%
  --(6.407,18.685)--(6.414,18.690)--(6.422,18.695)--(6.429,18.699)--(6.437,18.704)--(6.444,18.708)%
  --(6.452,18.713)--(6.459,18.717)--(6.466,18.721)--(6.481,18.728)--(6.489,18.732)--(6.496,18.736)%
  --(6.504,18.740)--(6.511,18.744)--(6.518,18.748)--(6.526,18.752)--(6.533,18.756)--(6.541,18.760)%
  --(6.548,18.764)--(6.556,18.768)--(6.563,18.772)--(6.570,18.776)--(6.578,18.779)--(6.585,18.782)%
  --(6.593,18.785)--(6.600,18.788)--(6.608,18.792)--(6.615,18.795)--(6.622,18.798)--(6.630,18.802)%
  --(6.637,18.806)--(6.645,18.809)--(6.652,18.813)--(6.660,18.816)--(6.667,18.820)--(6.682,18.825)%
  --(6.689,18.828)--(6.697,18.830)--(6.704,18.833)--(6.712,18.836)--(6.719,18.839)--(6.726,18.842)%
  --(6.734,18.845)--(6.741,18.848)--(6.749,18.851)--(6.756,18.854)--(6.764,18.856)--(6.771,18.858)%
  --(6.778,18.860)--(6.786,18.862)--(6.793,18.865)--(6.801,18.867)--(6.808,18.869)--(6.816,18.872)%
  --(6.823,18.875)--(6.830,18.877)--(6.838,18.879)--(6.845,18.881)--(6.860,18.884)--(6.868,18.886)%
  --(6.875,18.887)--(6.882,18.889)--(6.890,18.891)--(6.897,18.893)--(6.905,18.895)--(6.912,18.897)%
  --(6.920,18.898)--(6.927,18.898)--(6.934,18.900)--(6.942,18.901)--(6.949,18.903)--(6.957,18.904)%
  --(6.964,18.906)--(6.972,18.906)--(6.979,18.907)--(6.986,18.907)--(6.994,18.908)--(7.001,18.909)%
  --(7.009,18.910)--(7.024,18.912)--(7.031,18.912)--(7.038,18.912)--(7.046,18.912)--(7.053,18.913)%
  --(7.061,18.913)--(7.068,18.914)--(7.076,18.913)--(7.083,18.913)--(7.090,18.913)--(7.098,18.913)%
  --(7.105,18.913)--(7.113,18.913)--(7.120,18.912)--(7.128,18.911)--(7.135,18.911)--(7.142,18.911)%
  --(7.150,18.911)--(7.157,18.909)--(7.165,18.908)--(7.180,18.907)--(7.187,18.906)--(7.194,18.905)%
  --(7.202,18.903)--(7.209,18.902)--(7.217,18.901)--(7.224,18.900)--(7.232,18.898)--(7.239,18.896)%
  --(7.246,18.895)--(7.254,18.893)--(7.261,18.891)--(7.269,18.889)--(7.276,18.887)--(7.284,18.885)%
  --(7.291,18.883)--(7.298,18.880)--(7.306,18.878)--(7.328,18.870)--(7.336,18.868)--(7.343,18.866)%
  --(7.350,18.863)--(7.358,18.859)--(7.365,18.856)--(7.373,18.854)--(7.380,18.850)--(7.388,18.847)%
  --(7.395,18.844)--(7.402,18.841)--(7.410,18.837)--(7.417,18.833)--(7.425,18.830)--(7.432,18.826)%
  --(7.447,18.819);
\gpsetdashtype{gp dt 2}
\draw[gp path] (1.504,15.985)--(1.511,15.988)--(1.519,15.990)--(1.526,15.993)--(1.534,15.995)%
  --(1.541,15.997)--(1.549,16.000)--(1.556,16.002)--(1.563,16.004)--(1.571,16.007)--(1.578,15.940)%
  --(1.586,15.939)--(1.593,15.938)--(1.601,15.938)--(1.608,15.937)--(1.615,15.934)--(1.623,15.923)%
  --(1.630,15.919)--(1.638,15.915)--(1.645,15.896)--(1.653,15.889)--(1.660,15.880)--(1.667,15.864)%
  --(1.675,15.857)--(1.682,15.847)--(1.690,15.833)--(1.697,15.825)--(1.705,15.816)--(1.712,15.801)%
  --(1.719,15.792)--(1.727,15.782)--(1.734,15.765)--(1.742,15.753)--(1.749,15.742)--(1.757,15.723)%
  --(1.764,15.709)--(1.771,15.695)--(1.779,15.674)--(1.786,15.659)--(1.794,15.639)--(1.801,15.622)%
  --(1.809,15.602)--(1.816,15.585)--(1.823,15.566)--(1.831,15.549)--(1.838,15.533)--(1.846,15.517)%
  --(1.853,15.502)--(1.861,15.488)--(1.868,15.476)--(1.875,15.464)--(1.883,15.452)--(1.890,15.442)%
  --(1.898,15.431)--(1.905,15.422)--(1.913,15.413)--(1.920,15.404)--(1.927,15.396)--(1.935,15.389)%
  --(1.942,15.381)--(1.950,15.374)--(1.957,15.367)--(1.965,15.361)--(1.972,15.354)--(1.979,15.348)%
  --(1.987,15.342)--(1.994,15.337)--(2.002,15.331)--(2.009,15.326)--(2.017,15.321)--(2.024,15.316)%
  --(2.031,15.311)--(2.039,15.306)--(2.046,15.301)--(2.054,15.297)--(2.061,15.293)--(2.069,15.289)%
  --(2.076,15.285)--(2.083,15.281)--(2.091,15.277)--(2.098,15.274)--(2.106,15.270)--(2.113,15.267)%
  --(2.121,15.264)--(2.128,15.260)--(2.135,15.258)--(2.143,15.254)--(2.150,15.252)--(2.158,15.249)%
  --(2.165,15.246)--(2.173,15.244)--(2.180,15.241)--(2.187,15.239)--(2.195,15.237)--(2.202,15.235)%
  --(2.210,15.233)--(2.217,15.231)--(2.225,15.230)--(2.232,15.228)--(2.239,15.227)--(2.247,15.226)%
  --(2.254,15.225)--(2.262,15.224)--(2.269,15.224)--(2.277,15.223)--(2.284,15.222)--(2.291,15.222)%
  --(2.299,15.222)--(2.306,15.222)--(2.314,15.222)--(2.321,15.222)--(2.329,15.223)--(2.336,15.223)%
  --(2.343,15.224)--(2.351,15.225)--(2.358,15.226)--(2.366,15.227)--(2.373,15.228)--(2.381,15.230)%
  --(2.388,15.232)--(2.395,15.238)--(2.403,15.245)--(2.425,15.244)--(2.433,15.253)--(2.440,15.264)%
  --(2.447,15.276)--(2.455,15.288)--(2.462,15.299)--(2.470,15.312)--(2.477,15.327)--(2.485,15.342)%
  --(2.492,15.358)--(2.499,15.375)--(2.507,15.391)--(2.514,15.406)--(2.522,15.423)--(2.529,15.439)%
  --(2.537,15.445)--(2.544,15.446)--(2.551,15.447)--(2.559,15.449)--(2.566,15.451)--(2.574,15.453)%
  --(2.581,15.455)--(2.589,15.458)--(2.596,15.460)--(2.603,15.462)--(2.611,15.465)--(2.618,15.468)%
  --(2.626,15.471)--(2.633,15.473)--(2.641,15.475)--(2.648,15.478)--(2.655,15.481)--(2.663,15.483)%
  --(2.670,15.485)--(2.678,15.486)--(2.685,15.488)--(2.693,15.489)--(2.700,15.489)--(2.707,15.489)%
  --(2.715,15.490)--(2.722,15.494)--(2.730,15.499)--(2.737,15.504)--(2.745,15.509)--(2.752,15.515)%
  --(2.759,15.520)--(2.767,15.526)--(2.774,15.531)--(2.782,15.537)--(2.789,15.543)--(2.797,15.550)%
  --(2.804,15.556)--(2.811,15.562)--(2.819,15.569)--(2.826,15.575)--(2.834,15.582)--(2.841,15.588)%
  --(2.856,15.602)--(2.863,15.609)--(2.871,15.615)--(2.878,15.620)--(2.886,15.626)--(2.893,15.633)%
  --(2.901,15.640)--(2.908,15.646)--(2.915,15.653)--(2.930,15.668)--(2.938,15.675)--(2.945,15.683)%
  --(2.953,15.690)--(2.960,15.698)--(2.967,15.705)--(2.975,15.713)--(2.982,15.721)--(2.990,15.728)%
  --(2.997,15.736)--(3.005,15.744)--(3.012,15.752)--(3.019,15.759)--(3.027,15.767)--(3.034,15.775)%
  --(3.042,15.783)--(3.049,15.791)--(3.057,15.799)--(3.064,15.807)--(3.071,15.815)--(3.079,15.823)%
  --(3.086,15.831)--(3.101,15.847)--(3.109,15.855)--(3.116,15.864)--(3.123,15.872)--(3.131,15.880)%
  --(3.138,15.888)--(3.146,15.897)--(3.153,15.905)--(3.161,15.913)--(3.168,15.921)--(3.183,15.937)%
  --(3.190,15.945)--(3.198,15.953)--(3.205,15.961)--(3.213,15.969)--(3.220,15.978)--(3.227,15.986)%
  --(3.235,15.994)--(3.242,16.003)--(3.250,16.011)--(3.257,16.019)--(3.272,16.036)--(3.279,16.045)%
  --(3.287,16.053)--(3.294,16.062)--(3.302,16.070)--(3.309,16.079)--(3.317,16.088)--(3.324,16.096)%
  --(3.331,16.105)--(3.339,16.113)--(3.346,16.122)--(3.354,16.130)--(3.369,16.147)--(3.376,16.156)%
  --(3.383,16.165)--(3.391,16.173)--(3.398,16.182)--(3.406,16.190)--(3.413,16.199)--(3.421,16.207)%
  --(3.428,16.216)--(3.435,16.225)--(3.443,16.233)--(3.450,16.242)--(3.465,16.259)--(3.473,16.268)%
  --(3.480,16.276)--(3.487,16.285)--(3.495,16.294)--(3.502,16.302)--(3.510,16.311)--(3.517,16.320)%
  --(3.525,16.329)--(3.532,16.337)--(3.539,16.346)--(3.547,16.355)--(3.554,16.363)--(3.569,16.381)%
  --(3.577,16.390)--(3.584,16.398)--(3.591,16.407)--(3.599,16.416)--(3.606,16.424)--(3.614,16.433)%
  --(3.621,16.442)--(3.629,16.451)--(3.636,16.460)--(3.643,16.468)--(3.651,16.477)--(3.658,16.486)%
  --(3.673,16.503)--(3.681,16.512)--(3.688,16.521)--(3.695,16.530)--(3.703,16.538)--(3.710,16.547)%
  --(3.718,16.556)--(3.725,16.565)--(3.733,16.573)--(3.740,16.582)--(3.747,16.591)--(3.755,16.600)%
  --(3.762,16.608)--(3.770,16.617)--(3.777,16.626)--(3.785,16.635)--(3.792,16.644)--(3.799,16.653)%
  --(3.807,16.661)--(3.814,16.670)--(3.822,16.679)--(3.829,16.688)--(3.837,16.697)--(3.844,16.705)%
  --(3.851,16.714)--(3.859,16.723)--(3.866,16.732)--(3.881,16.749)--(3.889,16.758)--(3.896,16.767)%
  --(3.903,16.776)--(3.911,16.784)--(3.918,16.793)--(3.926,16.802)--(3.933,16.811)--(3.941,16.820)%
  --(3.948,16.828)--(3.955,16.837)--(3.963,16.846)--(3.970,16.855)--(3.978,16.863)--(3.993,16.881)%
  --(4.000,16.890)--(4.007,16.899)--(4.015,16.907)--(4.022,16.916)--(4.030,16.925)--(4.037,16.934)%
  --(4.045,16.943)--(4.052,16.951)--(4.074,16.978)--(4.082,16.986)--(4.089,16.995)--(4.097,17.004)%
  --(4.111,17.021)--(4.119,17.030)--(4.126,17.039)--(4.134,17.048)--(4.141,17.057)--(4.149,17.066)%
  --(4.156,17.074)--(4.171,17.092)--(4.178,17.101)--(4.186,17.110)--(4.193,17.119)--(4.201,17.127)%
  --(4.208,17.136)--(4.215,17.145)--(4.223,17.154)--(4.238,17.172)--(4.245,17.181)--(4.253,17.190)%
  --(4.260,17.199)--(4.267,17.208)--(4.275,17.217)--(4.282,17.226)--(4.290,17.235)--(4.297,17.244)%
  --(4.305,17.253)--(4.312,17.262)--(4.319,17.270)--(4.327,17.279)--(4.334,17.288)--(4.342,17.297)%
  --(4.349,17.306)--(4.357,17.315)--(4.364,17.324)--(4.379,17.342)--(4.386,17.351)--(4.394,17.360)%
  --(4.401,17.369)--(4.409,17.378)--(4.416,17.387)--(4.423,17.395)--(4.431,17.404)--(4.438,17.413)%
  --(4.446,17.422)--(4.453,17.431)--(4.461,17.440)--(4.468,17.449)--(4.476,17.458)--(4.483,17.467)%
  --(4.490,17.475)--(4.498,17.484)--(4.505,17.493)--(4.513,17.502)--(4.520,17.511)--(4.528,17.520)%
  --(4.535,17.529)--(4.542,17.538)--(4.550,17.547)--(4.557,17.556)--(4.565,17.565)--(4.572,17.574)%
  --(4.587,17.591)--(4.594,17.600)--(4.602,17.609)--(4.609,17.618)--(4.617,17.627)--(4.624,17.635)%
  --(4.632,17.644)--(4.639,17.653)--(4.646,17.662)--(4.654,17.671)--(4.661,17.680)--(4.669,17.689)%
  --(4.676,17.698)--(4.684,17.706)--(4.691,17.715)--(4.698,17.724)--(4.706,17.733)--(4.721,17.751)%
  --(4.728,17.760)--(4.736,17.768)--(4.743,17.777)--(4.750,17.786)--(4.758,17.795)--(4.765,17.803)%
  --(4.773,17.812)--(4.780,17.821)--(4.788,17.830)--(4.795,17.838)--(4.802,17.847)--(4.810,17.856)%
  --(4.817,17.865)--(4.825,17.874)--(4.832,17.882)--(4.840,17.891)--(4.854,17.908)--(4.862,17.917)%
  --(4.869,17.926)--(4.877,17.934)--(4.884,17.943)--(4.892,17.951)--(4.899,17.960)--(4.906,17.969)%
  --(4.914,17.977)--(4.921,17.985)--(4.929,17.994)--(4.936,18.002)--(4.944,18.010)--(4.951,18.017)%
  --(4.958,18.025)--(4.973,18.039)--(4.981,18.046)--(4.988,18.054)--(4.996,18.061)--(5.003,18.068)%
  --(5.010,18.075)--(5.018,18.082)--(5.025,18.089)--(5.033,18.097)--(5.040,18.104)--(5.048,18.111)%
  --(5.055,18.119)--(5.062,18.126)--(5.070,18.133)--(5.077,18.140)--(5.085,18.147)--(5.100,18.162)%
  --(5.107,18.169)--(5.122,18.182)--(5.129,18.189)--(5.137,18.196)--(5.144,18.202)--(5.152,18.209)%
  --(5.159,18.217)--(5.166,18.224)--(5.174,18.231)--(5.181,18.238)--(5.204,18.260)--(5.211,18.268)%
  --(5.218,18.275)--(5.226,18.282)--(5.233,18.290)--(5.241,18.297)--(5.248,18.304)--(5.256,18.312)%
  --(5.263,18.319)--(5.270,18.326)--(5.278,18.334)--(5.285,18.342)--(5.300,18.356)--(5.308,18.364)%
  --(5.315,18.371)--(5.322,18.378)--(5.330,18.386)--(5.337,18.393)--(5.345,18.401)--(5.352,18.408)%
  --(5.360,18.416)--(5.367,18.423)--(5.374,18.431)--(5.382,18.438)--(5.397,18.453)--(5.404,18.461)%
  --(5.412,18.469)--(5.419,18.476)--(5.426,18.484)--(5.434,18.491)--(5.441,18.499)--(5.449,18.506)%
  --(5.456,18.514)--(5.464,18.521)--(5.471,18.528)--(5.486,18.543)--(5.493,18.551)--(5.501,18.558)%
  --(5.508,18.566)--(5.516,18.573)--(5.523,18.580)--(5.530,18.587)--(5.538,18.594)--(5.545,18.601)%
  --(5.553,18.608)--(5.560,18.615)--(5.568,18.622)--(5.575,18.630)--(5.582,18.637)--(5.590,18.644)%
  --(5.597,18.651)--(5.605,18.657)--(5.612,18.664)--(5.620,18.670)--(5.627,18.677)--(5.634,18.683)%
  --(5.642,18.690)--(5.649,18.697)--(5.657,18.703)--(5.664,18.710)--(5.672,18.716)--(5.679,18.723)%
  --(5.686,18.728)--(5.694,18.734)--(5.701,18.739)--(5.709,18.743)--(5.716,18.747)--(5.724,18.752)%
  --(5.731,18.757)--(5.738,18.761)--(5.746,18.766)--(5.753,18.771)--(5.761,18.776)--(5.768,18.780)%
  --(5.776,18.785)--(5.783,18.789)--(5.790,18.794)--(5.798,18.798)--(5.805,18.803)--(5.820,18.811)%
  --(5.828,18.816)--(5.835,18.820)--(5.842,18.824)--(5.850,18.829)--(5.857,18.833)--(5.865,18.837)%
  --(5.872,18.841)--(5.880,18.845)--(5.887,18.848)--(5.894,18.852)--(5.902,18.857)--(5.909,18.862)%
  --(5.917,18.866)--(5.924,18.871)--(5.932,18.875)--(5.939,18.879)--(5.954,18.888)--(5.961,18.893)%
  --(5.969,18.897)--(5.976,18.901)--(5.984,18.906)--(5.991,18.910)--(5.998,18.914)--(6.006,18.918)%
  --(6.013,18.922)--(6.021,18.926)--(6.028,18.930)--(6.036,18.934)--(6.043,18.938)--(6.050,18.942)%
  --(6.058,18.946)--(6.065,18.950)--(6.073,18.954)--(6.080,18.958)--(6.088,18.962)--(6.095,18.966)%
  --(6.110,18.973)--(6.117,18.977)--(6.125,18.981)--(6.132,18.984)--(6.140,18.988)--(6.154,18.995)%
  --(6.162,18.998)--(6.169,19.002)--(6.177,19.005)--(6.184,19.009)--(6.192,19.012)--(6.199,19.015)%
  --(6.206,19.018)--(6.214,19.021)--(6.221,19.024)--(6.236,19.028)--(6.244,19.030)--(6.251,19.032)%
  --(6.258,19.033)--(6.266,19.036)--(6.273,19.036)--(6.281,19.038)--(6.288,19.040)--(6.296,19.044)%
  --(6.303,19.048)--(6.310,19.051)--(6.318,19.054)--(6.325,19.057)--(6.333,19.061)--(6.340,19.064)%
  --(6.348,19.068)--(6.355,19.071)--(6.362,19.074)--(6.370,19.077)--(6.377,19.080)--(6.385,19.083)%
  --(6.392,19.086)--(6.400,19.089)--(6.407,19.091)--(6.414,19.094)--(6.422,19.097)--(6.429,19.100)%
  --(6.437,19.103)--(6.444,19.105)--(6.452,19.107)--(6.459,19.110)--(6.466,19.112)--(6.474,19.115)%
  --(6.481,19.118)--(6.489,19.120)--(6.496,19.122)--(6.504,19.124)--(6.511,19.127)--(6.518,19.129)%
  --(6.526,19.131)--(6.533,19.133)--(6.541,19.135)--(6.548,19.137)--(6.556,19.139)--(6.563,19.141)%
  --(6.570,19.143)--(6.578,19.145)--(6.585,19.146)--(6.593,19.148)--(6.600,19.150)--(6.608,19.152)%
  --(6.615,19.154)--(6.622,19.155)--(6.630,19.156)--(6.637,19.158)--(6.645,19.160)--(6.652,19.161)%
  --(6.660,19.163)--(6.667,19.164)--(6.674,19.165)--(6.682,19.166)--(6.689,19.168)--(6.697,19.169)%
  --(6.704,19.170)--(6.712,19.171)--(6.719,19.172)--(6.726,19.173)--(6.734,19.174)--(6.741,19.175)%
  --(6.749,19.175)--(6.756,19.176)--(6.764,19.176)--(6.771,19.177)--(6.778,19.178)--(6.786,19.179)%
  --(6.793,19.179)--(6.801,19.179)--(6.808,19.180)--(6.816,19.180)--(6.823,19.181)--(6.830,19.181)%
  --(6.838,19.180)--(6.845,19.180)--(6.853,19.181)--(6.860,19.181)--(6.868,19.181)--(6.875,19.180)%
  --(6.882,19.180)--(6.890,19.180)--(6.897,19.180)--(6.905,19.180)--(6.912,19.179)--(6.920,19.178)%
  --(6.927,19.177)--(6.934,19.177)--(6.942,19.177)--(6.949,19.176)--(6.957,19.174)--(6.964,19.173)%
  --(6.972,19.172)--(6.979,19.172)--(6.986,19.171)--(6.994,19.169)--(7.001,19.167)--(7.009,19.166)%
  --(7.016,19.165)--(7.024,19.164)--(7.031,19.162)--(7.038,19.160)--(7.046,19.158)--(7.053,19.157)%
  --(7.061,19.155)--(7.068,19.153)--(7.076,19.151)--(7.083,19.149)--(7.090,19.148)--(7.098,19.146)%
  --(7.105,19.144)--(7.113,19.141)--(7.120,19.140)--(7.128,19.138)--(7.135,19.136)--(7.142,19.133)%
  --(7.150,19.130)--(7.157,19.128)--(7.165,19.126)--(7.172,19.123)--(7.180,19.120)--(7.187,19.117)%
  --(7.194,19.115)--(7.202,19.111)--(7.209,19.107)--(7.217,19.104)--(7.224,19.101)--(7.232,19.097)%
  --(7.239,19.093)--(7.246,19.090)--(7.254,19.086)--(7.261,19.081)--(7.269,19.078)--(7.276,19.074)%
  --(7.284,19.069)--(7.291,19.065)--(7.298,19.061)--(7.306,19.056)--(7.313,19.052)--(7.321,19.048)%
  --(7.328,19.043)--(7.336,19.038)--(7.343,19.034)--(7.350,19.029)--(7.358,19.024)--(7.365,19.020)%
  --(7.373,19.014)--(7.380,19.009)--(7.388,19.005)--(7.395,18.999)--(7.402,18.994)--(7.410,18.989)%
  --(7.417,18.983)--(7.425,18.978)--(7.432,18.973)--(7.440,18.967)--(7.447,18.962);
\gpsetdashtype{gp dt 3}
\draw[gp path] (1.504,15.938)--(1.511,15.941)--(1.519,15.943)--(1.526,15.946)--(1.534,15.948)%
  --(1.541,15.950)--(1.549,15.953)--(1.556,15.955)--(1.563,15.958)--(1.571,15.960)--(1.578,15.963)%
  --(1.586,15.965)--(1.593,15.967)--(1.601,15.970)--(1.608,15.972)--(1.615,15.974)--(1.623,15.976)%
  --(1.630,15.978)--(1.638,15.981)--(1.645,15.983)--(1.653,15.984)--(1.660,15.986)--(1.667,15.988)%
  --(1.675,15.990)--(1.697,15.995)--(1.705,15.997)--(1.712,16.000)--(1.719,16.002)--(1.727,16.004)%
  --(1.734,16.007)--(1.742,16.009)--(1.749,16.012)--(1.757,16.015)--(1.764,16.017)--(1.771,16.020)%
  --(1.779,16.022)--(1.786,16.024)--(1.794,16.027)--(1.801,16.029)--(1.809,16.032)--(1.816,16.034)%
  --(1.823,16.037)--(1.831,16.039)--(1.838,16.042)--(1.846,16.044)--(1.853,16.046)--(1.861,16.049)%
  --(1.868,16.051)--(1.875,16.053)--(1.883,16.055)--(1.890,16.057)--(1.898,16.059)--(1.905,16.061)%
  --(1.913,16.063)--(1.920,16.064)--(1.927,16.066)--(1.935,16.068)--(1.942,16.069)--(1.950,16.071)%
  --(1.957,16.072)--(1.965,16.074)--(1.972,16.075)--(1.979,16.077)--(1.987,16.078)--(1.994,16.080)%
  --(2.002,16.081)--(2.009,16.083)--(2.017,16.084)--(2.024,16.086)--(2.031,16.088)--(2.039,16.090)%
  --(2.046,16.092)--(2.054,16.093)--(2.061,16.095)--(2.069,16.095)--(2.076,16.096)--(2.083,16.095)%
  --(2.091,16.095)--(2.098,16.095)--(2.106,16.096)--(2.113,16.097)--(2.121,16.098)--(2.128,16.099)%
  --(2.135,16.100)--(2.150,16.066)--(2.158,16.064)--(2.165,16.063)--(2.173,16.061)--(2.180,16.059)%
  --(2.187,16.058)--(2.195,16.056)--(2.202,16.054)--(2.217,15.994)--(2.225,15.980)--(2.239,15.940)%
  --(2.247,15.929)--(2.254,15.918)--(2.262,15.903)--(2.269,15.884)--(2.277,15.872)--(2.284,15.860)%
  --(2.291,15.848)--(2.299,15.833)--(2.306,15.817)--(2.314,15.804)--(2.321,15.792)--(2.329,15.779)%
  --(2.336,15.765)--(2.343,15.751)--(2.351,15.737)--(2.358,15.724)--(2.366,15.711)--(2.373,15.697)%
  --(2.381,15.683)--(2.388,15.667)--(2.395,15.651)--(2.403,15.641)--(2.410,15.633)--(2.418,15.626)%
  --(2.425,15.620)--(2.433,15.614)--(2.440,15.608)--(2.447,15.603)--(2.455,15.597)--(2.462,15.592)%
  --(2.470,15.587)--(2.477,15.582)--(2.485,15.578)--(2.492,15.573)--(2.499,15.568)--(2.507,15.563)%
  --(2.514,15.558)--(2.522,15.554)--(2.529,15.549)--(2.537,15.544)--(2.544,15.540)--(2.551,15.535)%
  --(2.559,15.531)--(2.566,15.526)--(2.574,15.521)--(2.581,15.517)--(2.589,15.513)--(2.596,15.508)%
  --(2.603,15.504)--(2.611,15.500)--(2.618,15.496)--(2.626,15.491)--(2.633,15.487)--(2.641,15.483)%
  --(2.648,15.479)--(2.655,15.475)--(2.663,15.471)--(2.670,15.466)--(2.678,15.461)--(2.685,15.455)%
  --(2.693,15.452)--(2.700,15.455)--(2.707,15.458)--(2.715,15.461)--(2.722,15.459)--(2.730,15.462)%
  --(2.737,15.468)--(2.745,15.475)--(2.752,15.484)--(2.759,15.494)--(2.767,15.505)--(2.774,15.517)%
  --(2.782,15.528)--(2.789,15.540)--(2.797,15.551)--(2.804,15.559)--(2.811,15.564)--(2.819,15.568)%
  --(2.826,15.572)--(2.834,15.576)--(2.841,15.581)--(2.849,15.586)--(2.856,15.592)--(2.863,15.598)%
  --(2.871,15.604)--(2.878,15.610)--(2.886,15.616)--(2.893,15.623)--(2.901,15.629)--(2.908,15.636)%
  --(2.923,15.650)--(2.930,15.657)--(2.938,15.664)--(2.945,15.671)--(2.953,15.679)--(2.967,15.694)%
  --(2.975,15.701)--(2.982,15.707)--(2.990,15.714)--(2.997,15.721)--(3.005,15.729)--(3.012,15.736)%
  --(3.019,15.744)--(3.027,15.752)--(3.034,15.760)--(3.042,15.768)--(3.049,15.776)--(3.057,15.784)%
  --(3.064,15.793)--(3.071,15.801)--(3.079,15.809)--(3.086,15.817)--(3.094,15.826)--(3.101,15.834)%
  --(3.109,15.842)--(3.116,15.851)--(3.123,15.859)--(3.138,15.876)--(3.146,15.884)--(3.153,15.893)%
  --(3.161,15.901)--(3.168,15.910)--(3.175,15.918)--(3.183,15.927)--(3.190,15.935)--(3.205,15.952)%
  --(3.213,15.961)--(3.220,15.969)--(3.227,15.978)--(3.235,15.987)--(3.242,15.995)--(3.250,16.003)%
  --(3.257,16.012)--(3.265,16.020)--(3.272,16.028)--(3.279,16.037)--(3.287,16.046)--(3.294,16.054)%
  --(3.302,16.063)--(3.309,16.071)--(3.317,16.080)--(3.324,16.089)--(3.331,16.097)--(3.339,16.106)%
  --(3.346,16.115)--(3.354,16.123)--(3.361,16.132)--(3.369,16.141)--(3.376,16.150)--(3.383,16.158)%
  --(3.391,16.167)--(3.398,16.176)--(3.406,16.185)--(3.413,16.193)--(3.421,16.202)--(3.435,16.220)%
  --(3.443,16.228)--(3.450,16.237)--(3.458,16.246)--(3.465,16.254)--(3.473,16.263)--(3.480,16.272)%
  --(3.487,16.280)--(3.495,16.289)--(3.502,16.298)--(3.517,16.315)--(3.525,16.324)--(3.532,16.333)%
  --(3.539,16.342)--(3.547,16.350)--(3.554,16.359)--(3.562,16.368)--(3.569,16.376)--(3.577,16.385)%
  --(3.584,16.394)--(3.591,16.403)--(3.599,16.412)--(3.606,16.420)--(3.614,16.429)--(3.621,16.438)%
  --(3.629,16.447)--(3.636,16.456)--(3.643,16.464)--(3.651,16.473)--(3.658,16.482)--(3.666,16.490)%
  --(3.673,16.499)--(3.681,16.508)--(3.688,16.517)--(3.695,16.525)--(3.703,16.534)--(3.710,16.543)%
  --(3.718,16.552)--(3.725,16.560)--(3.733,16.569)--(3.740,16.578)--(3.747,16.587)--(3.755,16.595)%
  --(3.762,16.604)--(3.770,16.613)--(3.777,16.622)--(3.785,16.631)--(3.792,16.639)--(3.799,16.648)%
  --(3.807,16.657)--(3.814,16.666)--(3.822,16.674)--(3.829,16.683)--(3.837,16.692)--(3.844,16.701)%
  --(3.851,16.709)--(3.859,16.718)--(3.866,16.727)--(3.874,16.736)--(3.881,16.744)--(3.889,16.753)%
  --(3.896,16.762)--(3.903,16.771)--(3.911,16.779)--(3.918,16.788)--(3.926,16.797)--(3.933,16.806)%
  --(3.941,16.814)--(3.948,16.823)--(3.955,16.832)--(3.963,16.841)--(3.970,16.849)--(3.978,16.858)%
  --(3.985,16.867)--(3.993,16.876)--(4.000,16.884)--(4.007,16.893)--(4.015,16.902)--(4.022,16.911)%
  --(4.030,16.919)--(4.037,16.928)--(4.045,16.937)--(4.052,16.946)--(4.059,16.954)--(4.067,16.963)%
  --(4.082,16.980)--(4.089,16.989)--(4.097,16.998)--(4.104,17.006)--(4.111,17.015)--(4.119,17.024)%
  --(4.126,17.033)--(4.134,17.041)--(4.141,17.050)--(4.149,17.059)--(4.163,17.076)--(4.171,17.085)%
  --(4.178,17.094)--(4.186,17.102)--(4.193,17.111)--(4.201,17.120)--(4.208,17.128)--(4.215,17.137)%
  --(4.223,17.146)--(4.230,17.155)--(4.238,17.163)--(4.245,17.172)--(4.253,17.181)--(4.260,17.190)%
  --(4.267,17.198)--(4.275,17.207)--(4.282,17.216)--(4.290,17.224)--(4.297,17.233)--(4.305,17.242)%
  --(4.312,17.251)--(4.319,17.259)--(4.327,17.268)--(4.334,17.277)--(4.342,17.286)--(4.349,17.294)%
  --(4.357,17.303)--(4.364,17.312)--(4.371,17.320)--(4.379,17.329)--(4.386,17.338)--(4.394,17.347)%
  --(4.401,17.356)--(4.409,17.364)--(4.416,17.373)--(4.423,17.382)--(4.438,17.399)--(4.446,17.408)%
  --(4.453,17.417)--(4.461,17.425)--(4.468,17.434)--(4.476,17.443)--(4.483,17.451)--(4.490,17.460)%
  --(4.498,17.469)--(4.505,17.477)--(4.513,17.486)--(4.520,17.495)--(4.528,17.503)--(4.535,17.512)%
  --(4.542,17.521)--(4.550,17.529)--(4.557,17.538)--(4.565,17.547)--(4.572,17.556)--(4.580,17.564)%
  --(4.587,17.573)--(4.594,17.582)--(4.602,17.590)--(4.609,17.599)--(4.617,17.608)--(4.624,17.616)%
  --(4.632,17.625)--(4.639,17.634)--(4.646,17.642)--(4.654,17.651)--(4.661,17.660)--(4.669,17.668)%
  --(4.676,17.677)--(4.684,17.686)--(4.691,17.694)--(4.698,17.703)--(4.706,17.711)--(4.713,17.720)%
  --(4.721,17.729)--(4.728,17.737)--(4.743,17.755)--(4.750,17.763)--(4.758,17.772)--(4.765,17.780)%
  --(4.773,17.788)--(4.780,17.797)--(4.788,17.805)--(4.795,17.814)--(4.802,17.822)--(4.810,17.830)%
  --(4.825,17.846)--(4.832,17.854)--(4.840,17.863)--(4.847,17.871)--(4.854,17.879)--(4.862,17.887)%
  --(4.869,17.895)--(4.877,17.903)--(4.892,17.918)--(4.906,17.935)--(4.914,17.943)--(4.921,17.951)%
  --(4.929,17.958)--(4.936,17.966)--(4.944,17.974)--(4.951,17.982)--(4.958,17.990)--(4.973,18.004)%
  --(4.981,18.012)--(4.988,18.019)--(4.996,18.026)--(5.003,18.034)--(5.010,18.041)--(5.018,18.049)%
  --(5.025,18.056)--(5.033,18.064)--(5.040,18.072)--(5.048,18.080)--(5.055,18.087)--(5.062,18.096)%
  --(5.077,18.111)--(5.085,18.119)--(5.092,18.127)--(5.100,18.134)--(5.107,18.142)--(5.114,18.150)%
  --(5.122,18.158)--(5.129,18.166)--(5.137,18.174)--(5.144,18.182)--(5.152,18.190)--(5.159,18.197)%
  --(5.166,18.205)--(5.174,18.214)--(5.181,18.221)--(5.189,18.229)--(5.196,18.238)--(5.204,18.246)%
  --(5.211,18.254)--(5.218,18.262)--(5.226,18.270)--(5.233,18.278)--(5.241,18.287)--(5.248,18.295)%
  --(5.256,18.303)--(5.263,18.311)--(5.270,18.319)--(5.278,18.327)--(5.285,18.335)--(5.293,18.344)%
  --(5.308,18.360)--(5.315,18.368)--(5.322,18.376)--(5.330,18.385)--(5.337,18.393)--(5.345,18.401)%
  --(5.352,18.409)--(5.360,18.417)--(5.367,18.425)--(5.374,18.433)--(5.382,18.441)--(5.389,18.449)%
  --(5.397,18.457)--(5.404,18.465)--(5.412,18.473)--(5.419,18.481)--(5.426,18.489)--(5.434,18.497)%
  --(5.441,18.505)--(5.449,18.513)--(5.456,18.521)--(5.464,18.529)--(5.471,18.537)--(5.486,18.552)%
  --(5.493,18.560)--(5.501,18.568)--(5.508,18.576)--(5.516,18.584)--(5.523,18.592)--(5.530,18.600)%
  --(5.538,18.607)--(5.545,18.615)--(5.553,18.623)--(5.560,18.631)--(5.568,18.639)--(5.575,18.647)%
  --(5.582,18.655)--(5.590,18.662)--(5.597,18.670)--(5.605,18.678)--(5.620,18.693)--(5.627,18.701)%
  --(5.634,18.709)--(5.642,18.716)--(5.649,18.724)--(5.657,18.731)--(5.664,18.739)--(5.679,18.754)%
  --(5.686,18.761)--(5.694,18.769)--(5.701,18.776)--(5.709,18.783)--(5.716,18.790)--(5.724,18.797)%
  --(5.731,18.804)--(5.738,18.811)--(5.746,18.818)--(5.753,18.825)--(5.761,18.831)--(5.768,18.838)%
  --(5.776,18.844)--(5.783,18.851)--(5.790,18.858)--(5.798,18.866)--(5.805,18.873)--(5.813,18.880)%
  --(5.820,18.887)--(5.828,18.894)--(5.835,18.901)--(5.842,18.907)--(5.850,18.914)--(5.857,18.921)%
  --(5.865,18.927)--(5.872,18.934)--(5.880,18.940)--(5.887,18.947)--(5.894,18.953)--(5.902,18.959)%
  --(5.909,18.965)--(5.917,18.971)--(5.924,18.977)--(5.932,18.983)--(5.939,18.989)--(5.946,18.995)%
  --(5.954,19.000)--(5.961,19.006)--(5.969,19.011)--(5.976,19.016)--(5.984,19.020)--(5.991,19.022)%
  --(6.006,19.027)--(6.013,19.030)--(6.021,19.033)--(6.028,19.037)--(6.036,19.040)--(6.043,19.044)%
  --(6.050,19.047)--(6.058,19.050)--(6.065,19.054)--(6.073,19.057)--(6.080,19.060)--(6.088,19.063)%
  --(6.095,19.067)--(6.102,19.070)--(6.110,19.073)--(6.117,19.076)--(6.125,19.079)--(6.132,19.082)%
  --(6.140,19.085)--(6.147,19.088)--(6.154,19.091)--(6.162,19.094)--(6.169,19.096)--(6.177,19.099)%
  --(6.184,19.102)--(6.192,19.105)--(6.199,19.107)--(6.206,19.110)--(6.214,19.113)--(6.221,19.115)%
  --(6.229,19.118)--(6.236,19.121)--(6.244,19.123)--(6.251,19.125)--(6.258,19.127)--(6.266,19.129)%
  --(6.273,19.131)--(6.281,19.132)--(6.288,19.133)--(6.296,19.134)--(6.303,19.134)--(6.310,19.134)%
  --(6.318,19.134)--(6.325,19.134)--(6.333,19.133)--(6.340,19.131)--(6.348,19.129)--(6.355,19.126)%
  --(6.362,19.127)--(6.370,19.129)--(6.377,19.130)--(6.385,19.132)--(6.392,19.133)--(6.400,19.134)%
  --(6.407,19.135)--(6.414,19.137)--(6.422,19.138)--(6.429,19.139)--(6.437,19.140)--(6.444,19.140)%
  --(6.452,19.141)--(6.459,19.142)--(6.466,19.142)--(6.474,19.142)--(6.481,19.143)--(6.489,19.143)%
  --(6.496,19.143)--(6.504,19.143)--(6.511,19.143)--(6.518,19.142)--(6.526,19.142)--(6.533,19.141)%
  --(6.541,19.141)--(6.548,19.140)--(6.556,19.139)--(6.563,19.138)--(6.570,19.137)--(6.578,19.136)%
  --(6.585,19.135)--(6.593,19.133)--(6.600,19.131)--(6.608,19.130)--(6.615,19.128)--(6.622,19.126)%
  --(6.630,19.124)--(6.637,19.122)--(6.645,19.120)--(6.652,19.118)--(6.660,19.115)--(6.667,19.113)%
  --(6.674,19.110)--(6.682,19.107)--(6.689,19.105)--(6.697,19.102)--(6.704,19.099)--(6.712,19.096)%
  --(6.719,19.093)--(6.726,19.089)--(6.734,19.086)--(6.741,19.083)--(6.749,19.079)--(6.756,19.076)%
  --(6.764,19.072)--(6.771,19.069)--(6.778,19.065)--(6.786,19.061)--(6.793,19.057)--(6.801,19.053)%
  --(6.808,19.050)--(6.816,19.046)--(6.823,19.042)--(6.830,19.038)--(6.838,19.034)--(6.845,19.030)%
  --(6.853,19.026)--(6.860,19.022)--(6.868,19.018)--(6.875,19.013)--(6.882,19.009)--(6.890,19.005)%
  --(6.897,19.000)--(6.905,18.996)--(6.912,18.991)--(6.920,18.987)--(6.927,18.982)--(6.934,18.978)%
  --(6.942,18.973)--(6.949,18.968)--(6.957,18.964)--(6.964,18.959)--(6.972,18.954)--(6.979,18.949)%
  --(6.986,18.944)--(6.994,18.939)--(7.001,18.934)--(7.009,18.929)--(7.016,18.924)--(7.024,18.919)%
  --(7.031,18.914)--(7.038,18.909)--(7.046,18.903)--(7.053,18.898)--(7.061,18.893)--(7.068,18.888)%
  --(7.076,18.882)--(7.083,18.877)--(7.090,18.871)--(7.098,18.866)--(7.105,18.860)--(7.113,18.855)%
  --(7.120,18.849)--(7.128,18.843)--(7.135,18.837)--(7.142,18.832)--(7.150,18.826)--(7.157,18.820)%
  --(7.165,18.814)--(7.172,18.808)--(7.180,18.802)--(7.187,18.796)--(7.194,18.789)--(7.202,18.784)%
  --(7.209,18.777)--(7.217,18.771)--(7.224,18.765)--(7.232,18.758)--(7.239,18.752)--(7.246,18.745)%
  --(7.254,18.739)--(7.261,18.733)--(7.269,18.726)--(7.276,18.720)--(7.284,18.714)--(7.291,18.707)%
  --(7.298,18.701)--(7.306,18.696)--(7.313,18.688)--(7.321,18.682)--(7.328,18.677)--(7.336,18.672)%
  --(7.343,18.664)--(7.350,18.658)--(7.358,18.653)--(7.365,18.649)--(7.373,18.641)--(7.380,18.635)%
  --(7.388,18.630)--(7.395,18.626)--(7.402,18.619)--(7.410,18.611)--(7.417,18.607)--(7.425,18.602)%
  --(7.432,18.590)--(7.440,18.585)--(7.447,18.580);
\gpsetdashtype{gp dt 4}
\draw[gp path] (1.504,15.994)--(1.511,15.997)--(1.519,15.999)--(1.526,16.002)--(1.534,16.004)%
  --(1.541,16.007)--(1.549,16.009)--(1.556,16.012)--(1.563,16.014)--(1.571,16.017)--(1.578,16.019)%
  --(1.586,16.022)--(1.593,16.024)--(1.601,16.027)--(1.608,16.029)--(1.615,16.032)--(1.623,16.034)%
  --(1.630,16.037)--(1.638,16.039)--(1.645,16.042)--(1.653,16.044)--(1.660,16.046)--(1.667,16.049)%
  --(1.675,16.051)--(1.682,16.054)--(1.690,16.056)--(1.697,16.058)--(1.705,16.059)--(1.712,16.062)%
  --(1.719,16.064)--(1.727,16.066)--(1.734,16.069)--(1.742,16.071)--(1.749,16.073)--(1.757,16.076)%
  --(1.764,16.078)--(1.771,16.080)--(1.779,16.082)--(1.786,16.085)--(1.794,16.087)--(1.801,16.089)%
  --(1.809,16.091)--(1.816,16.093)--(1.823,16.095)--(1.831,16.097)--(1.838,16.099)--(1.846,16.101)%
  --(1.853,16.103)--(1.861,16.104)--(1.868,16.106)--(1.875,16.108)--(1.883,16.110)--(1.890,16.112)%
  --(1.898,16.112)--(1.905,16.114)--(1.913,16.117)--(1.920,16.119)--(1.927,16.121)--(1.935,16.124)%
  --(1.942,16.126)--(1.950,16.129)--(1.957,16.131)--(1.965,16.134)--(1.972,16.136)--(1.979,16.139)%
  --(1.987,16.141)--(1.994,16.144)--(2.002,16.146)--(2.009,16.149)--(2.017,16.151)--(2.024,16.153)%
  --(2.031,16.156)--(2.039,16.158)--(2.046,16.160)--(2.054,16.162)--(2.061,16.165)--(2.069,16.167)%
  --(2.076,16.169)--(2.083,16.171)--(2.091,16.173)--(2.098,16.174)--(2.106,16.176)--(2.113,16.177)%
  --(2.121,16.179)--(2.128,16.180)--(2.135,16.181)--(2.143,16.181)--(2.150,16.182)--(2.158,16.183)%
  --(2.165,16.184)--(2.173,16.185)--(2.180,16.186)--(2.187,16.187)--(2.195,16.189)--(2.202,16.190)%
  --(2.210,16.191)--(2.217,16.193)--(2.225,16.194)--(2.232,16.196)--(2.239,16.198)--(2.247,16.200)%
  --(2.254,16.202)--(2.262,16.203)--(2.269,16.205)--(2.277,16.206)--(2.284,16.207)--(2.291,16.208)%
  --(2.299,16.209)--(2.306,16.210)--(2.314,16.211)--(2.321,16.211)--(2.329,16.212)--(2.336,16.212)%
  --(2.343,16.212)--(2.351,16.212)--(2.358,16.212)--(2.366,16.212)--(2.373,16.212)--(2.381,16.213)%
  --(2.388,16.214)--(2.395,16.214)--(2.403,16.215)--(2.410,16.216)--(2.418,16.216)--(2.425,16.217)%
  --(2.433,16.218)--(2.440,16.218)--(2.447,16.219)--(2.455,16.219)--(2.462,16.220)--(2.470,16.187)%
  --(2.477,16.184)--(2.485,16.182)--(2.492,16.181)--(2.499,16.180)--(2.507,16.179)--(2.514,16.178)%
  --(2.522,16.177)--(2.529,16.177)--(2.537,16.176)--(2.544,16.175)--(2.551,16.174)--(2.559,16.172)%
  --(2.566,16.171)--(2.574,16.169)--(2.581,16.167)--(2.589,16.165)--(2.596,16.162)--(2.603,16.158)%
  --(2.611,16.154)--(2.618,16.147)--(2.626,16.137)--(2.633,16.121)--(2.641,16.105)--(2.648,16.083)%
  --(2.655,16.051)--(2.663,16.022)--(2.670,16.006)--(2.678,15.991)--(2.685,15.973)--(2.693,15.963)%
  --(2.700,15.954)--(2.707,15.946)--(2.715,15.939)--(2.722,15.931)--(2.745,15.909)--(2.752,15.901)%
  --(2.759,15.894)--(2.767,15.886)--(2.774,15.877)--(2.782,15.867)--(2.789,15.857)--(2.797,15.844)%
  --(2.804,15.833)--(2.811,15.823)--(2.819,15.813)--(2.826,15.801)--(2.834,15.790)--(2.841,15.779)%
  --(2.849,15.767)--(2.856,15.755)--(2.863,15.749)--(2.871,15.745)--(2.878,15.742)--(2.886,15.740)%
  --(2.893,15.740)--(2.901,15.740)--(2.908,15.741)--(2.915,15.743)--(2.923,15.745)--(2.930,15.748)%
  --(2.938,15.752)--(2.945,15.756)--(2.953,15.760)--(2.960,15.764)--(2.967,15.769)--(2.975,15.774)%
  --(2.982,15.779)--(2.990,15.785)--(2.997,15.791)--(3.005,15.797)--(3.012,15.803)--(3.019,15.810)%
  --(3.027,15.816)--(3.034,15.823)--(3.049,15.837)--(3.057,15.844)--(3.064,15.851)--(3.071,15.859)%
  --(3.079,15.867)--(3.086,15.874)--(3.094,15.878)--(3.101,15.884)--(3.109,15.891)--(3.123,15.905)%
  --(3.131,15.913)--(3.138,15.921)--(3.146,15.929)--(3.153,15.937)--(3.161,15.944)--(3.168,15.952)%
  --(3.175,15.960)--(3.183,15.969)--(3.190,15.977)--(3.198,15.985)--(3.205,15.993)--(3.213,16.001)%
  --(3.220,16.009)--(3.227,16.017)--(3.235,16.026)--(3.242,16.034)--(3.250,16.042)--(3.257,16.050)%
  --(3.265,16.058)--(3.272,16.066)--(3.279,16.075)--(3.287,16.083)--(3.294,16.091)--(3.302,16.100)%
  --(3.309,16.108)--(3.317,16.116)--(3.324,16.125)--(3.331,16.133)--(3.339,16.141)--(3.346,16.150)%
  --(3.354,16.158)--(3.361,16.166)--(3.369,16.175)--(3.376,16.183)--(3.383,16.192)--(3.391,16.199)%
  --(3.398,16.207)--(3.406,16.215)--(3.413,16.223)--(3.421,16.231)--(3.428,16.240)--(3.435,16.248)%
  --(3.443,16.256)--(3.450,16.265)--(3.458,16.274)--(3.465,16.282)--(3.473,16.291)--(3.480,16.299)%
  --(3.487,16.308)--(3.495,16.317)--(3.502,16.325)--(3.510,16.334)--(3.517,16.342)--(3.525,16.351)%
  --(3.532,16.359)--(3.539,16.368)--(3.547,16.376)--(3.554,16.385)--(3.562,16.393)--(3.569,16.402)%
  --(3.577,16.411)--(3.584,16.419)--(3.591,16.428)--(3.599,16.436)--(3.606,16.445)--(3.614,16.454)%
  --(3.621,16.462)--(3.629,16.471)--(3.636,16.479)--(3.643,16.488)--(3.651,16.496)--(3.658,16.505)%
  --(3.666,16.513)--(3.673,16.522)--(3.681,16.530)--(3.688,16.539)--(3.695,16.547)--(3.703,16.556)%
  --(3.710,16.564)--(3.718,16.573)--(3.725,16.582)--(3.733,16.590)--(3.747,16.607)--(3.755,16.616)%
  --(3.762,16.625)--(3.770,16.633)--(3.777,16.642)--(3.785,16.651)--(3.792,16.659)--(3.799,16.668)%
  --(3.807,16.677)--(3.814,16.685)--(3.822,16.694)--(3.829,16.703)--(3.837,16.711)--(3.844,16.720)%
  --(3.859,16.737)--(3.866,16.746)--(3.874,16.755)--(3.881,16.763)--(3.889,16.772)--(3.896,16.780)%
  --(3.903,16.789)--(3.911,16.798)--(3.918,16.806)--(3.926,16.815)--(3.933,16.824)--(3.941,16.832)%
  --(3.948,16.841)--(3.955,16.849)--(3.970,16.867)--(3.978,16.876)--(3.985,16.884)--(3.993,16.893)%
  --(4.000,16.902)--(4.007,16.910)--(4.015,16.919)--(4.022,16.928)--(4.030,16.936)--(4.037,16.945)%
  --(4.045,16.954)--(4.052,16.962)--(4.059,16.971)--(4.067,16.980)--(4.074,16.988)--(4.089,17.006)%
  --(4.097,17.014)--(4.104,17.023)--(4.111,17.032)--(4.119,17.040)--(4.126,17.049)--(4.134,17.058)%
  --(4.141,17.066)--(4.149,17.075)--(4.156,17.084)--(4.163,17.092)--(4.171,17.101)--(4.178,17.110)%
  --(4.186,17.118)--(4.193,17.127)--(4.208,17.144)--(4.215,17.153)--(4.223,17.161)--(4.230,17.170)%
  --(4.238,17.179)--(4.245,17.187)--(4.253,17.196)--(4.260,17.204)--(4.267,17.213)--(4.275,17.222)%
  --(4.282,17.230)--(4.290,17.239)--(4.297,17.247)--(4.305,17.256)--(4.312,17.264)--(4.319,17.273)%
  --(4.327,17.282)--(4.342,17.299)--(4.349,17.308)--(4.357,17.316)--(4.364,17.325)--(4.371,17.334)%
  --(4.379,17.343)--(4.386,17.351)--(4.394,17.360)--(4.401,17.369)--(4.409,17.377)--(4.416,17.386)%
  --(4.423,17.395)--(4.431,17.403)--(4.438,17.412)--(4.446,17.421)--(4.453,17.429)--(4.461,17.438)%
  --(4.468,17.447)--(4.476,17.456)--(4.483,17.464)--(4.490,17.473)--(4.498,17.481)--(4.505,17.490)%
  --(4.513,17.499)--(4.520,17.507)--(4.528,17.516)--(4.535,17.525)--(4.542,17.534)--(4.550,17.542)%
  --(4.557,17.551)--(4.572,17.569)--(4.580,17.577)--(4.587,17.586)--(4.594,17.595)--(4.602,17.603)%
  --(4.609,17.612)--(4.617,17.620)--(4.624,17.629)--(4.632,17.637)--(4.639,17.646)--(4.646,17.654)%
  --(4.661,17.671)--(4.669,17.680)--(4.676,17.688)--(4.684,17.697)--(4.691,17.705)--(4.698,17.713)%
  --(4.706,17.722)--(4.713,17.730)--(4.721,17.738)--(4.728,17.747)--(4.736,17.755)--(4.743,17.763)%
  --(4.750,17.771)--(4.758,17.779)--(4.765,17.787)--(4.780,17.802)--(4.788,17.809)--(4.795,17.815)%
  --(4.802,17.822)--(4.810,17.828)--(4.817,17.834)--(4.825,17.840)--(4.832,17.846)--(4.840,17.852)%
  --(4.847,17.858)--(4.854,17.864)--(4.862,17.871)--(4.869,17.878)--(4.877,17.886)--(4.892,17.900)%
  --(4.899,17.907)--(4.906,17.914)--(4.914,17.921)--(4.921,17.928)--(4.929,17.935)--(4.936,17.942)%
  --(4.944,17.949)--(4.951,17.955)--(4.958,17.962)--(4.966,17.970)--(4.973,17.977)--(4.981,17.985)%
  --(4.988,17.992)--(5.003,18.007)--(5.010,18.014)--(5.018,18.022)--(5.025,18.030)--(5.033,18.038)%
  --(5.040,18.046)--(5.048,18.054)--(5.055,18.062)--(5.062,18.071)--(5.070,18.079)--(5.077,18.087)%
  --(5.085,18.095)--(5.092,18.103)--(5.100,18.111)--(5.107,18.119)--(5.114,18.127)--(5.122,18.135)%
  --(5.129,18.143)--(5.137,18.151)--(5.144,18.159)--(5.152,18.168)--(5.159,18.176)--(5.166,18.184)%
  --(5.174,18.192)--(5.181,18.199)--(5.189,18.207)--(5.196,18.215)--(5.211,18.231)--(5.218,18.239)%
  --(5.226,18.247)--(5.233,18.255)--(5.241,18.263)--(5.248,18.271)--(5.256,18.279)--(5.263,18.286)%
  --(5.270,18.294)--(5.278,18.302)--(5.285,18.310)--(5.293,18.317)--(5.308,18.333)--(5.315,18.341)%
  --(5.322,18.349)--(5.330,18.357)--(5.337,18.364)--(5.345,18.372)--(5.352,18.380)--(5.360,18.388)%
  --(5.367,18.396)--(5.374,18.403)--(5.382,18.411)--(5.397,18.426)--(5.404,18.434)--(5.412,18.442)%
  --(5.419,18.449)--(5.426,18.457)--(5.434,18.464)--(5.441,18.472)--(5.449,18.479)--(5.456,18.487)%
  --(5.464,18.494)--(5.478,18.508)--(5.486,18.515)--(5.493,18.523)--(5.501,18.529)--(5.508,18.536)%
  --(5.516,18.543)--(5.523,18.549)--(5.530,18.556)--(5.538,18.563)--(5.545,18.570)--(5.553,18.577)%
  --(5.560,18.585)--(5.568,18.592)--(5.575,18.599)--(5.582,18.606)--(5.590,18.613)--(5.597,18.620)%
  --(5.605,18.627)--(5.620,18.641)--(5.627,18.647)--(5.634,18.654)--(5.642,18.661)--(5.649,18.668)%
  --(5.657,18.674)--(5.664,18.681)--(5.672,18.688)--(5.679,18.694)--(5.686,18.701)--(5.694,18.708)%
  --(5.701,18.714)--(5.709,18.720)--(5.716,18.727)--(5.724,18.733)--(5.731,18.739)--(5.738,18.745)%
  --(5.746,18.751)--(5.753,18.756)--(5.761,18.761)--(5.768,18.766)--(5.776,18.770)--(5.783,18.775)%
  --(5.790,18.779)--(5.798,18.784)--(5.805,18.788)--(5.813,18.793)--(5.820,18.797)--(5.828,18.802)%
  --(5.835,18.806)--(5.842,18.810)--(5.850,18.815)--(5.857,18.819)--(5.865,18.823)--(5.872,18.828)%
  --(5.880,18.832)--(5.887,18.836)--(5.894,18.840)--(5.902,18.844)--(5.909,18.848)--(5.917,18.852)%
  --(5.924,18.856)--(5.932,18.860)--(5.939,18.864)--(5.946,18.867)--(5.954,18.871)--(5.961,18.874)%
  --(5.969,18.877)--(5.976,18.880)--(5.984,18.883)--(5.991,18.885)--(5.998,18.887)--(6.006,18.889)%
  --(6.013,18.890)--(6.043,18.898)--(6.050,18.900)--(6.058,18.900)--(6.065,18.900)--(6.073,18.899)%
  --(6.080,18.900)--(6.088,18.903)--(6.095,18.907)--(6.102,18.911)--(6.110,18.915)--(6.117,18.919)%
  --(6.125,18.923)--(6.132,18.928)--(6.140,18.932)--(6.147,18.935)--(6.154,18.939)--(6.162,18.942)%
  --(6.169,18.945)--(6.177,18.948)--(6.184,18.951)--(6.192,18.954)--(6.199,18.956)--(6.206,18.959)%
  --(6.214,18.962)--(6.221,18.964)--(6.229,18.966)--(6.236,18.969)--(6.244,18.971)--(6.251,18.973)%
  --(6.258,18.975)--(6.266,18.977)--(6.273,18.978)--(6.281,18.980)--(6.288,18.981)--(6.296,18.983)%
  --(6.303,18.984)--(6.310,18.985)--(6.318,18.986)--(6.325,18.986)--(6.333,18.987)--(6.340,18.987)%
  --(6.348,18.987)--(6.355,18.988)--(6.362,18.987)--(6.370,18.987)--(6.377,18.987)--(6.385,18.986)%
  --(6.392,18.986)--(6.400,18.985)--(6.407,18.984)--(6.414,18.983)--(6.422,18.981)--(6.429,18.980)%
  --(6.437,18.978)--(6.444,18.976)--(6.452,18.974)--(6.459,18.972)--(6.466,18.970)--(6.474,18.967)%
  --(6.481,18.965)--(6.489,18.962)--(6.496,18.960)--(6.504,18.957)--(6.511,18.954)--(6.518,18.951)%
  --(6.526,18.947)--(6.533,18.944)--(6.541,18.940)--(6.548,18.937)--(6.556,18.933)--(6.563,18.930)%
  --(6.570,18.926)--(6.578,18.923)--(6.585,18.919)--(6.593,18.916)--(6.600,18.912)--(6.608,18.909)%
  --(6.615,18.905)--(6.622,18.901)--(6.630,18.898)--(6.637,18.894)--(6.645,18.890)--(6.652,18.887)%
  --(6.660,18.883)--(6.667,18.879)--(6.674,18.875)--(6.682,18.871)--(6.689,18.867)--(6.697,18.863)%
  --(6.704,18.859)--(6.712,18.855)--(6.719,18.851)--(6.726,18.847)--(6.734,18.842)--(6.741,18.838)%
  --(6.749,18.834)--(6.756,18.829)--(6.764,18.825)--(6.771,18.820)--(6.778,18.816)--(6.786,18.811)%
  --(6.793,18.806)--(6.801,18.802)--(6.808,18.797)--(6.816,18.792)--(6.823,18.787)--(6.830,18.783)%
  --(6.838,18.778)--(6.845,18.773)--(6.853,18.768)--(6.860,18.763)--(6.868,18.758)--(6.875,18.753)%
  --(6.882,18.748)--(6.890,18.743)--(6.897,18.738)--(6.905,18.733)--(6.912,18.728)--(6.920,18.723)%
  --(6.927,18.718)--(6.934,18.712)--(6.942,18.707)--(6.949,18.701)--(6.957,18.695)--(6.964,18.689)%
  --(6.972,18.683)--(6.979,18.676)--(6.986,18.669)--(6.994,18.661)--(7.001,18.653)--(7.009,18.643)%
  --(7.016,18.632)--(7.024,18.619)--(7.031,18.604)--(7.038,18.588)--(7.046,18.570)--(7.053,18.550)%
  --(7.061,18.525)--(7.076,18.413)--(7.083,18.404)--(7.090,18.400)--(7.098,18.398)--(7.105,18.397)%
  --(7.113,18.397)--(7.120,18.397)--(7.128,18.397)--(7.135,18.398)--(7.142,18.398)--(7.150,18.399)%
  --(7.157,18.400)--(7.165,18.400)--(7.172,18.400)--(7.180,18.399)--(7.187,18.375)--(7.194,18.376)%
  --(7.202,18.377)--(7.209,18.378)--(7.217,18.378)--(7.224,18.378)--(7.232,18.377)--(7.239,18.377)%
  --(7.246,18.377)--(7.254,18.378)--(7.261,18.379)--(7.269,18.381)--(7.276,18.382)--(7.284,18.384)%
  --(7.291,18.386)--(7.298,18.387)--(7.306,18.389)--(7.313,18.391)--(7.321,18.393)--(7.328,18.395)%
  --(7.336,18.397)--(7.343,18.399)--(7.350,18.401)--(7.358,18.403)--(7.365,18.405)--(7.373,18.407)%
  --(7.380,18.409)--(7.388,18.411)--(7.395,18.413)--(7.402,18.415)--(7.410,18.417)--(7.417,18.420)%
  --(7.425,18.422)--(7.432,18.424)--(7.440,18.426)--(7.447,18.428);
\gpsetdashtype{gp dt solid}
\draw[gp path] (1.504,19.767)--(1.504,14.779)--(7.447,14.779)--(7.447,19.767)--cycle;
%% coordinates of the plot area
\gpdefrectangularnode{gp plot 1}{\pgfpoint{1.504cm}{14.779cm}}{\pgfpoint{7.447cm}{19.767cm}}
\gpcolor{color=gp lt color axes}
\gpsetlinetype{gp lt axes}
\gpsetdashtype{gp dt axes}
\gpsetlinewidth{0.50}
\draw[gp path] (1.504,7.882)--(7.447,7.882);
\gpcolor{color=gp lt color border}
\gpsetlinetype{gp lt border}
\gpsetdashtype{gp dt solid}
\gpsetlinewidth{1.00}
\draw[gp path] (1.504,7.882)--(1.684,7.882);
\draw[gp path] (7.447,7.882)--(7.267,7.882);
\node[gp node right] at (1.320,7.882) {$-100$};
\gpcolor{color=gp lt color axes}
\gpsetlinetype{gp lt axes}
\gpsetdashtype{gp dt axes}
\gpsetlinewidth{0.50}
\draw[gp path] (1.504,8.713)--(7.447,8.713);
\gpcolor{color=gp lt color border}
\gpsetlinetype{gp lt border}
\gpsetdashtype{gp dt solid}
\gpsetlinewidth{1.00}
\draw[gp path] (1.504,8.713)--(1.684,8.713);
\draw[gp path] (7.447,8.713)--(7.267,8.713);
\node[gp node right] at (1.320,8.713) {$-50$};
\gpcolor{color=gp lt color axes}
\gpsetlinetype{gp lt axes}
\gpsetdashtype{gp dt axes}
\gpsetlinewidth{0.50}
\draw[gp path] (1.504,9.545)--(7.447,9.545);
\gpcolor{color=gp lt color border}
\gpsetlinetype{gp lt border}
\gpsetdashtype{gp dt solid}
\gpsetlinewidth{1.00}
\draw[gp path] (1.504,9.545)--(1.684,9.545);
\draw[gp path] (7.447,9.545)--(7.267,9.545);
\node[gp node right] at (1.320,9.545) {$0$};
\gpcolor{color=gp lt color axes}
\gpsetlinetype{gp lt axes}
\gpsetdashtype{gp dt axes}
\gpsetlinewidth{0.50}
\draw[gp path] (1.504,10.376)--(7.447,10.376);
\gpcolor{color=gp lt color border}
\gpsetlinetype{gp lt border}
\gpsetdashtype{gp dt solid}
\gpsetlinewidth{1.00}
\draw[gp path] (1.504,10.376)--(1.684,10.376);
\draw[gp path] (7.447,10.376)--(7.267,10.376);
\node[gp node right] at (1.320,10.376) {$50$};
\gpcolor{color=gp lt color axes}
\gpsetlinetype{gp lt axes}
\gpsetdashtype{gp dt axes}
\gpsetlinewidth{0.50}
\draw[gp path] (1.504,11.207)--(7.447,11.207);
\gpcolor{color=gp lt color border}
\gpsetlinetype{gp lt border}
\gpsetdashtype{gp dt solid}
\gpsetlinewidth{1.00}
\draw[gp path] (1.504,11.207)--(1.684,11.207);
\draw[gp path] (7.447,11.207)--(7.267,11.207);
\node[gp node right] at (1.320,11.207) {$100$};
\gpcolor{color=gp lt color axes}
\gpsetlinetype{gp lt axes}
\gpsetdashtype{gp dt axes}
\gpsetlinewidth{0.50}
\draw[gp path] (1.504,12.039)--(7.447,12.039);
\gpcolor{color=gp lt color border}
\gpsetlinetype{gp lt border}
\gpsetdashtype{gp dt solid}
\gpsetlinewidth{1.00}
\draw[gp path] (1.504,12.039)--(1.684,12.039);
\draw[gp path] (7.447,12.039)--(7.267,12.039);
\node[gp node right] at (1.320,12.039) {$150$};
\gpcolor{color=gp lt color axes}
\gpsetlinetype{gp lt axes}
\gpsetdashtype{gp dt axes}
\gpsetlinewidth{0.50}
\draw[gp path] (1.504,12.870)--(7.447,12.870);
\gpcolor{color=gp lt color border}
\gpsetlinetype{gp lt border}
\gpsetdashtype{gp dt solid}
\gpsetlinewidth{1.00}
\draw[gp path] (1.504,12.870)--(1.684,12.870);
\draw[gp path] (7.447,12.870)--(7.267,12.870);
\node[gp node right] at (1.320,12.870) {$200$};
\gpcolor{color=gp lt color axes}
\gpsetlinetype{gp lt axes}
\gpsetdashtype{gp dt axes}
\gpsetlinewidth{0.50}
\draw[gp path] (1.504,7.882)--(1.504,12.870);
\gpcolor{color=gp lt color border}
\gpsetlinetype{gp lt border}
\gpsetdashtype{gp dt solid}
\gpsetlinewidth{1.00}
\draw[gp path] (1.504,7.882)--(1.504,8.062);
\draw[gp path] (1.504,12.870)--(1.504,12.690);
\node[gp node center] at (1.504,7.574) {$-20$};
\gpcolor{color=gp lt color axes}
\gpsetlinetype{gp lt axes}
\gpsetdashtype{gp dt axes}
\gpsetlinewidth{0.50}
\draw[gp path] (2.247,7.882)--(2.247,12.870);
\gpcolor{color=gp lt color border}
\gpsetlinetype{gp lt border}
\gpsetdashtype{gp dt solid}
\gpsetlinewidth{1.00}
\draw[gp path] (2.247,7.882)--(2.247,8.062);
\draw[gp path] (2.247,12.870)--(2.247,12.690);
\node[gp node center] at (2.247,7.574) {$-15$};
\gpcolor{color=gp lt color axes}
\gpsetlinetype{gp lt axes}
\gpsetdashtype{gp dt axes}
\gpsetlinewidth{0.50}
\draw[gp path] (2.990,7.882)--(2.990,12.870);
\gpcolor{color=gp lt color border}
\gpsetlinetype{gp lt border}
\gpsetdashtype{gp dt solid}
\gpsetlinewidth{1.00}
\draw[gp path] (2.990,7.882)--(2.990,8.062);
\draw[gp path] (2.990,12.870)--(2.990,12.690);
\node[gp node center] at (2.990,7.574) {$-10$};
\gpcolor{color=gp lt color axes}
\gpsetlinetype{gp lt axes}
\gpsetdashtype{gp dt axes}
\gpsetlinewidth{0.50}
\draw[gp path] (3.733,7.882)--(3.733,12.870);
\gpcolor{color=gp lt color border}
\gpsetlinetype{gp lt border}
\gpsetdashtype{gp dt solid}
\gpsetlinewidth{1.00}
\draw[gp path] (3.733,7.882)--(3.733,8.062);
\draw[gp path] (3.733,12.870)--(3.733,12.690);
\node[gp node center] at (3.733,7.574) {$-5$};
\gpcolor{color=gp lt color axes}
\gpsetlinetype{gp lt axes}
\gpsetdashtype{gp dt axes}
\gpsetlinewidth{0.50}
\draw[gp path] (4.476,7.882)--(4.476,12.870);
\gpcolor{color=gp lt color border}
\gpsetlinetype{gp lt border}
\gpsetdashtype{gp dt solid}
\gpsetlinewidth{1.00}
\draw[gp path] (4.476,7.882)--(4.476,8.062);
\draw[gp path] (4.476,12.870)--(4.476,12.690);
\node[gp node center] at (4.476,7.574) {$0$};
\gpcolor{color=gp lt color axes}
\gpsetlinetype{gp lt axes}
\gpsetdashtype{gp dt axes}
\gpsetlinewidth{0.50}
\draw[gp path] (5.218,7.882)--(5.218,12.870);
\gpcolor{color=gp lt color border}
\gpsetlinetype{gp lt border}
\gpsetdashtype{gp dt solid}
\gpsetlinewidth{1.00}
\draw[gp path] (5.218,7.882)--(5.218,8.062);
\draw[gp path] (5.218,12.870)--(5.218,12.690);
\node[gp node center] at (5.218,7.574) {$5$};
\gpcolor{color=gp lt color axes}
\gpsetlinetype{gp lt axes}
\gpsetdashtype{gp dt axes}
\gpsetlinewidth{0.50}
\draw[gp path] (5.961,7.882)--(5.961,12.870);
\gpcolor{color=gp lt color border}
\gpsetlinetype{gp lt border}
\gpsetdashtype{gp dt solid}
\gpsetlinewidth{1.00}
\draw[gp path] (5.961,7.882)--(5.961,8.062);
\draw[gp path] (5.961,12.870)--(5.961,12.690);
\node[gp node center] at (5.961,7.574) {$10$};
\gpcolor{color=gp lt color axes}
\gpsetlinetype{gp lt axes}
\gpsetdashtype{gp dt axes}
\gpsetlinewidth{0.50}
\draw[gp path] (6.704,7.882)--(6.704,12.870);
\gpcolor{color=gp lt color border}
\gpsetlinetype{gp lt border}
\gpsetdashtype{gp dt solid}
\gpsetlinewidth{1.00}
\draw[gp path] (6.704,7.882)--(6.704,8.062);
\draw[gp path] (6.704,12.870)--(6.704,12.690);
\node[gp node center] at (6.704,7.574) {$15$};
\gpcolor{color=gp lt color axes}
\gpsetlinetype{gp lt axes}
\gpsetdashtype{gp dt axes}
\gpsetlinewidth{0.50}
\draw[gp path] (7.447,7.882)--(7.447,12.870);
\gpcolor{color=gp lt color border}
\gpsetlinetype{gp lt border}
\gpsetdashtype{gp dt solid}
\gpsetlinewidth{1.00}
\draw[gp path] (7.447,7.882)--(7.447,8.062);
\draw[gp path] (7.447,12.870)--(7.447,12.690);
\node[gp node center] at (7.447,7.574) {$20$};
\draw[gp path] (1.504,12.870)--(1.504,7.882)--(7.447,7.882)--(7.447,12.870)--cycle;
\node[gp node center,rotate=-270] at (0.246,10.376) {$C_l/C_d$};
\node[gp node center] at (4.475,7.112) {$\alpha [\si{deg}]$};
\node[gp node center] at (4.475,13.332) {$L/D \times \alpha$};
\draw[gp path] (1.504,9.456)--(1.511,9.453)--(1.519,9.449)--(1.526,9.446)--(1.549,9.436)%
  --(1.556,9.432)--(1.563,9.429)--(1.571,9.425)--(1.578,9.421)--(1.586,9.417)--(1.593,9.414)%
  --(1.601,9.410)--(1.608,9.406)--(1.615,9.403)--(1.623,9.399)--(1.630,9.395)--(1.638,9.391)%
  --(1.645,9.387)--(1.653,9.383)--(1.660,9.379)--(1.675,9.370)--(1.682,9.366)--(1.690,9.362)%
  --(1.697,9.357)--(1.705,9.353)--(1.712,9.348)--(1.719,9.344)--(1.727,9.339)--(1.734,9.335)%
  --(1.742,9.330)--(1.749,9.325)--(1.757,9.321)--(1.764,9.316)--(1.771,9.311)--(1.779,9.306)%
  --(1.794,9.296)--(1.801,9.291)--(1.809,9.286)--(1.816,9.281)--(1.823,9.276)--(1.831,9.271)%
  --(1.838,9.266)--(1.846,9.261)--(1.853,9.256)--(1.861,9.251)--(1.868,9.245)--(1.875,9.240)%
  --(1.883,9.235)--(1.890,9.230)--(1.898,9.225)--(1.905,9.219)--(1.920,9.209)--(1.927,9.204)%
  --(1.935,9.198)--(1.942,9.193)--(1.950,9.188)--(1.957,9.182)--(1.965,9.177)--(1.972,9.172)%
  --(1.979,9.167)--(1.987,9.161)--(1.994,9.156)--(2.002,9.151)--(2.009,9.146)--(2.017,9.140)%
  --(2.024,9.135)--(2.031,9.130)--(2.039,9.125)--(2.046,9.120)--(2.061,9.109)--(2.069,9.103)%
  --(2.076,9.099)--(2.083,9.094)--(2.091,9.089)--(2.098,9.084)--(2.106,9.079)--(2.113,9.074)%
  --(2.121,9.070)--(2.128,9.066)--(2.135,9.062)--(2.143,9.057)--(2.150,9.053)--(2.158,9.048)%
  --(2.165,9.043)--(2.173,9.038)--(2.180,9.034)--(2.187,9.029)--(2.195,9.025)--(2.202,9.019)%
  --(2.217,9.009)--(2.225,9.004)--(2.232,8.999)--(2.239,8.995)--(2.247,8.990)--(2.254,8.985)%
  --(2.262,8.980)--(2.269,8.976)--(2.277,8.971)--(2.284,8.966)--(2.291,8.961)--(2.299,8.957)%
  --(2.306,8.953)--(2.314,8.949)--(2.321,8.944)--(2.329,8.940)--(2.336,8.936)--(2.343,8.931)%
  --(2.351,8.926)--(2.358,8.922)--(2.366,8.918)--(2.373,8.915)--(2.388,8.907)--(2.395,8.904)%
  --(2.403,8.900)--(2.410,8.896)--(2.418,8.893)--(2.425,8.888)--(2.433,8.885)--(2.440,8.881)%
  --(2.447,8.878)--(2.455,8.875)--(2.462,8.872)--(2.470,8.869)--(2.477,8.866)--(2.485,8.864)%
  --(2.492,8.861)--(2.499,8.858)--(2.507,8.855)--(2.514,8.852)--(2.522,8.849)--(2.529,8.847)%
  --(2.537,8.845)--(2.544,8.842)--(2.551,8.840)--(2.559,8.839)--(2.566,8.837)--(2.574,8.835)%
  --(2.581,8.833)--(2.596,8.829)--(2.603,8.828)--(2.611,8.826)--(2.618,8.824)--(2.626,8.822)%
  --(2.633,8.820)--(2.641,8.819)--(2.648,8.817)--(2.655,8.813)--(2.663,8.806)--(2.670,8.800)%
  --(2.678,8.795)--(2.685,8.791)--(2.693,8.787)--(2.700,8.783)--(2.707,8.779)--(2.715,8.776)%
  --(2.722,8.773)--(2.730,8.771)--(2.737,8.769)--(2.745,8.767)--(2.752,8.766)--(2.759,8.764)%
  --(2.767,8.764)--(2.774,8.763)--(2.782,8.762)--(2.789,8.762)--(2.797,8.762)--(2.804,8.761)%
  --(2.811,8.761)--(2.826,8.760)--(2.834,8.760)--(2.841,8.759)--(2.849,8.759)--(2.856,8.759)%
  --(2.863,8.758)--(2.871,8.758)--(2.878,8.759)--(2.886,8.759)--(2.893,8.759)--(2.901,8.759)%
  --(2.908,8.759)--(2.915,8.760)--(2.923,8.760)--(2.930,8.760)--(2.938,8.761)--(2.945,8.761)%
  --(2.953,8.761)--(2.960,8.761)--(2.967,8.762)--(2.975,8.763)--(2.982,8.764)--(2.990,8.765)%
  --(2.997,8.766)--(3.005,8.767)--(3.012,8.769)--(3.019,8.770)--(3.027,8.771)--(3.034,8.772)%
  --(3.042,8.773)--(3.049,8.774)--(3.057,8.776)--(3.064,8.777)--(3.071,8.778)--(3.079,8.780)%
  --(3.086,8.782)--(3.094,8.784)--(3.101,8.787)--(3.116,8.791)--(3.123,8.794)--(3.131,8.796)%
  --(3.138,8.798)--(3.146,8.801)--(3.153,8.803)--(3.161,8.805)--(3.168,8.808)--(3.175,8.811)%
  --(3.183,8.814)--(3.190,8.814)--(3.198,8.814)--(3.205,8.814)--(3.213,8.815)--(3.220,8.817)%
  --(3.227,8.818)--(3.235,8.820)--(3.242,8.822)--(3.250,8.824)--(3.257,8.826)--(3.265,8.828)%
  --(3.272,8.832)--(3.279,8.834)--(3.287,8.837)--(3.294,8.841)--(3.302,8.844)--(3.309,8.848)%
  --(3.317,8.851)--(3.324,8.855)--(3.331,8.858)--(3.339,8.861)--(3.346,8.865)--(3.354,8.869)%
  --(3.361,8.873)--(3.369,8.878)--(3.376,8.881)--(3.383,8.886)--(3.391,8.890)--(3.398,8.894)%
  --(3.406,8.898)--(3.413,8.903)--(3.421,8.907)--(3.428,8.912)--(3.435,8.916)--(3.443,8.921)%
  --(3.450,8.925)--(3.458,8.930)--(3.465,8.934)--(3.473,8.939)--(3.487,8.949)--(3.495,8.954)%
  --(3.502,8.957)--(3.510,8.959)--(3.517,8.962)--(3.525,8.964)--(3.532,8.967)--(3.539,8.969)%
  --(3.547,8.971)--(3.554,8.974)--(3.562,8.978)--(3.569,8.983)--(3.577,8.988)--(3.584,8.995)%
  --(3.591,9.002)--(3.599,9.009)--(3.606,9.017)--(3.614,9.025)--(3.621,9.033)--(3.629,9.041)%
  --(3.636,9.050)--(3.643,9.058)--(3.651,9.067)--(3.658,9.076)--(3.666,9.085)--(3.673,9.095)%
  --(3.681,9.104)--(3.688,9.112)--(3.695,9.121)--(3.703,9.130)--(3.710,9.139)--(3.718,9.149)%
  --(3.725,9.159)--(3.733,9.168)--(3.740,9.178)--(3.747,9.189)--(3.755,9.199)--(3.762,9.209)%
  --(3.770,9.220)--(3.777,9.232)--(3.785,9.243)--(3.792,9.253)--(3.799,9.263)--(3.807,9.274)%
  --(3.814,9.285)--(3.822,9.296)--(3.829,9.308)--(3.837,9.319)--(3.844,9.331)--(3.851,9.343)%
  --(3.859,9.355)--(3.866,9.366)--(3.874,9.378)--(3.881,9.390)--(3.889,9.402)--(3.896,9.414)%
  --(3.903,9.426)--(3.911,9.438)--(3.918,9.451)--(3.926,9.463)--(3.933,9.475)--(3.941,9.487)%
  --(3.948,9.499)--(3.955,9.511)--(3.963,9.523)--(3.970,9.535)--(3.978,9.547)--(3.985,9.560)%
  --(3.993,9.573)--(4.000,9.586)--(4.007,9.598)--(4.015,9.611)--(4.022,9.624)--(4.030,9.638)%
  --(4.037,9.651)--(4.045,9.666)--(4.052,9.681)--(4.059,9.696)--(4.067,9.711)--(4.074,9.727)%
  --(4.089,9.762)--(4.097,9.779)--(4.104,9.798)--(4.111,9.817)--(4.119,9.836)--(4.126,9.857)%
  --(4.134,9.877)--(4.141,9.899)--(4.149,9.921)--(4.163,9.969)--(4.171,9.993)--(4.178,10.018)%
  --(4.186,10.046)--(4.193,10.075)--(4.201,10.105)--(4.208,10.139)--(4.215,10.177)--(4.223,10.222)%
  --(4.230,10.280)--(4.238,10.319)--(4.245,10.350)--(4.253,10.379)--(4.260,10.411)--(4.267,10.438)%
  --(4.275,10.467)--(4.282,10.495)--(4.290,10.525)--(4.297,10.555)--(4.305,10.583)--(4.312,10.614)%
  --(4.319,10.642)--(4.327,10.670)--(4.334,10.699)--(4.342,10.724)--(4.349,10.752)--(4.357,10.780)%
  --(4.364,10.805)--(4.371,10.832)--(4.379,10.857)--(4.386,10.884)--(4.394,10.912)--(4.401,10.939)%
  --(4.409,10.967)--(4.416,10.994)--(4.423,11.021)--(4.431,11.049)--(4.438,11.076)--(4.446,11.103)%
  --(4.453,11.126)--(4.461,11.153)--(4.468,11.176)--(4.476,11.199)--(4.483,11.226)--(4.490,11.248)%
  --(4.498,11.270)--(4.505,11.293)--(4.513,11.320)--(4.520,11.348)--(4.528,11.376)--(4.535,11.399)%
  --(4.542,11.426)--(4.550,11.454)--(4.557,11.482)--(4.565,11.509)--(4.572,11.537)--(4.580,11.559)%
  --(4.587,11.586)--(4.594,11.614)--(4.602,11.636)--(4.609,11.669)--(4.617,11.696)--(4.624,11.723)%
  --(4.632,11.756)--(4.639,11.783)--(4.646,11.810)--(4.654,11.831)--(4.661,11.857)--(4.669,11.884)%
  --(4.676,11.904)--(4.684,11.930)--(4.691,11.950)--(4.698,11.976)--(4.706,11.995)--(4.713,12.015)%
  --(4.721,12.034)--(4.728,12.054)--(4.736,12.079)--(4.743,12.099)--(4.750,12.126)--(4.758,12.153)%
  --(4.765,12.180)--(4.773,12.207)--(4.780,12.234)--(4.788,12.261)--(4.795,12.287)--(4.802,12.315)%
  --(4.810,12.341)--(4.817,12.367)--(4.825,12.386)--(4.832,12.412)--(4.840,12.430)--(4.847,12.448)%
  --(4.854,12.466)--(4.862,12.483)--(4.869,12.492)--(4.877,12.508)--(4.884,12.527)--(4.892,12.553)%
  --(4.899,12.570)--(4.906,12.578)--(4.914,12.567)--(4.921,12.567)--(4.929,12.529)--(4.936,12.444)%
  --(4.944,12.325)--(4.951,12.215)--(4.958,12.126)--(4.966,12.054)--(4.973,11.991)--(4.981,11.942)%
  --(4.988,11.889)--(4.996,11.838)--(5.003,11.804)--(5.010,11.768)--(5.018,11.739)--(5.025,11.708)%
  --(5.033,11.672)--(5.040,11.633)--(5.048,11.590)--(5.062,11.501)--(5.070,11.465)--(5.077,11.419)%
  --(5.085,11.383)--(5.092,11.344)--(5.100,11.305)--(5.107,11.270)--(5.114,11.234)--(5.129,11.167)%
  --(5.144,11.112)--(5.152,11.089)--(5.159,11.062)--(5.166,11.039)--(5.174,11.017)--(5.181,10.992)%
  --(5.189,10.974)--(5.196,10.947)--(5.204,10.933)--(5.211,10.908)--(5.218,10.890)--(5.226,10.869)%
  --(5.233,10.852)--(5.241,10.839)--(5.248,10.818)--(5.256,10.808)--(5.263,10.792)--(5.270,10.775)%
  --(5.278,10.765)--(5.285,10.747)--(5.293,10.737)--(5.300,10.722)--(5.308,10.706)--(5.315,10.693)%
  --(5.322,10.675)--(5.330,10.666)--(5.337,10.649)--(5.345,10.642)--(5.352,10.627)--(5.360,10.616)%
  --(5.367,10.604)--(5.374,10.592)--(5.382,10.584)--(5.389,10.572)--(5.397,10.567)--(5.404,10.559)%
  --(5.412,10.552)--(5.419,10.547)--(5.426,10.536)--(5.434,10.530)--(5.441,10.523)--(5.449,10.515)%
  --(5.456,10.509)--(5.464,10.502)--(5.471,10.499)--(5.478,10.493)--(5.486,10.490)--(5.493,10.488)%
  --(5.501,10.485)--(5.508,10.482)--(5.516,10.480)--(5.523,10.478)--(5.530,10.475)--(5.538,10.472)%
  --(5.545,10.469)--(5.553,10.467)--(5.560,10.466)--(5.568,10.465)--(5.575,10.464)--(5.582,10.462)%
  --(5.590,10.460)--(5.597,10.459)--(5.605,10.457)--(5.612,10.457)--(5.620,10.456)--(5.627,10.456)%
  --(5.634,10.455)--(5.642,10.454)--(5.649,10.452)--(5.657,10.449)--(5.664,10.447)--(5.672,10.444)%
  --(5.679,10.442)--(5.686,10.441)--(5.694,10.440)--(5.701,10.439)--(5.709,10.438)--(5.716,10.437)%
  --(5.724,10.435)--(5.731,10.434)--(5.738,10.431)--(5.746,10.430)--(5.753,10.429)--(5.761,10.427)%
  --(5.768,10.427)--(5.776,10.425)--(5.783,10.423)--(5.790,10.421)--(5.798,10.419)--(5.805,10.416)%
  --(5.813,10.413)--(5.820,10.411)--(5.828,10.408)--(5.835,10.407)--(5.842,10.405)--(5.850,10.404)%
  --(5.857,10.402)--(5.865,10.401)--(5.872,10.399)--(5.880,10.398)--(5.887,10.396)--(5.894,10.395)%
  --(5.902,10.393)--(5.909,10.392)--(5.917,10.390)--(5.924,10.387)--(5.932,10.385)--(5.939,10.382)%
  --(5.946,10.379)--(5.954,10.376)--(5.961,10.373)--(5.969,10.370)--(5.976,10.367)--(5.984,10.365)%
  --(5.991,10.364)--(5.998,10.363)--(6.006,10.361)--(6.013,10.359)--(6.021,10.358)--(6.028,10.356)%
  --(6.036,10.355)--(6.043,10.353)--(6.050,10.352)--(6.058,10.350)--(6.065,10.348)--(6.073,10.346)%
  --(6.080,10.343)--(6.088,10.340)--(6.095,10.337)--(6.102,10.335)--(6.110,10.332)--(6.117,10.330)%
  --(6.125,10.328)--(6.132,10.325)--(6.140,10.323)--(6.147,10.321)--(6.154,10.319)--(6.162,10.317)%
  --(6.169,10.315)--(6.177,10.313)--(6.184,10.310)--(6.192,10.308)--(6.199,10.306)--(6.206,10.303)%
  --(6.214,10.300)--(6.221,10.297)--(6.229,10.294)--(6.251,10.285)--(6.258,10.282)--(6.266,10.279)%
  --(6.273,10.276)--(6.281,10.273)--(6.288,10.270)--(6.296,10.267)--(6.303,10.264)--(6.310,10.260)%
  --(6.318,10.257)--(6.325,10.253)--(6.333,10.250)--(6.340,10.247)--(6.348,10.244)--(6.355,10.241)%
  --(6.362,10.238)--(6.370,10.235)--(6.377,10.232)--(6.385,10.230)--(6.392,10.227)--(6.400,10.225)%
  --(6.407,10.222)--(6.414,10.220)--(6.422,10.217)--(6.429,10.215)--(6.437,10.212)--(6.444,10.210)%
  --(6.452,10.207)--(6.459,10.204)--(6.466,10.201)--(6.481,10.194)--(6.489,10.191)--(6.496,10.188)%
  --(6.504,10.185)--(6.511,10.182)--(6.518,10.179)--(6.526,10.176)--(6.533,10.173)--(6.541,10.171)%
  --(6.548,10.168)--(6.556,10.165)--(6.563,10.162)--(6.570,10.159)--(6.578,10.155)--(6.585,10.151)%
  --(6.593,10.148)--(6.600,10.144)--(6.608,10.141)--(6.615,10.138)--(6.622,10.134)--(6.630,10.131)%
  --(6.637,10.128)--(6.645,10.125)--(6.652,10.122)--(6.660,10.119)--(6.667,10.116)--(6.682,10.108)%
  --(6.689,10.104)--(6.697,10.101)--(6.704,10.097)--(6.712,10.094)--(6.719,10.090)--(6.726,10.087)%
  --(6.734,10.084)--(6.741,10.081)--(6.749,10.078)--(6.756,10.074)--(6.764,10.070)--(6.771,10.066)%
  --(6.778,10.062)--(6.786,10.058)--(6.793,10.055)--(6.801,10.051)--(6.808,10.048)--(6.816,10.044)%
  --(6.823,10.041)--(6.830,10.038)--(6.838,10.034)--(6.845,10.030)--(6.860,10.022)--(6.868,10.019)%
  --(6.875,10.015)--(6.882,10.011)--(6.890,10.008)--(6.897,10.005)--(6.905,10.001)--(6.912,9.998)%
  --(6.920,9.993)--(6.927,9.989)--(6.934,9.986)--(6.942,9.982)--(6.949,9.978)--(6.957,9.975)%
  --(6.964,9.972)--(6.972,9.968)--(6.979,9.964)--(6.986,9.960)--(6.994,9.956)--(7.001,9.952)%
  --(7.009,9.949)--(7.024,9.942)--(7.031,9.938)--(7.038,9.934)--(7.046,9.930)--(7.053,9.927)%
  --(7.061,9.923)--(7.068,9.920)--(7.076,9.916)--(7.083,9.912)--(7.090,9.908)--(7.098,9.905)%
  --(7.105,9.901)--(7.113,9.898)--(7.120,9.894)--(7.128,9.890)--(7.135,9.886)--(7.142,9.883)%
  --(7.150,9.879)--(7.157,9.876)--(7.165,9.872)--(7.180,9.865)--(7.187,9.862)--(7.194,9.858)%
  --(7.202,9.854)--(7.209,9.851)--(7.217,9.848)--(7.224,9.844)--(7.232,9.840)--(7.239,9.837)%
  --(7.246,9.834)--(7.254,9.830)--(7.261,9.827)--(7.269,9.823)--(7.276,9.820)--(7.284,9.817)%
  --(7.291,9.814)--(7.298,9.810)--(7.306,9.807)--(7.328,9.797)--(7.336,9.794)--(7.343,9.791)%
  --(7.350,9.788)--(7.358,9.785)--(7.365,9.782)--(7.373,9.779)--(7.380,9.776)--(7.388,9.773)%
  --(7.395,9.770)--(7.402,9.767)--(7.410,9.764)--(7.417,9.761)--(7.425,9.759)--(7.432,9.756)%
  --(7.447,9.750);
\gpsetdashtype{gp dt 2}
\draw[gp path] (1.504,9.493)--(1.511,9.493)--(1.519,9.493)--(1.526,9.493)--(1.534,9.493)%
  --(1.541,9.493)--(1.549,9.493)--(1.556,9.493)--(1.563,9.493)--(1.571,9.493)--(1.578,9.485)%
  --(1.586,9.484)--(1.593,9.484)--(1.601,9.483)--(1.608,9.483)--(1.615,9.482)--(1.623,9.480)%
  --(1.630,9.479)--(1.638,9.478)--(1.645,9.475)--(1.653,9.474)--(1.660,9.473)--(1.667,9.470)%
  --(1.675,9.468)--(1.682,9.467)--(1.690,9.464)--(1.697,9.462)--(1.705,9.460)--(1.712,9.457)%
  --(1.719,9.455)--(1.727,9.453)--(1.734,9.450)--(1.742,9.447)--(1.749,9.444)--(1.757,9.440)%
  --(1.764,9.437)--(1.771,9.434)--(1.779,9.429)--(1.786,9.425)--(1.794,9.420)--(1.801,9.415)%
  --(1.809,9.409)--(1.816,9.404)--(1.823,9.399)--(1.831,9.393)--(1.838,9.387)--(1.846,9.382)%
  --(1.853,9.376)--(1.861,9.370)--(1.868,9.364)--(1.875,9.358)--(1.883,9.353)--(1.890,9.347)%
  --(1.898,9.341)--(1.905,9.335)--(1.913,9.329)--(1.920,9.324)--(1.927,9.318)--(1.935,9.312)%
  --(1.942,9.306)--(1.950,9.300)--(1.957,9.293)--(1.965,9.287)--(1.972,9.281)--(1.979,9.275)%
  --(1.987,9.268)--(1.994,9.262)--(2.002,9.255)--(2.009,9.248)--(2.017,9.241)--(2.024,9.235)%
  --(2.031,9.228)--(2.039,9.221)--(2.046,9.213)--(2.054,9.206)--(2.061,9.199)--(2.069,9.191)%
  --(2.076,9.184)--(2.083,9.176)--(2.091,9.168)--(2.098,9.160)--(2.106,9.152)--(2.113,9.144)%
  --(2.121,9.135)--(2.128,9.127)--(2.135,9.118)--(2.143,9.109)--(2.150,9.100)--(2.158,9.092)%
  --(2.165,9.082)--(2.173,9.073)--(2.180,9.064)--(2.187,9.054)--(2.195,9.045)--(2.202,9.035)%
  --(2.210,9.025)--(2.217,9.015)--(2.225,9.005)--(2.232,8.995)--(2.239,8.985)--(2.247,8.974)%
  --(2.254,8.964)--(2.262,8.953)--(2.269,8.943)--(2.277,8.931)--(2.284,8.921)--(2.291,8.910)%
  --(2.299,8.899)--(2.306,8.888)--(2.314,8.877)--(2.321,8.865)--(2.329,8.854)--(2.336,8.843)%
  --(2.343,8.832)--(2.351,8.820)--(2.358,8.810)--(2.366,8.799)--(2.373,8.788)--(2.381,8.777)%
  --(2.388,8.767)--(2.395,8.756)--(2.403,8.745)--(2.425,8.712)--(2.433,8.702)--(2.440,8.692)%
  --(2.447,8.682)--(2.455,8.673)--(2.462,8.653)--(2.470,8.633)--(2.477,8.617)--(2.485,8.604)%
  --(2.492,8.594)--(2.499,8.585)--(2.507,8.578)--(2.514,8.571)--(2.522,8.566)--(2.529,8.562)%
  --(2.537,8.557)--(2.544,8.550)--(2.551,8.543)--(2.559,8.535)--(2.566,8.527)--(2.574,8.520)%
  --(2.581,8.512)--(2.589,8.504)--(2.596,8.497)--(2.603,8.489)--(2.611,8.482)--(2.618,8.474)%
  --(2.626,8.466)--(2.633,8.458)--(2.641,8.451)--(2.648,8.443)--(2.655,8.434)--(2.663,8.426)%
  --(2.670,8.418)--(2.678,8.410)--(2.685,8.401)--(2.693,8.392)--(2.700,8.385)--(2.707,8.376)%
  --(2.715,8.371)--(2.722,8.367)--(2.730,8.362)--(2.737,8.357)--(2.745,8.353)--(2.752,8.350)%
  --(2.759,8.346)--(2.767,8.342)--(2.774,8.340)--(2.782,8.337)--(2.789,8.335)--(2.797,8.332)%
  --(2.804,8.329)--(2.811,8.327)--(2.819,8.325)--(2.826,8.324)--(2.834,8.322)--(2.841,8.322)%
  --(2.856,8.319)--(2.863,8.319)--(2.871,8.309)--(2.878,8.301)--(2.886,8.294)--(2.893,8.290)%
  --(2.901,8.287)--(2.908,8.285)--(2.915,8.284)--(2.930,8.281)--(2.938,8.281)--(2.945,8.281)%
  --(2.953,8.280)--(2.960,8.281)--(2.967,8.281)--(2.975,8.281)--(2.982,8.282)--(2.990,8.283)%
  --(2.997,8.285)--(3.005,8.286)--(3.012,8.287)--(3.019,8.288)--(3.027,8.289)--(3.034,8.291)%
  --(3.042,8.292)--(3.049,8.294)--(3.057,8.297)--(3.064,8.299)--(3.071,8.302)--(3.079,8.305)%
  --(3.086,8.307)--(3.101,8.313)--(3.109,8.316)--(3.116,8.320)--(3.123,8.323)--(3.131,8.327)%
  --(3.138,8.330)--(3.146,8.335)--(3.153,8.339)--(3.161,8.344)--(3.168,8.349)--(3.183,8.348)%
  --(3.190,8.349)--(3.198,8.350)--(3.205,8.354)--(3.213,8.356)--(3.220,8.360)--(3.227,8.365)%
  --(3.235,8.370)--(3.242,8.376)--(3.250,8.380)--(3.257,8.385)--(3.272,8.397)--(3.279,8.402)%
  --(3.287,8.408)--(3.294,8.414)--(3.302,8.420)--(3.309,8.426)--(3.317,8.433)--(3.324,8.440)%
  --(3.331,8.447)--(3.339,8.454)--(3.346,8.462)--(3.354,8.469)--(3.369,8.485)--(3.376,8.492)%
  --(3.383,8.500)--(3.391,8.509)--(3.398,8.516)--(3.406,8.521)--(3.413,8.527)--(3.421,8.534)%
  --(3.428,8.542)--(3.435,8.549)--(3.443,8.558)--(3.450,8.565)--(3.465,8.583)--(3.473,8.591)%
  --(3.480,8.600)--(3.487,8.610)--(3.495,8.619)--(3.502,8.628)--(3.510,8.638)--(3.517,8.648)%
  --(3.525,8.658)--(3.532,8.668)--(3.539,8.679)--(3.547,8.690)--(3.554,8.701)--(3.569,8.721)%
  --(3.577,8.730)--(3.584,8.739)--(3.591,8.750)--(3.599,8.760)--(3.606,8.771)--(3.614,8.782)%
  --(3.621,8.792)--(3.629,8.804)--(3.636,8.816)--(3.643,8.828)--(3.651,8.840)--(3.658,8.852)%
  --(3.673,8.876)--(3.681,8.889)--(3.688,8.902)--(3.695,8.912)--(3.703,8.924)--(3.710,8.937)%
  --(3.718,8.948)--(3.725,8.961)--(3.733,8.973)--(3.740,8.986)--(3.747,9.000)--(3.755,9.013)%
  --(3.762,9.027)--(3.770,9.041)--(3.777,9.054)--(3.785,9.068)--(3.792,9.081)--(3.799,9.094)%
  --(3.807,9.107)--(3.814,9.120)--(3.822,9.134)--(3.829,9.148)--(3.837,9.163)--(3.844,9.177)%
  --(3.851,9.191)--(3.859,9.206)--(3.866,9.220)--(3.881,9.248)--(3.889,9.263)--(3.896,9.277)%
  --(3.903,9.292)--(3.911,9.307)--(3.918,9.322)--(3.926,9.336)--(3.933,9.352)--(3.941,9.367)%
  --(3.948,9.382)--(3.955,9.397)--(3.963,9.413)--(3.970,9.429)--(3.978,9.445)--(3.993,9.477)%
  --(4.000,9.493)--(4.007,9.510)--(4.015,9.527)--(4.022,9.544)--(4.030,9.561)--(4.037,9.578)%
  --(4.045,9.596)--(4.052,9.614)--(4.074,9.669)--(4.082,9.689)--(4.089,9.708)--(4.097,9.728)%
  --(4.111,9.770)--(4.119,9.792)--(4.126,9.815)--(4.134,9.839)--(4.141,9.863)--(4.149,9.888)%
  --(4.156,9.913)--(4.171,9.964)--(4.178,9.991)--(4.186,10.018)--(4.193,10.047)--(4.201,10.075)%
  --(4.208,10.099)--(4.215,10.125)--(4.223,10.149)--(4.238,10.196)--(4.245,10.222)--(4.253,10.247)%
  --(4.260,10.270)--(4.267,10.293)--(4.275,10.319)--(4.282,10.344)--(4.290,10.369)--(4.297,10.392)%
  --(4.305,10.417)--(4.312,10.443)--(4.319,10.468)--(4.327,10.491)--(4.334,10.514)--(4.342,10.536)%
  --(4.349,10.559)--(4.357,10.582)--(4.364,10.605)--(4.379,10.654)--(4.386,10.677)--(4.394,10.701)%
  --(4.401,10.724)--(4.409,10.748)--(4.416,10.774)--(4.423,10.797)--(4.431,10.823)--(4.438,10.847)%
  --(4.446,10.870)--(4.453,10.890)--(4.461,10.911)--(4.468,10.930)--(4.476,10.953)--(4.483,10.976)%
  --(4.490,10.996)--(4.498,11.019)--(4.505,11.043)--(4.513,11.066)--(4.520,11.089)--(4.528,11.113)%
  --(4.535,11.136)--(4.542,11.155)--(4.550,11.179)--(4.557,11.198)--(4.565,11.221)--(4.572,11.240)%
  --(4.587,11.282)--(4.594,11.301)--(4.602,11.324)--(4.609,11.343)--(4.617,11.361)--(4.624,11.379)%
  --(4.632,11.402)--(4.639,11.421)--(4.646,11.444)--(4.654,11.462)--(4.661,11.481)--(4.669,11.499)%
  --(4.676,11.516)--(4.684,11.530)--(4.691,11.548)--(4.698,11.561)--(4.706,11.579)--(4.721,11.623)%
  --(4.728,11.646)--(4.736,11.663)--(4.743,11.685)--(4.750,11.703)--(4.758,11.720)--(4.765,11.732)%
  --(4.773,11.744)--(4.780,11.751)--(4.788,11.773)--(4.795,11.790)--(4.802,11.807)--(4.810,11.823)%
  --(4.817,11.840)--(4.825,11.851)--(4.832,11.863)--(4.840,11.869)--(4.854,11.901)--(4.862,11.917)%
  --(4.869,11.928)--(4.877,11.938)--(4.884,11.939)--(4.892,11.949)--(4.899,11.965)--(4.906,11.975)%
  --(4.914,11.970)--(4.921,11.975)--(4.929,11.980)--(4.936,11.969)--(4.944,11.964)--(4.951,11.940)%
  --(4.958,11.925)--(4.973,11.875)--(4.981,11.853)--(4.988,11.836)--(4.996,11.811)--(5.003,11.787)%
  --(5.010,11.772)--(5.018,11.749)--(5.025,11.727)--(5.033,11.717)--(5.040,11.696)--(5.048,11.676)%
  --(5.055,11.663)--(5.062,11.644)--(5.070,11.632)--(5.077,11.617)--(5.085,11.599)--(5.100,11.574)%
  --(5.107,11.557)--(5.122,11.519)--(5.129,11.497)--(5.137,11.485)--(5.144,11.468)--(5.152,11.454)%
  --(5.159,11.445)--(5.166,11.434)--(5.174,11.429)--(5.181,11.421)--(5.204,11.397)--(5.211,11.397)%
  --(5.218,11.391)--(5.226,11.382)--(5.233,11.384)--(5.241,11.379)--(5.248,11.372)--(5.256,11.372)%
  --(5.263,11.369)--(5.270,11.360)--(5.278,11.360)--(5.285,11.357)--(5.300,11.348)--(5.308,11.348)%
  --(5.315,11.344)--(5.322,11.339)--(5.330,11.341)--(5.337,11.341)--(5.345,11.339)--(5.352,11.335)%
  --(5.360,11.335)--(5.367,11.336)--(5.374,11.336)--(5.382,11.334)--(5.397,11.332)--(5.404,11.334)%
  --(5.412,11.335)--(5.419,11.337)--(5.426,11.337)--(5.434,11.337)--(5.441,11.337)--(5.449,11.336)%
  --(5.456,11.334)--(5.464,11.334)--(5.471,11.332)--(5.486,11.335)--(5.493,11.335)--(5.501,11.336)%
  --(5.508,11.336)--(5.516,11.335)--(5.523,11.333)--(5.530,11.333)--(5.538,11.330)--(5.545,11.328)%
  --(5.553,11.325)--(5.560,11.325)--(5.568,11.325)--(5.575,11.326)--(5.582,11.325)--(5.590,11.323)%
  --(5.597,11.322)--(5.605,11.320)--(5.612,11.316)--(5.620,11.313)--(5.627,11.309)--(5.634,11.306)%
  --(5.642,11.304)--(5.649,11.303)--(5.657,11.301)--(5.664,11.298)--(5.672,11.296)--(5.679,11.293)%
  --(5.686,11.289)--(5.694,11.284)--(5.701,11.280)--(5.709,11.276)--(5.716,11.270)--(5.724,11.264)%
  --(5.731,11.261)--(5.738,11.257)--(5.746,11.253)--(5.753,11.249)--(5.761,11.244)--(5.768,11.239)%
  --(5.776,11.234)--(5.783,11.228)--(5.790,11.222)--(5.798,11.215)--(5.805,11.209)--(5.820,11.196)%
  --(5.828,11.188)--(5.835,11.181)--(5.842,11.174)--(5.850,11.166)--(5.857,11.157)--(5.865,11.149)%
  --(5.872,11.140)--(5.880,11.131)--(5.887,11.122)--(5.894,11.113)--(5.902,11.107)--(5.909,11.100)%
  --(5.917,11.093)--(5.924,11.085)--(5.932,11.078)--(5.939,11.070)--(5.954,11.054)--(5.961,11.046)%
  --(5.969,11.038)--(5.976,11.030)--(5.984,11.022)--(5.991,11.013)--(5.998,11.004)--(6.006,10.994)%
  --(6.013,10.985)--(6.021,10.977)--(6.028,10.968)--(6.036,10.958)--(6.043,10.948)--(6.050,10.939)%
  --(6.058,10.930)--(6.065,10.921)--(6.073,10.912)--(6.080,10.902)--(6.088,10.892)--(6.095,10.882)%
  --(6.110,10.863)--(6.117,10.852)--(6.125,10.843)--(6.132,10.833)--(6.140,10.823)--(6.154,10.802)%
  --(6.162,10.791)--(6.169,10.781)--(6.177,10.771)--(6.184,10.761)--(6.192,10.750)--(6.199,10.739)%
  --(6.206,10.728)--(6.214,10.717)--(6.221,10.707)--(6.236,10.685)--(6.244,10.674)--(6.251,10.663)%
  --(6.258,10.653)--(6.266,10.645)--(6.273,10.634)--(6.281,10.620)--(6.288,10.609)--(6.296,10.600)%
  --(6.303,10.591)--(6.310,10.582)--(6.318,10.573)--(6.325,10.563)--(6.333,10.554)--(6.340,10.546)%
  --(6.348,10.537)--(6.355,10.528)--(6.362,10.518)--(6.370,10.509)--(6.377,10.501)--(6.385,10.492)%
  --(6.392,10.484)--(6.400,10.475)--(6.407,10.465)--(6.414,10.456)--(6.422,10.448)--(6.429,10.440)%
  --(6.437,10.431)--(6.444,10.422)--(6.452,10.413)--(6.459,10.404)--(6.466,10.396)--(6.474,10.388)%
  --(6.481,10.380)--(6.489,10.371)--(6.496,10.362)--(6.504,10.354)--(6.511,10.346)--(6.518,10.338)%
  --(6.526,10.330)--(6.533,10.321)--(6.541,10.313)--(6.548,10.305)--(6.556,10.297)--(6.563,10.290)%
  --(6.570,10.282)--(6.578,10.273)--(6.585,10.265)--(6.593,10.258)--(6.600,10.250)--(6.608,10.243)%
  --(6.615,10.236)--(6.622,10.228)--(6.630,10.220)--(6.637,10.213)--(6.645,10.205)--(6.652,10.199)%
  --(6.660,10.191)--(6.667,10.184)--(6.674,10.176)--(6.682,10.169)--(6.689,10.163)--(6.697,10.156)%
  --(6.704,10.149)--(6.712,10.142)--(6.719,10.135)--(6.726,10.128)--(6.734,10.122)--(6.741,10.116)%
  --(6.749,10.109)--(6.756,10.102)--(6.764,10.095)--(6.771,10.089)--(6.778,10.083)--(6.786,10.077)%
  --(6.793,10.071)--(6.801,10.064)--(6.808,10.058)--(6.816,10.052)--(6.823,10.047)--(6.830,10.040)%
  --(6.838,10.034)--(6.845,10.028)--(6.853,10.023)--(6.860,10.017)--(6.868,10.012)--(6.875,10.005)%
  --(6.882,10.000)--(6.890,9.994)--(6.897,9.989)--(6.905,9.984)--(6.912,9.978)--(6.920,9.972)%
  --(6.927,9.967)--(6.934,9.962)--(6.942,9.957)--(6.949,9.952)--(6.957,9.946)--(6.964,9.941)%
  --(6.972,9.936)--(6.979,9.932)--(6.986,9.927)--(6.994,9.922)--(7.001,9.917)--(7.009,9.912)%
  --(7.016,9.907)--(7.024,9.903)--(7.031,9.898)--(7.038,9.893)--(7.046,9.889)--(7.053,9.884)%
  --(7.061,9.880)--(7.068,9.875)--(7.076,9.871)--(7.083,9.867)--(7.090,9.863)--(7.098,9.859)%
  --(7.105,9.855)--(7.113,9.850)--(7.120,9.847)--(7.128,9.843)--(7.135,9.839)--(7.142,9.835)%
  --(7.150,9.831)--(7.157,9.828)--(7.165,9.824)--(7.172,9.820)--(7.180,9.816)--(7.187,9.813)%
  --(7.194,9.809)--(7.202,9.805)--(7.209,9.802)--(7.217,9.798)--(7.224,9.795)--(7.232,9.791)%
  --(7.239,9.788)--(7.246,9.784)--(7.254,9.781)--(7.261,9.777)--(7.269,9.774)--(7.276,9.771)%
  --(7.284,9.768)--(7.291,9.765)--(7.298,9.762)--(7.306,9.758)--(7.313,9.755)--(7.321,9.753)%
  --(7.328,9.749)--(7.336,9.746)--(7.343,9.744)--(7.350,9.741)--(7.358,9.738)--(7.365,9.735)%
  --(7.373,9.733)--(7.380,9.730)--(7.388,9.727)--(7.395,9.725)--(7.402,9.722)--(7.410,9.720)%
  --(7.417,9.717)--(7.425,9.715)--(7.432,9.712)--(7.440,9.710)--(7.447,9.708);
\gpsetdashtype{gp dt 3}
\draw[gp path] (1.504,9.497)--(1.511,9.497)--(1.519,9.497)--(1.526,9.497)--(1.534,9.497)%
  --(1.541,9.497)--(1.549,9.497)--(1.556,9.497)--(1.563,9.497)--(1.571,9.497)--(1.578,9.497)%
  --(1.586,9.497)--(1.593,9.497)--(1.601,9.496)--(1.608,9.496)--(1.615,9.496)--(1.623,9.496)%
  --(1.630,9.496)--(1.638,9.496)--(1.645,9.496)--(1.653,9.496)--(1.660,9.496)--(1.667,9.496)%
  --(1.675,9.495)--(1.697,9.495)--(1.705,9.495)--(1.712,9.495)--(1.719,9.495)--(1.727,9.495)%
  --(1.734,9.495)--(1.742,9.495)--(1.749,9.495)--(1.757,9.495)--(1.764,9.495)--(1.771,9.495)%
  --(1.779,9.495)--(1.786,9.495)--(1.794,9.495)--(1.801,9.495)--(1.809,9.495)--(1.816,9.495)%
  --(1.823,9.495)--(1.831,9.495)--(1.838,9.495)--(1.846,9.495)--(1.853,9.495)--(1.861,9.494)%
  --(1.868,9.494)--(1.875,9.494)--(1.883,9.494)--(1.890,9.494)--(1.898,9.494)--(1.905,9.494)%
  --(1.913,9.494)--(1.920,9.494)--(1.927,9.494)--(1.935,9.493)--(1.942,9.493)--(1.950,9.493)%
  --(1.957,9.493)--(1.965,9.493)--(1.972,9.493)--(1.979,9.492)--(1.987,9.492)--(1.994,9.492)%
  --(2.002,9.492)--(2.009,9.492)--(2.017,9.492)--(2.024,9.491)--(2.031,9.491)--(2.039,9.491)%
  --(2.046,9.491)--(2.054,9.491)--(2.061,9.491)--(2.069,9.491)--(2.076,9.490)--(2.083,9.490)%
  --(2.091,9.489)--(2.098,9.489)--(2.106,9.489)--(2.113,9.488)--(2.121,9.488)--(2.128,9.488)%
  --(2.135,9.488)--(2.150,9.482)--(2.158,9.481)--(2.165,9.480)--(2.173,9.479)--(2.180,9.479)%
  --(2.187,9.478)--(2.195,9.477)--(2.202,9.476)--(2.217,9.465)--(2.225,9.462)--(2.239,9.453)%
  --(2.247,9.450)--(2.254,9.447)--(2.262,9.443)--(2.269,9.438)--(2.277,9.435)--(2.284,9.431)%
  --(2.291,9.427)--(2.299,9.422)--(2.306,9.417)--(2.314,9.412)--(2.321,9.407)--(2.329,9.402)%
  --(2.336,9.396)--(2.343,9.390)--(2.351,9.384)--(2.358,9.378)--(2.366,9.371)--(2.373,9.364)%
  --(2.381,9.355)--(2.388,9.346)--(2.395,9.336)--(2.403,9.326)--(2.410,9.318)--(2.418,9.310)%
  --(2.425,9.302)--(2.433,9.294)--(2.440,9.286)--(2.447,9.278)--(2.455,9.270)--(2.462,9.261)%
  --(2.470,9.253)--(2.477,9.244)--(2.485,9.235)--(2.492,9.226)--(2.499,9.216)--(2.507,9.206)%
  --(2.514,9.196)--(2.522,9.186)--(2.529,9.175)--(2.537,9.164)--(2.544,9.153)--(2.551,9.141)%
  --(2.559,9.129)--(2.566,9.117)--(2.574,9.104)--(2.581,9.091)--(2.589,9.078)--(2.596,9.064)%
  --(2.603,9.050)--(2.611,9.036)--(2.618,9.022)--(2.626,9.007)--(2.633,8.992)--(2.641,8.977)%
  --(2.648,8.962)--(2.655,8.947)--(2.663,8.932)--(2.670,8.918)--(2.678,8.904)--(2.685,8.892)%
  --(2.693,8.875)--(2.700,8.850)--(2.707,8.827)--(2.715,8.804)--(2.722,8.789)--(2.730,8.767)%
  --(2.737,8.745)--(2.745,8.721)--(2.752,8.698)--(2.759,8.675)--(2.767,8.652)--(2.774,8.631)%
  --(2.782,8.611)--(2.789,8.593)--(2.797,8.576)--(2.804,8.563)--(2.811,8.551)--(2.819,8.541)%
  --(2.826,8.531)--(2.834,8.521)--(2.841,8.511)--(2.849,8.502)--(2.856,8.493)--(2.863,8.483)%
  --(2.871,8.475)--(2.878,8.467)--(2.886,8.459)--(2.893,8.452)--(2.901,8.445)--(2.908,8.439)%
  --(2.923,8.426)--(2.930,8.421)--(2.938,8.415)--(2.945,8.411)--(2.953,8.407)--(2.967,8.402)%
  --(2.975,8.393)--(2.982,8.376)--(2.990,8.364)--(2.997,8.356)--(3.005,8.349)--(3.012,8.344)%
  --(3.019,8.341)--(3.027,8.337)--(3.034,8.335)--(3.042,8.332)--(3.049,8.331)--(3.057,8.331)%
  --(3.064,8.330)--(3.071,8.330)--(3.079,8.329)--(3.086,8.329)--(3.094,8.329)--(3.101,8.330)%
  --(3.109,8.330)--(3.116,8.331)--(3.123,8.332)--(3.138,8.335)--(3.146,8.336)--(3.153,8.338)%
  --(3.161,8.339)--(3.168,8.341)--(3.175,8.344)--(3.183,8.346)--(3.190,8.349)--(3.205,8.356)%
  --(3.213,8.359)--(3.220,8.364)--(3.227,8.367)--(3.235,8.372)--(3.242,8.376)--(3.250,8.374)%
  --(3.257,8.370)--(3.265,8.370)--(3.272,8.372)--(3.279,8.374)--(3.287,8.378)--(3.294,8.381)%
  --(3.302,8.386)--(3.309,8.391)--(3.317,8.396)--(3.324,8.402)--(3.331,8.408)--(3.339,8.415)%
  --(3.346,8.421)--(3.354,8.427)--(3.361,8.434)--(3.369,8.441)--(3.376,8.448)--(3.383,8.455)%
  --(3.391,8.462)--(3.398,8.470)--(3.406,8.478)--(3.413,8.487)--(3.421,8.496)--(3.435,8.514)%
  --(3.443,8.523)--(3.450,8.531)--(3.458,8.539)--(3.465,8.541)--(3.473,8.546)--(3.480,8.553)%
  --(3.487,8.560)--(3.495,8.568)--(3.502,8.576)--(3.517,8.595)--(3.525,8.604)--(3.532,8.614)%
  --(3.539,8.624)--(3.547,8.633)--(3.554,8.644)--(3.562,8.654)--(3.569,8.666)--(3.577,8.676)%
  --(3.584,8.688)--(3.591,8.699)--(3.599,8.710)--(3.606,8.722)--(3.614,8.733)--(3.621,8.746)%
  --(3.629,8.758)--(3.636,8.771)--(3.643,8.782)--(3.651,8.791)--(3.658,8.803)--(3.666,8.814)%
  --(3.673,8.825)--(3.681,8.837)--(3.688,8.849)--(3.695,8.861)--(3.703,8.874)--(3.710,8.887)%
  --(3.718,8.900)--(3.725,8.914)--(3.733,8.927)--(3.740,8.941)--(3.747,8.955)--(3.755,8.969)%
  --(3.762,8.983)--(3.770,8.997)--(3.777,9.011)--(3.785,9.026)--(3.792,9.039)--(3.799,9.053)%
  --(3.807,9.067)--(3.814,9.081)--(3.822,9.096)--(3.829,9.111)--(3.837,9.126)--(3.844,9.140)%
  --(3.851,9.156)--(3.859,9.172)--(3.866,9.187)--(3.874,9.203)--(3.881,9.218)--(3.889,9.234)%
  --(3.896,9.250)--(3.903,9.265)--(3.911,9.281)--(3.918,9.297)--(3.926,9.313)--(3.933,9.330)%
  --(3.941,9.346)--(3.948,9.363)--(3.955,9.379)--(3.963,9.396)--(3.970,9.412)--(3.978,9.429)%
  --(3.985,9.446)--(3.993,9.463)--(4.000,9.480)--(4.007,9.497)--(4.015,9.515)--(4.022,9.532)%
  --(4.030,9.550)--(4.037,9.568)--(4.045,9.586)--(4.052,9.604)--(4.059,9.622)--(4.067,9.640)%
  --(4.082,9.678)--(4.089,9.697)--(4.097,9.717)--(4.104,9.737)--(4.111,9.757)--(4.119,9.778)%
  --(4.126,9.798)--(4.134,9.820)--(4.141,9.841)--(4.149,9.862)--(4.163,9.907)--(4.171,9.931)%
  --(4.178,9.954)--(4.186,9.979)--(4.193,10.004)--(4.201,10.030)--(4.208,10.056)--(4.215,10.084)%
  --(4.223,10.111)--(4.230,10.138)--(4.238,10.166)--(4.245,10.194)--(4.253,10.222)--(4.260,10.252)%
  --(4.267,10.281)--(4.275,10.309)--(4.282,10.337)--(4.290,10.362)--(4.297,10.389)--(4.305,10.415)%
  --(4.312,10.440)--(4.319,10.468)--(4.327,10.494)--(4.334,10.524)--(4.342,10.550)--(4.349,10.579)%
  --(4.357,10.605)--(4.364,10.634)--(4.371,10.660)--(4.379,10.690)--(4.386,10.718)--(4.394,10.745)%
  --(4.401,10.769)--(4.409,10.794)--(4.416,10.815)--(4.423,10.836)--(4.438,10.881)--(4.446,10.908)%
  --(4.453,10.933)--(4.461,10.957)--(4.468,10.981)--(4.476,11.005)--(4.483,11.030)--(4.490,11.050)%
  --(4.498,11.074)--(4.505,11.098)--(4.513,11.127)--(4.520,11.147)--(4.528,11.167)--(4.535,11.187)%
  --(4.542,11.207)--(4.550,11.227)--(4.557,11.251)--(4.565,11.275)--(4.572,11.294)--(4.580,11.314)%
  --(4.587,11.329)--(4.594,11.348)--(4.602,11.363)--(4.609,11.387)--(4.617,11.411)--(4.624,11.434)%
  --(4.632,11.453)--(4.639,11.472)--(4.646,11.486)--(4.654,11.505)--(4.661,11.519)--(4.669,11.542)%
  --(4.676,11.561)--(4.684,11.579)--(4.691,11.597)--(4.698,11.606)--(4.706,11.623)--(4.713,11.641)%
  --(4.721,11.659)--(4.728,11.673)--(4.743,11.698)--(4.750,11.715)--(4.758,11.732)--(4.765,11.739)%
  --(4.773,11.746)--(4.780,11.753)--(4.788,11.755)--(4.795,11.761)--(4.802,11.763)--(4.810,11.760)%
  --(4.825,11.753)--(4.832,11.759)--(4.840,11.756)--(4.847,11.758)--(4.854,11.750)--(4.862,11.747)%
  --(4.869,11.735)--(4.877,11.723)--(4.892,11.709)--(4.906,11.700)--(4.914,11.697)--(4.921,11.691)%
  --(4.929,11.680)--(4.936,11.675)--(4.944,11.661)--(4.951,11.651)--(4.958,11.638)--(4.973,11.594)%
  --(4.981,11.575)--(4.988,11.554)--(4.996,11.533)--(5.003,11.519)--(5.010,11.509)--(5.018,11.495)%
  --(5.025,11.486)--(5.033,11.482)--(5.040,11.470)--(5.048,11.467)--(5.055,11.455)--(5.062,11.455)%
  --(5.077,11.440)--(5.085,11.429)--(5.092,11.429)--(5.100,11.418)--(5.107,11.418)--(5.114,11.411)%
  --(5.122,11.408)--(5.129,11.408)--(5.137,11.401)--(5.144,11.401)--(5.152,11.401)--(5.159,11.396)%
  --(5.166,11.396)--(5.174,11.397)--(5.181,11.394)--(5.189,11.392)--(5.196,11.392)--(5.204,11.395)%
  --(5.211,11.397)--(5.218,11.398)--(5.226,11.398)--(5.233,11.400)--(5.241,11.403)--(5.248,11.408)%
  --(5.256,11.410)--(5.263,11.413)--(5.270,11.413)--(5.278,11.413)--(5.285,11.413)--(5.293,11.415)%
  --(5.308,11.420)--(5.315,11.424)--(5.322,11.427)--(5.330,11.429)--(5.337,11.429)--(5.345,11.428)%
  --(5.352,11.426)--(5.360,11.426)--(5.367,11.428)--(5.374,11.430)--(5.382,11.430)--(5.389,11.430)%
  --(5.397,11.428)--(5.404,11.426)--(5.412,11.426)--(5.419,11.428)--(5.426,11.428)--(5.434,11.430)%
  --(5.441,11.430)--(5.449,11.429)--(5.456,11.429)--(5.464,11.427)--(5.471,11.427)--(5.486,11.425)%
  --(5.493,11.422)--(5.501,11.420)--(5.508,11.418)--(5.516,11.416)--(5.523,11.418)--(5.530,11.418)%
  --(5.538,11.418)--(5.545,11.418)--(5.553,11.418)--(5.560,11.416)--(5.568,11.416)--(5.575,11.415)%
  --(5.582,11.415)--(5.590,11.413)--(5.597,11.411)--(5.605,11.409)--(5.620,11.405)--(5.627,11.403)%
  --(5.634,11.401)--(5.642,11.399)--(5.649,11.396)--(5.657,11.394)--(5.664,11.390)--(5.679,11.383)%
  --(5.686,11.379)--(5.694,11.376)--(5.701,11.372)--(5.709,11.367)--(5.716,11.362)--(5.724,11.355)%
  --(5.731,11.350)--(5.738,11.343)--(5.746,11.335)--(5.753,11.327)--(5.761,11.317)--(5.768,11.306)%
  --(5.776,11.299)--(5.783,11.296)--(5.790,11.294)--(5.798,11.291)--(5.805,11.288)--(5.813,11.284)%
  --(5.820,11.281)--(5.828,11.276)--(5.835,11.272)--(5.842,11.267)--(5.850,11.262)--(5.857,11.257)%
  --(5.865,11.253)--(5.872,11.247)--(5.880,11.242)--(5.887,11.236)--(5.894,11.231)--(5.902,11.225)%
  --(5.909,11.220)--(5.917,11.213)--(5.924,11.206)--(5.932,11.200)--(5.939,11.194)--(5.946,11.187)%
  --(5.954,11.180)--(5.961,11.174)--(5.969,11.167)--(5.976,11.160)--(5.984,11.155)--(5.991,11.151)%
  --(6.006,11.142)--(6.013,11.133)--(6.021,11.124)--(6.028,11.114)--(6.036,11.105)--(6.043,11.096)%
  --(6.050,11.085)--(6.058,11.075)--(6.065,11.066)--(6.073,11.055)--(6.080,11.045)--(6.088,11.034)%
  --(6.095,11.024)--(6.102,11.012)--(6.110,11.001)--(6.117,10.990)--(6.125,10.979)--(6.132,10.967)%
  --(6.140,10.955)--(6.147,10.943)--(6.154,10.931)--(6.162,10.919)--(6.169,10.907)--(6.177,10.895)%
  --(6.184,10.882)--(6.192,10.869)--(6.199,10.857)--(6.206,10.845)--(6.214,10.833)--(6.221,10.820)%
  --(6.229,10.808)--(6.236,10.795)--(6.244,10.782)--(6.251,10.769)--(6.258,10.755)--(6.266,10.740)%
  --(6.273,10.725)--(6.281,10.710)--(6.288,10.693)--(6.296,10.677)--(6.303,10.660)--(6.310,10.643)%
  --(6.318,10.624)--(6.325,10.606)--(6.333,10.586)--(6.340,10.566)--(6.348,10.545)--(6.355,10.522)%
  --(6.362,10.508)--(6.370,10.496)--(6.377,10.484)--(6.385,10.473)--(6.392,10.461)--(6.400,10.449)%
  --(6.407,10.438)--(6.414,10.426)--(6.422,10.414)--(6.429,10.402)--(6.437,10.391)--(6.444,10.380)%
  --(6.452,10.368)--(6.459,10.356)--(6.466,10.345)--(6.474,10.333)--(6.481,10.322)--(6.489,10.311)%
  --(6.496,10.299)--(6.504,10.287)--(6.511,10.276)--(6.518,10.265)--(6.526,10.254)--(6.533,10.243)%
  --(6.541,10.232)--(6.548,10.221)--(6.556,10.210)--(6.563,10.199)--(6.570,10.188)--(6.578,10.177)%
  --(6.585,10.167)--(6.593,10.156)--(6.600,10.146)--(6.608,10.136)--(6.615,10.125)--(6.622,10.115)%
  --(6.630,10.105)--(6.637,10.095)--(6.645,10.085)--(6.652,10.076)--(6.660,10.066)--(6.667,10.057)%
  --(6.674,10.047)--(6.682,10.038)--(6.689,10.029)--(6.697,10.020)--(6.704,10.011)--(6.712,10.002)%
  --(6.719,9.994)--(6.726,9.985)--(6.734,9.977)--(6.741,9.969)--(6.749,9.961)--(6.756,9.953)%
  --(6.764,9.945)--(6.771,9.938)--(6.778,9.930)--(6.786,9.923)--(6.793,9.916)--(6.801,9.909)%
  --(6.808,9.902)--(6.816,9.895)--(6.823,9.888)--(6.830,9.882)--(6.838,9.876)--(6.845,9.870)%
  --(6.853,9.864)--(6.860,9.858)--(6.868,9.852)--(6.875,9.846)--(6.882,9.841)--(6.890,9.835)%
  --(6.897,9.830)--(6.905,9.825)--(6.912,9.819)--(6.920,9.814)--(6.927,9.810)--(6.934,9.805)%
  --(6.942,9.800)--(6.949,9.795)--(6.957,9.791)--(6.964,9.786)--(6.972,9.782)--(6.979,9.778)%
  --(6.986,9.773)--(6.994,9.769)--(7.001,9.765)--(7.009,9.761)--(7.016,9.758)--(7.024,9.754)%
  --(7.031,9.750)--(7.038,9.747)--(7.046,9.743)--(7.053,9.740)--(7.061,9.736)--(7.068,9.733)%
  --(7.076,9.729)--(7.083,9.726)--(7.090,9.723)--(7.098,9.720)--(7.105,9.717)--(7.113,9.714)%
  --(7.120,9.711)--(7.128,9.709)--(7.135,9.706)--(7.142,9.703)--(7.150,9.701)--(7.157,9.698)%
  --(7.165,9.696)--(7.172,9.693)--(7.180,9.691)--(7.187,9.688)--(7.194,9.686)--(7.202,9.684)%
  --(7.209,9.682)--(7.217,9.680)--(7.224,9.677)--(7.232,9.675)--(7.239,9.673)--(7.246,9.671)%
  --(7.254,9.669)--(7.261,9.667)--(7.269,9.665)--(7.276,9.664)--(7.284,9.662)--(7.291,9.660)%
  --(7.298,9.658)--(7.306,9.657)--(7.313,9.655)--(7.321,9.653)--(7.328,9.652)--(7.336,9.650)%
  --(7.343,9.649)--(7.350,9.647)--(7.358,9.646)--(7.365,9.645)--(7.373,9.643)--(7.380,9.642)%
  --(7.388,9.640)--(7.395,9.639)--(7.402,9.638)--(7.410,9.636)--(7.417,9.635)--(7.425,9.634)%
  --(7.432,9.632)--(7.440,9.631)--(7.447,9.630);
\gpsetdashtype{gp dt 4}
\draw[gp path] (1.504,9.499)--(1.511,9.499)--(1.519,9.499)--(1.526,9.499)--(1.534,9.499)%
  --(1.541,9.499)--(1.549,9.499)--(1.556,9.499)--(1.563,9.499)--(1.571,9.499)--(1.578,9.499)%
  --(1.586,9.499)--(1.593,9.499)--(1.601,9.499)--(1.608,9.499)--(1.615,9.499)--(1.623,9.499)%
  --(1.630,9.499)--(1.638,9.499)--(1.645,9.499)--(1.653,9.499)--(1.660,9.499)--(1.667,9.499)%
  --(1.675,9.499)--(1.682,9.499)--(1.690,9.499)--(1.697,9.499)--(1.705,9.499)--(1.712,9.499)%
  --(1.719,9.499)--(1.727,9.499)--(1.734,9.499)--(1.742,9.498)--(1.749,9.498)--(1.757,9.498)%
  --(1.764,9.498)--(1.771,9.498)--(1.779,9.498)--(1.786,9.498)--(1.794,9.498)--(1.801,9.498)%
  --(1.809,9.498)--(1.816,9.498)--(1.823,9.498)--(1.831,9.498)--(1.838,9.498)--(1.846,9.498)%
  --(1.853,9.498)--(1.861,9.497)--(1.868,9.497)--(1.875,9.497)--(1.883,9.497)--(1.890,9.497)%
  --(1.898,9.497)--(1.905,9.497)--(1.913,9.497)--(1.920,9.497)--(1.927,9.497)--(1.935,9.497)%
  --(1.942,9.496)--(1.950,9.496)--(1.957,9.496)--(1.965,9.496)--(1.972,9.496)--(1.979,9.496)%
  --(1.987,9.496)--(1.994,9.496)--(2.002,9.496)--(2.009,9.496)--(2.017,9.496)--(2.024,9.496)%
  --(2.031,9.496)--(2.039,9.496)--(2.046,9.496)--(2.054,9.496)--(2.061,9.496)--(2.069,9.496)%
  --(2.076,9.496)--(2.083,9.496)--(2.091,9.496)--(2.098,9.496)--(2.106,9.496)--(2.113,9.496)%
  --(2.121,9.495)--(2.128,9.495)--(2.135,9.495)--(2.143,9.495)--(2.150,9.494)--(2.158,9.494)%
  --(2.165,9.494)--(2.173,9.494)--(2.180,9.493)--(2.187,9.493)--(2.195,9.493)--(2.202,9.493)%
  --(2.210,9.493)--(2.217,9.493)--(2.225,9.492)--(2.232,9.492)--(2.239,9.492)--(2.247,9.492)%
  --(2.254,9.492)--(2.262,9.492)--(2.269,9.492)--(2.277,9.491)--(2.284,9.491)--(2.291,9.491)%
  --(2.299,9.491)--(2.306,9.490)--(2.314,9.490)--(2.321,9.490)--(2.329,9.490)--(2.336,9.489)%
  --(2.343,9.489)--(2.351,9.488)--(2.358,9.488)--(2.366,9.487)--(2.373,9.487)--(2.381,9.487)%
  --(2.388,9.486)--(2.395,9.486)--(2.403,9.486)--(2.410,9.485)--(2.418,9.485)--(2.425,9.485)%
  --(2.433,9.484)--(2.440,9.484)--(2.447,9.484)--(2.455,9.483)--(2.462,9.483)--(2.470,9.477)%
  --(2.477,9.475)--(2.485,9.475)--(2.492,9.474)--(2.499,9.473)--(2.507,9.472)--(2.514,9.472)%
  --(2.522,9.471)--(2.529,9.470)--(2.537,9.469)--(2.544,9.468)--(2.551,9.468)--(2.559,9.467)%
  --(2.566,9.466)--(2.574,9.465)--(2.581,9.464)--(2.589,9.462)--(2.596,9.461)--(2.603,9.460)%
  --(2.611,9.458)--(2.618,9.455)--(2.626,9.452)--(2.633,9.447)--(2.641,9.442)--(2.648,9.434)%
  --(2.655,9.423)--(2.663,9.411)--(2.670,9.403)--(2.678,9.394)--(2.685,9.378)--(2.693,9.367)%
  --(2.700,9.359)--(2.707,9.350)--(2.715,9.340)--(2.722,9.330)--(2.745,9.308)--(2.752,9.297)%
  --(2.759,9.284)--(2.767,9.267)--(2.774,9.247)--(2.782,9.220)--(2.789,9.179)--(2.797,9.121)%
  --(2.804,9.101)--(2.811,9.090)--(2.819,9.079)--(2.826,9.060)--(2.834,9.036)--(2.841,9.011)%
  --(2.849,8.980)--(2.856,8.944)--(2.863,8.916)--(2.871,8.891)--(2.878,8.868)--(2.886,8.845)%
  --(2.893,8.822)--(2.901,8.801)--(2.908,8.781)--(2.915,8.762)--(2.923,8.744)--(2.930,8.727)%
  --(2.938,8.710)--(2.945,8.694)--(2.953,8.679)--(2.960,8.665)--(2.967,8.652)--(2.975,8.640)%
  --(2.982,8.628)--(2.990,8.616)--(2.997,8.606)--(3.005,8.596)--(3.012,8.587)--(3.019,8.579)%
  --(3.027,8.571)--(3.034,8.565)--(3.049,8.556)--(3.057,8.551)--(3.064,8.549)--(3.071,8.546)%
  --(3.079,8.544)--(3.086,8.542)--(3.094,8.512)--(3.101,8.496)--(3.109,8.485)--(3.123,8.474)%
  --(3.131,8.471)--(3.138,8.469)--(3.146,8.468)--(3.153,8.468)--(3.161,8.469)--(3.168,8.470)%
  --(3.175,8.471)--(3.183,8.473)--(3.190,8.475)--(3.198,8.477)--(3.205,8.478)--(3.213,8.481)%
  --(3.220,8.483)--(3.227,8.485)--(3.235,8.487)--(3.242,8.489)--(3.250,8.491)--(3.257,8.495)%
  --(3.265,8.498)--(3.272,8.500)--(3.279,8.504)--(3.287,8.508)--(3.294,8.511)--(3.302,8.515)%
  --(3.309,8.519)--(3.317,8.523)--(3.324,8.528)--(3.331,8.533)--(3.339,8.538)--(3.346,8.544)%
  --(3.354,8.549)--(3.361,8.556)--(3.369,8.561)--(3.376,8.567)--(3.383,8.572)--(3.391,8.562)%
  --(3.398,8.559)--(3.406,8.559)--(3.413,8.562)--(3.421,8.567)--(3.428,8.573)--(3.435,8.580)%
  --(3.443,8.589)--(3.450,8.598)--(3.458,8.606)--(3.465,8.615)--(3.473,8.623)--(3.480,8.632)%
  --(3.487,8.640)--(3.495,8.648)--(3.502,8.658)--(3.510,8.666)--(3.517,8.675)--(3.525,8.684)%
  --(3.532,8.693)--(3.539,8.703)--(3.547,8.712)--(3.554,8.720)--(3.562,8.730)--(3.569,8.740)%
  --(3.577,8.750)--(3.584,8.760)--(3.591,8.771)--(3.599,8.781)--(3.606,8.792)--(3.614,8.802)%
  --(3.621,8.813)--(3.629,8.825)--(3.636,8.836)--(3.643,8.848)--(3.651,8.859)--(3.658,8.865)%
  --(3.666,8.871)--(3.673,8.879)--(3.681,8.890)--(3.688,8.901)--(3.695,8.913)--(3.703,8.924)%
  --(3.710,8.936)--(3.718,8.949)--(3.725,8.962)--(3.733,8.975)--(3.747,9.001)--(3.755,9.014)%
  --(3.762,9.028)--(3.770,9.041)--(3.777,9.055)--(3.785,9.068)--(3.792,9.082)--(3.799,9.096)%
  --(3.807,9.111)--(3.814,9.125)--(3.822,9.139)--(3.829,9.154)--(3.837,9.169)--(3.844,9.183)%
  --(3.859,9.213)--(3.866,9.229)--(3.874,9.244)--(3.881,9.259)--(3.889,9.273)--(3.896,9.287)%
  --(3.903,9.303)--(3.911,9.318)--(3.918,9.334)--(3.926,9.350)--(3.933,9.366)--(3.941,9.382)%
  --(3.948,9.398)--(3.955,9.415)--(3.970,9.447)--(3.978,9.464)--(3.985,9.481)--(3.993,9.498)%
  --(4.000,9.515)--(4.007,9.532)--(4.015,9.549)--(4.022,9.566)--(4.030,9.584)--(4.037,9.601)%
  --(4.045,9.619)--(4.052,9.636)--(4.059,9.654)--(4.067,9.671)--(4.074,9.689)--(4.089,9.725)%
  --(4.097,9.743)--(4.104,9.761)--(4.111,9.780)--(4.119,9.798)--(4.126,9.817)--(4.134,9.836)%
  --(4.141,9.854)--(4.149,9.873)--(4.156,9.892)--(4.163,9.911)--(4.171,9.930)--(4.178,9.950)%
  --(4.186,9.969)--(4.193,9.991)--(4.208,10.035)--(4.215,10.057)--(4.223,10.081)--(4.230,10.106)%
  --(4.238,10.132)--(4.245,10.159)--(4.253,10.186)--(4.260,10.213)--(4.267,10.242)--(4.275,10.273)%
  --(4.282,10.307)--(4.290,10.342)--(4.297,10.380)--(4.305,10.416)--(4.312,10.452)--(4.319,10.494)%
  --(4.327,10.537)--(4.342,10.629)--(4.349,10.674)--(4.357,10.713)--(4.364,10.754)--(4.371,10.789)%
  --(4.379,10.825)--(4.386,10.858)--(4.394,10.891)--(4.401,10.918)--(4.409,10.944)--(4.416,10.974)%
  --(4.423,11.004)--(4.431,11.031)--(4.438,11.061)--(4.446,11.092)--(4.453,11.127)--(4.461,11.158)%
  --(4.468,11.181)--(4.476,11.208)--(4.483,11.235)--(4.490,11.266)--(4.498,11.288)--(4.505,11.310)%
  --(4.513,11.336)--(4.520,11.363)--(4.528,11.395)--(4.535,11.422)--(4.542,11.444)--(4.550,11.471)%
  --(4.557,11.498)--(4.572,11.546)--(4.580,11.573)--(4.587,11.594)--(4.594,11.615)--(4.602,11.635)%
  --(4.609,11.656)--(4.617,11.683)--(4.624,11.698)--(4.632,11.706)--(4.639,11.720)--(4.646,11.740)%
  --(4.661,11.767)--(4.669,11.781)--(4.676,11.800)--(4.684,11.814)--(4.691,11.821)--(4.698,11.833)%
  --(4.706,11.847)--(4.713,11.859)--(4.721,11.860)--(4.728,11.878)--(4.736,11.885)--(4.743,11.885)%
  --(4.750,11.891)--(4.758,11.879)--(4.765,11.874)--(4.780,11.829)--(4.788,11.797)--(4.795,11.751)%
  --(4.802,11.708)--(4.810,11.657)--(4.817,11.610)--(4.825,11.561)--(4.832,11.519)--(4.840,11.479)%
  --(4.847,11.441)--(4.854,11.416)--(4.862,11.398)--(4.869,11.381)--(4.877,11.365)--(4.892,11.342)%
  --(4.899,11.327)--(4.906,11.315)--(4.914,11.304)--(4.921,11.290)--(4.929,11.276)--(4.936,11.263)%
  --(4.944,11.247)--(4.951,11.235)--(4.958,11.225)--(4.966,11.221)--(4.973,11.214)--(4.981,11.210)%
  --(4.988,11.204)--(5.003,11.196)--(5.010,11.195)--(5.018,11.198)--(5.025,11.204)--(5.033,11.208)%
  --(5.040,11.211)--(5.048,11.214)--(5.055,11.220)--(5.062,11.225)--(5.070,11.231)--(5.077,11.236)%
  --(5.085,11.242)--(5.092,11.245)--(5.100,11.248)--(5.107,11.251)--(5.114,11.254)--(5.122,11.259)%
  --(5.129,11.262)--(5.137,11.269)--(5.144,11.275)--(5.152,11.280)--(5.159,11.285)--(5.166,11.288)%
  --(5.174,11.290)--(5.181,11.293)--(5.189,11.293)--(5.196,11.296)--(5.211,11.301)--(5.218,11.306)%
  --(5.226,11.310)--(5.233,11.313)--(5.241,11.315)--(5.248,11.318)--(5.256,11.320)--(5.263,11.320)%
  --(5.270,11.322)--(5.278,11.322)--(5.285,11.322)--(5.293,11.323)--(5.308,11.325)--(5.315,11.329)%
  --(5.322,11.331)--(5.330,11.334)--(5.337,11.336)--(5.345,11.338)--(5.352,11.340)--(5.360,11.342)%
  --(5.367,11.342)--(5.374,11.344)--(5.382,11.344)--(5.397,11.346)--(5.404,11.345)--(5.412,11.347)%
  --(5.419,11.347)--(5.426,11.347)--(5.434,11.345)--(5.441,11.344)--(5.449,11.344)--(5.456,11.342)%
  --(5.464,11.340)--(5.478,11.334)--(5.486,11.331)--(5.493,11.327)--(5.501,11.321)--(5.508,11.315)%
  --(5.516,11.309)--(5.523,11.300)--(5.530,11.293)--(5.538,11.291)--(5.545,11.289)--(5.553,11.287)%
  --(5.560,11.287)--(5.568,11.285)--(5.575,11.283)--(5.582,11.279)--(5.590,11.278)--(5.597,11.276)%
  --(5.605,11.272)--(5.620,11.267)--(5.627,11.263)--(5.634,11.260)--(5.642,11.256)--(5.649,11.253)%
  --(5.657,11.251)--(5.664,11.247)--(5.672,11.244)--(5.679,11.241)--(5.686,11.237)--(5.694,11.235)%
  --(5.701,11.230)--(5.709,11.227)--(5.716,11.222)--(5.724,11.217)--(5.731,11.212)--(5.738,11.206)%
  --(5.746,11.201)--(5.753,11.196)--(5.761,11.191)--(5.768,11.185)--(5.776,11.179)--(5.783,11.173)%
  --(5.790,11.166)--(5.798,11.161)--(5.805,11.154)--(5.813,11.147)--(5.820,11.140)--(5.828,11.132)%
  --(5.835,11.126)--(5.842,11.118)--(5.850,11.111)--(5.857,11.103)--(5.865,11.096)--(5.872,11.089)%
  --(5.880,11.080)--(5.887,11.072)--(5.894,11.064)--(5.902,11.056)--(5.909,11.047)--(5.917,11.038)%
  --(5.924,11.030)--(5.932,11.020)--(5.939,11.010)--(5.946,11.001)--(5.954,10.990)--(5.961,10.980)%
  --(5.969,10.969)--(5.976,10.958)--(5.984,10.946)--(5.991,10.935)--(5.998,10.923)--(6.006,10.912)%
  --(6.013,10.901)--(6.043,10.835)--(6.050,10.816)--(6.058,10.794)--(6.065,10.770)--(6.073,10.743)%
  --(6.080,10.723)--(6.088,10.713)--(6.095,10.704)--(6.102,10.695)--(6.110,10.688)--(6.117,10.679)%
  --(6.125,10.672)--(6.132,10.664)--(6.140,10.656)--(6.147,10.647)--(6.154,10.637)--(6.162,10.626)%
  --(6.169,10.616)--(6.177,10.605)--(6.184,10.595)--(6.192,10.584)--(6.199,10.573)--(6.206,10.562)%
  --(6.214,10.551)--(6.221,10.540)--(6.229,10.529)--(6.236,10.517)--(6.244,10.506)--(6.251,10.495)%
  --(6.258,10.483)--(6.266,10.471)--(6.273,10.460)--(6.281,10.448)--(6.288,10.436)--(6.296,10.425)%
  --(6.303,10.412)--(6.310,10.400)--(6.318,10.389)--(6.325,10.376)--(6.333,10.364)--(6.340,10.352)%
  --(6.348,10.340)--(6.355,10.328)--(6.362,10.316)--(6.370,10.304)--(6.377,10.291)--(6.385,10.279)%
  --(6.392,10.267)--(6.400,10.255)--(6.407,10.243)--(6.414,10.231)--(6.422,10.219)--(6.429,10.207)%
  --(6.437,10.196)--(6.444,10.184)--(6.452,10.173)--(6.459,10.161)--(6.466,10.150)--(6.474,10.139)%
  --(6.481,10.128)--(6.489,10.117)--(6.496,10.106)--(6.504,10.095)--(6.511,10.085)--(6.518,10.074)%
  --(6.526,10.064)--(6.533,10.054)--(6.541,10.044)--(6.548,10.035)--(6.556,10.025)--(6.563,10.016)%
  --(6.570,10.007)--(6.578,9.998)--(6.585,9.989)--(6.593,9.981)--(6.600,9.972)--(6.608,9.964)%
  --(6.615,9.956)--(6.622,9.949)--(6.630,9.941)--(6.637,9.934)--(6.645,9.926)--(6.652,9.919)%
  --(6.660,9.912)--(6.667,9.905)--(6.674,9.899)--(6.682,9.892)--(6.689,9.885)--(6.697,9.879)%
  --(6.704,9.873)--(6.712,9.866)--(6.719,9.860)--(6.726,9.854)--(6.734,9.848)--(6.741,9.843)%
  --(6.749,9.837)--(6.756,9.831)--(6.764,9.826)--(6.771,9.820)--(6.778,9.815)--(6.786,9.810)%
  --(6.793,9.805)--(6.801,9.800)--(6.808,9.795)--(6.816,9.790)--(6.823,9.785)--(6.830,9.780)%
  --(6.838,9.776)--(6.845,9.771)--(6.853,9.767)--(6.860,9.762)--(6.868,9.758)--(6.875,9.754)%
  --(6.882,9.750)--(6.890,9.746)--(6.897,9.742)--(6.905,9.738)--(6.912,9.734)--(6.920,9.731)%
  --(6.927,9.727)--(6.934,9.723)--(6.942,9.720)--(6.949,9.716)--(6.957,9.713)--(6.964,9.709)%
  --(6.972,9.706)--(6.979,9.703)--(6.986,9.699)--(6.994,9.696)--(7.001,9.692)--(7.009,9.688)%
  --(7.016,9.684)--(7.024,9.680)--(7.031,9.676)--(7.038,9.671)--(7.046,9.667)--(7.053,9.662)%
  --(7.061,9.656)--(7.076,9.636)--(7.083,9.634)--(7.090,9.632)--(7.098,9.631)--(7.105,9.631)%
  --(7.113,9.630)--(7.120,9.630)--(7.128,9.629)--(7.135,9.629)--(7.142,9.628)--(7.150,9.628)%
  --(7.157,9.628)--(7.165,9.627)--(7.172,9.627)--(7.180,9.626)--(7.187,9.622)--(7.194,9.621)%
  --(7.202,9.621)--(7.209,9.621)--(7.217,9.620)--(7.224,9.620)--(7.232,9.619)--(7.239,9.619)%
  --(7.246,9.618)--(7.254,9.618)--(7.261,9.618)--(7.269,9.617)--(7.276,9.617)--(7.284,9.617)%
  --(7.291,9.617)--(7.298,9.617)--(7.306,9.616)--(7.313,9.616)--(7.321,9.616)--(7.328,9.616)%
  --(7.336,9.616)--(7.343,9.616)--(7.350,9.616)--(7.358,9.615)--(7.365,9.615)--(7.373,9.615)%
  --(7.380,9.615)--(7.388,9.615)--(7.395,9.615)--(7.402,9.615)--(7.410,9.615)--(7.417,9.614)%
  --(7.425,9.614)--(7.432,9.614)--(7.440,9.614)--(7.447,9.614);
\gpsetdashtype{gp dt solid}
\draw[gp path] (1.504,12.870)--(1.504,7.882)--(7.447,7.882)--(7.447,12.870)--cycle;
%% coordinates of the plot area
\gpdefrectangularnode{gp plot 2}{\pgfpoint{1.504cm}{7.882cm}}{\pgfpoint{7.447cm}{12.870cm}}
\gpcolor{color=gp lt color axes}
\gpsetlinetype{gp lt axes}
\gpsetdashtype{gp dt axes}
\gpsetlinewidth{0.50}
\draw[gp path] (1.688,0.985)--(7.447,0.985);
\gpcolor{color=gp lt color border}
\gpsetlinetype{gp lt border}
\gpsetdashtype{gp dt solid}
\gpsetlinewidth{1.00}
\draw[gp path] (1.688,0.985)--(1.868,0.985);
\draw[gp path] (7.447,0.985)--(7.267,0.985);
\node[gp node right] at (1.504,0.985) {$-0.16$};
\gpcolor{color=gp lt color axes}
\gpsetlinetype{gp lt axes}
\gpsetdashtype{gp dt axes}
\gpsetlinewidth{0.50}
\draw[gp path] (1.688,1.484)--(7.447,1.484);
\gpcolor{color=gp lt color border}
\gpsetlinetype{gp lt border}
\gpsetdashtype{gp dt solid}
\gpsetlinewidth{1.00}
\draw[gp path] (1.688,1.484)--(1.868,1.484);
\draw[gp path] (7.447,1.484)--(7.267,1.484);
\node[gp node right] at (1.504,1.484) {$-0.14$};
\gpcolor{color=gp lt color axes}
\gpsetlinetype{gp lt axes}
\gpsetdashtype{gp dt axes}
\gpsetlinewidth{0.50}
\draw[gp path] (1.688,1.983)--(7.447,1.983);
\gpcolor{color=gp lt color border}
\gpsetlinetype{gp lt border}
\gpsetdashtype{gp dt solid}
\gpsetlinewidth{1.00}
\draw[gp path] (1.688,1.983)--(1.868,1.983);
\draw[gp path] (7.447,1.983)--(7.267,1.983);
\node[gp node right] at (1.504,1.983) {$-0.12$};
\gpcolor{color=gp lt color axes}
\gpsetlinetype{gp lt axes}
\gpsetdashtype{gp dt axes}
\gpsetlinewidth{0.50}
\draw[gp path] (1.688,2.481)--(7.447,2.481);
\gpcolor{color=gp lt color border}
\gpsetlinetype{gp lt border}
\gpsetdashtype{gp dt solid}
\gpsetlinewidth{1.00}
\draw[gp path] (1.688,2.481)--(1.868,2.481);
\draw[gp path] (7.447,2.481)--(7.267,2.481);
\node[gp node right] at (1.504,2.481) {$-0.1$};
\gpcolor{color=gp lt color axes}
\gpsetlinetype{gp lt axes}
\gpsetdashtype{gp dt axes}
\gpsetlinewidth{0.50}
\draw[gp path] (1.688,2.980)--(7.447,2.980);
\gpcolor{color=gp lt color border}
\gpsetlinetype{gp lt border}
\gpsetdashtype{gp dt solid}
\gpsetlinewidth{1.00}
\draw[gp path] (1.688,2.980)--(1.868,2.980);
\draw[gp path] (7.447,2.980)--(7.267,2.980);
\node[gp node right] at (1.504,2.980) {$-0.08$};
\gpcolor{color=gp lt color axes}
\gpsetlinetype{gp lt axes}
\gpsetdashtype{gp dt axes}
\gpsetlinewidth{0.50}
\draw[gp path] (1.688,3.479)--(7.447,3.479);
\gpcolor{color=gp lt color border}
\gpsetlinetype{gp lt border}
\gpsetdashtype{gp dt solid}
\gpsetlinewidth{1.00}
\draw[gp path] (1.688,3.479)--(1.868,3.479);
\draw[gp path] (7.447,3.479)--(7.267,3.479);
\node[gp node right] at (1.504,3.479) {$-0.06$};
\gpcolor{color=gp lt color axes}
\gpsetlinetype{gp lt axes}
\gpsetdashtype{gp dt axes}
\gpsetlinewidth{0.50}
\draw[gp path] (1.688,3.978)--(7.447,3.978);
\gpcolor{color=gp lt color border}
\gpsetlinetype{gp lt border}
\gpsetdashtype{gp dt solid}
\gpsetlinewidth{1.00}
\draw[gp path] (1.688,3.978)--(1.868,3.978);
\draw[gp path] (7.447,3.978)--(7.267,3.978);
\node[gp node right] at (1.504,3.978) {$-0.04$};
\gpcolor{color=gp lt color axes}
\gpsetlinetype{gp lt axes}
\gpsetdashtype{gp dt axes}
\gpsetlinewidth{0.50}
\draw[gp path] (1.688,4.477)--(7.447,4.477);
\gpcolor{color=gp lt color border}
\gpsetlinetype{gp lt border}
\gpsetdashtype{gp dt solid}
\gpsetlinewidth{1.00}
\draw[gp path] (1.688,4.477)--(1.868,4.477);
\draw[gp path] (7.447,4.477)--(7.267,4.477);
\node[gp node right] at (1.504,4.477) {$-0.02$};
\gpcolor{color=gp lt color axes}
\gpsetlinetype{gp lt axes}
\gpsetdashtype{gp dt axes}
\gpsetlinewidth{0.50}
\draw[gp path] (1.688,4.975)--(7.447,4.975);
\gpcolor{color=gp lt color border}
\gpsetlinetype{gp lt border}
\gpsetdashtype{gp dt solid}
\gpsetlinewidth{1.00}
\draw[gp path] (1.688,4.975)--(1.868,4.975);
\draw[gp path] (7.447,4.975)--(7.267,4.975);
\node[gp node right] at (1.504,4.975) {$0$};
\gpcolor{color=gp lt color axes}
\gpsetlinetype{gp lt axes}
\gpsetdashtype{gp dt axes}
\gpsetlinewidth{0.50}
\draw[gp path] (1.688,5.474)--(7.447,5.474);
\gpcolor{color=gp lt color border}
\gpsetlinetype{gp lt border}
\gpsetdashtype{gp dt solid}
\gpsetlinewidth{1.00}
\draw[gp path] (1.688,5.474)--(1.868,5.474);
\draw[gp path] (7.447,5.474)--(7.267,5.474);
\node[gp node right] at (1.504,5.474) {$0.02$};
\gpcolor{color=gp lt color axes}
\gpsetlinetype{gp lt axes}
\gpsetdashtype{gp dt axes}
\gpsetlinewidth{0.50}
\draw[gp path] (1.688,5.973)--(7.447,5.973);
\gpcolor{color=gp lt color border}
\gpsetlinetype{gp lt border}
\gpsetdashtype{gp dt solid}
\gpsetlinewidth{1.00}
\draw[gp path] (1.688,5.973)--(1.868,5.973);
\draw[gp path] (7.447,5.973)--(7.267,5.973);
\node[gp node right] at (1.504,5.973) {$0.04$};
\gpcolor{color=gp lt color axes}
\gpsetlinetype{gp lt axes}
\gpsetdashtype{gp dt axes}
\gpsetlinewidth{0.50}
\draw[gp path] (1.688,0.985)--(1.688,5.973);
\gpcolor{color=gp lt color border}
\gpsetlinetype{gp lt border}
\gpsetdashtype{gp dt solid}
\gpsetlinewidth{1.00}
\draw[gp path] (1.688,0.985)--(1.688,1.165);
\draw[gp path] (1.688,5.973)--(1.688,5.793);
\node[gp node center] at (1.688,0.677) {$-20$};
\gpcolor{color=gp lt color axes}
\gpsetlinetype{gp lt axes}
\gpsetdashtype{gp dt axes}
\gpsetlinewidth{0.50}
\draw[gp path] (2.408,0.985)--(2.408,5.973);
\gpcolor{color=gp lt color border}
\gpsetlinetype{gp lt border}
\gpsetdashtype{gp dt solid}
\gpsetlinewidth{1.00}
\draw[gp path] (2.408,0.985)--(2.408,1.165);
\draw[gp path] (2.408,5.973)--(2.408,5.793);
\node[gp node center] at (2.408,0.677) {$-15$};
\gpcolor{color=gp lt color axes}
\gpsetlinetype{gp lt axes}
\gpsetdashtype{gp dt axes}
\gpsetlinewidth{0.50}
\draw[gp path] (3.128,0.985)--(3.128,5.973);
\gpcolor{color=gp lt color border}
\gpsetlinetype{gp lt border}
\gpsetdashtype{gp dt solid}
\gpsetlinewidth{1.00}
\draw[gp path] (3.128,0.985)--(3.128,1.165);
\draw[gp path] (3.128,5.973)--(3.128,5.793);
\node[gp node center] at (3.128,0.677) {$-10$};
\gpcolor{color=gp lt color axes}
\gpsetlinetype{gp lt axes}
\gpsetdashtype{gp dt axes}
\gpsetlinewidth{0.50}
\draw[gp path] (3.848,0.985)--(3.848,5.973);
\gpcolor{color=gp lt color border}
\gpsetlinetype{gp lt border}
\gpsetdashtype{gp dt solid}
\gpsetlinewidth{1.00}
\draw[gp path] (3.848,0.985)--(3.848,1.165);
\draw[gp path] (3.848,5.973)--(3.848,5.793);
\node[gp node center] at (3.848,0.677) {$-5$};
\gpcolor{color=gp lt color axes}
\gpsetlinetype{gp lt axes}
\gpsetdashtype{gp dt axes}
\gpsetlinewidth{0.50}
\draw[gp path] (4.568,0.985)--(4.568,5.973);
\gpcolor{color=gp lt color border}
\gpsetlinetype{gp lt border}
\gpsetdashtype{gp dt solid}
\gpsetlinewidth{1.00}
\draw[gp path] (4.568,0.985)--(4.568,1.165);
\draw[gp path] (4.568,5.973)--(4.568,5.793);
\node[gp node center] at (4.568,0.677) {$0$};
\gpcolor{color=gp lt color axes}
\gpsetlinetype{gp lt axes}
\gpsetdashtype{gp dt axes}
\gpsetlinewidth{0.50}
\draw[gp path] (5.287,0.985)--(5.287,5.973);
\gpcolor{color=gp lt color border}
\gpsetlinetype{gp lt border}
\gpsetdashtype{gp dt solid}
\gpsetlinewidth{1.00}
\draw[gp path] (5.287,0.985)--(5.287,1.165);
\draw[gp path] (5.287,5.973)--(5.287,5.793);
\node[gp node center] at (5.287,0.677) {$5$};
\gpcolor{color=gp lt color axes}
\gpsetlinetype{gp lt axes}
\gpsetdashtype{gp dt axes}
\gpsetlinewidth{0.50}
\draw[gp path] (6.007,0.985)--(6.007,5.973);
\gpcolor{color=gp lt color border}
\gpsetlinetype{gp lt border}
\gpsetdashtype{gp dt solid}
\gpsetlinewidth{1.00}
\draw[gp path] (6.007,0.985)--(6.007,1.165);
\draw[gp path] (6.007,5.973)--(6.007,5.793);
\node[gp node center] at (6.007,0.677) {$10$};
\gpcolor{color=gp lt color axes}
\gpsetlinetype{gp lt axes}
\gpsetdashtype{gp dt axes}
\gpsetlinewidth{0.50}
\draw[gp path] (6.727,0.985)--(6.727,5.973);
\gpcolor{color=gp lt color border}
\gpsetlinetype{gp lt border}
\gpsetdashtype{gp dt solid}
\gpsetlinewidth{1.00}
\draw[gp path] (6.727,0.985)--(6.727,1.165);
\draw[gp path] (6.727,5.973)--(6.727,5.793);
\node[gp node center] at (6.727,0.677) {$15$};
\gpcolor{color=gp lt color axes}
\gpsetlinetype{gp lt axes}
\gpsetdashtype{gp dt axes}
\gpsetlinewidth{0.50}
\draw[gp path] (7.447,0.985)--(7.447,5.973);
\gpcolor{color=gp lt color border}
\gpsetlinetype{gp lt border}
\gpsetdashtype{gp dt solid}
\gpsetlinewidth{1.00}
\draw[gp path] (7.447,0.985)--(7.447,1.165);
\draw[gp path] (7.447,5.973)--(7.447,5.793);
\node[gp node center] at (7.447,0.677) {$20$};
\draw[gp path] (1.688,5.973)--(1.688,0.985)--(7.447,0.985)--(7.447,5.973)--cycle;
\node[gp node center,rotate=-270] at (0.246,3.479) {$C_m$};
\node[gp node center] at (4.567,0.215) {$\alpha [\si{deg}]$};
\node[gp node center] at (4.567,6.435) {$C_m \times \alpha$};
\draw[gp path] (1.688,2.062)--(1.695,2.027)--(1.702,1.995)--(1.710,1.963)--(1.731,1.870)%
  --(1.738,1.840)--(1.746,1.813)--(1.753,1.783)--(1.760,1.758)--(1.767,1.728)--(1.774,1.703)%
  --(1.782,1.678)--(1.789,1.651)--(1.796,1.626)--(1.803,1.601)--(1.810,1.579)--(1.818,1.554)%
  --(1.825,1.531)--(1.832,1.506)--(1.839,1.484)--(1.854,1.441)--(1.861,1.419)--(1.868,1.399)%
  --(1.875,1.377)--(1.882,1.359)--(1.890,1.339)--(1.897,1.319)--(1.904,1.302)--(1.911,1.284)%
  --(1.918,1.267)--(1.926,1.249)--(1.933,1.234)--(1.940,1.219)--(1.947,1.204)--(1.954,1.192)%
  --(1.969,1.167)--(1.976,1.155)--(1.983,1.145)--(1.990,1.132)--(1.998,1.125)--(2.005,1.115)%
  --(2.012,1.105)--(2.019,1.097)--(2.026,1.090)--(2.034,1.082)--(2.041,1.077)--(2.048,1.072)%
  --(2.055,1.067)--(2.062,1.062)--(2.070,1.057)--(2.077,1.055)--(2.091,1.050)--(2.098,1.047)%
  --(2.106,1.045)--(2.113,1.045)--(2.120,1.045)--(2.127,1.045)--(2.134,1.042)--(2.142,1.045)%
  --(2.149,1.045)--(2.156,1.047)--(2.163,1.047)--(2.170,1.050)--(2.178,1.052)--(2.185,1.055)%
  --(2.192,1.060)--(2.199,1.062)--(2.206,1.067)--(2.214,1.070)--(2.228,1.080)--(2.235,1.082)%
  --(2.242,1.087)--(2.250,1.092)--(2.257,1.097)--(2.264,1.102)--(2.271,1.107)--(2.278,1.112)%
  --(2.285,1.117)--(2.293,1.122)--(2.300,1.130)--(2.307,1.135)--(2.314,1.142)--(2.321,1.147)%
  --(2.329,1.155)--(2.336,1.162)--(2.343,1.167)--(2.350,1.175)--(2.357,1.182)--(2.365,1.190)%
  --(2.379,1.207)--(2.386,1.214)--(2.393,1.219)--(2.401,1.227)--(2.408,1.234)--(2.415,1.244)%
  --(2.422,1.252)--(2.429,1.259)--(2.437,1.267)--(2.444,1.277)--(2.451,1.284)--(2.458,1.292)%
  --(2.465,1.299)--(2.473,1.309)--(2.480,1.317)--(2.487,1.324)--(2.494,1.334)--(2.501,1.344)%
  --(2.509,1.352)--(2.516,1.362)--(2.523,1.369)--(2.530,1.379)--(2.545,1.394)--(2.552,1.404)%
  --(2.559,1.411)--(2.566,1.419)--(2.573,1.429)--(2.581,1.436)--(2.588,1.444)--(2.595,1.454)%
  --(2.602,1.461)--(2.609,1.469)--(2.617,1.476)--(2.624,1.486)--(2.631,1.494)--(2.638,1.501)%
  --(2.645,1.509)--(2.653,1.519)--(2.660,1.526)--(2.667,1.536)--(2.674,1.544)--(2.681,1.554)%
  --(2.689,1.561)--(2.696,1.569)--(2.703,1.576)--(2.710,1.586)--(2.717,1.594)--(2.725,1.601)%
  --(2.732,1.609)--(2.746,1.623)--(2.753,1.631)--(2.761,1.638)--(2.768,1.648)--(2.775,1.656)%
  --(2.782,1.663)--(2.789,1.671)--(2.797,1.678)--(2.804,1.688)--(2.811,1.701)--(2.818,1.713)%
  --(2.825,1.723)--(2.833,1.733)--(2.840,1.743)--(2.847,1.753)--(2.854,1.763)--(2.861,1.773)%
  --(2.869,1.783)--(2.876,1.793)--(2.883,1.801)--(2.890,1.808)--(2.897,1.818)--(2.905,1.825)%
  --(2.912,1.833)--(2.919,1.840)--(2.926,1.848)--(2.933,1.855)--(2.941,1.865)--(2.948,1.873)%
  --(2.955,1.880)--(2.969,1.895)--(2.977,1.903)--(2.984,1.910)--(2.991,1.918)--(2.998,1.925)%
  --(3.005,1.930)--(3.013,1.938)--(3.020,1.945)--(3.027,1.950)--(3.034,1.958)--(3.041,1.965)%
  --(3.049,1.973)--(3.056,1.980)--(3.063,1.988)--(3.070,1.995)--(3.077,2.000)--(3.085,2.008)%
  --(3.092,2.015)--(3.099,2.023)--(3.106,2.030)--(3.113,2.037)--(3.121,2.045)--(3.128,2.052)%
  --(3.135,2.060)--(3.142,2.065)--(3.149,2.072)--(3.157,2.080)--(3.164,2.087)--(3.171,2.095)%
  --(3.178,2.102)--(3.185,2.110)--(3.193,2.117)--(3.200,2.125)--(3.207,2.132)--(3.214,2.137)%
  --(3.221,2.145)--(3.229,2.152)--(3.236,2.160)--(3.250,2.172)--(3.257,2.180)--(3.265,2.185)%
  --(3.272,2.192)--(3.279,2.197)--(3.286,2.205)--(3.293,2.210)--(3.301,2.217)--(3.308,2.222)%
  --(3.315,2.230)--(3.322,2.237)--(3.329,2.247)--(3.337,2.254)--(3.344,2.262)--(3.351,2.272)%
  --(3.358,2.279)--(3.365,2.287)--(3.373,2.294)--(3.380,2.302)--(3.387,2.309)--(3.394,2.317)%
  --(3.401,2.324)--(3.409,2.332)--(3.416,2.339)--(3.423,2.344)--(3.430,2.352)--(3.437,2.359)%
  --(3.444,2.367)--(3.452,2.374)--(3.459,2.382)--(3.466,2.389)--(3.473,2.397)--(3.480,2.402)%
  --(3.488,2.409)--(3.495,2.417)--(3.502,2.424)--(3.509,2.432)--(3.516,2.439)--(3.524,2.446)%
  --(3.531,2.454)--(3.538,2.461)--(3.545,2.469)--(3.552,2.476)--(3.560,2.484)--(3.567,2.491)%
  --(3.574,2.501)--(3.581,2.509)--(3.588,2.516)--(3.596,2.526)--(3.610,2.541)--(3.617,2.551)%
  --(3.624,2.561)--(3.632,2.574)--(3.639,2.586)--(3.646,2.599)--(3.653,2.611)--(3.660,2.626)%
  --(3.668,2.644)--(3.675,2.658)--(3.682,2.676)--(3.689,2.688)--(3.696,2.698)--(3.704,2.706)%
  --(3.711,2.711)--(3.718,2.716)--(3.725,2.721)--(3.732,2.726)--(3.740,2.731)--(3.747,2.736)%
  --(3.754,2.738)--(3.761,2.743)--(3.768,2.746)--(3.776,2.748)--(3.783,2.753)--(3.790,2.756)%
  --(3.797,2.758)--(3.804,2.763)--(3.812,2.766)--(3.819,2.771)--(3.826,2.773)--(3.833,2.778)%
  --(3.840,2.781)--(3.848,2.783)--(3.855,2.786)--(3.862,2.788)--(3.869,2.791)--(3.876,2.791)%
  --(3.884,2.793)--(3.891,2.796)--(3.898,2.796)--(3.905,2.798)--(3.912,2.801)--(3.920,2.803)%
  --(3.927,2.806)--(3.934,2.808)--(3.941,2.808)--(3.948,2.811)--(3.956,2.811)--(3.963,2.813)%
  --(3.970,2.813)--(3.977,2.816)--(3.984,2.818)--(3.992,2.818)--(3.999,2.821)--(4.006,2.821)%
  --(4.013,2.823)--(4.020,2.823)--(4.028,2.826)--(4.035,2.826)--(4.042,2.828)--(4.049,2.831)%
  --(4.056,2.833)--(4.064,2.836)--(4.071,2.838)--(4.078,2.841)--(4.085,2.846)--(4.092,2.846)%
  --(4.100,2.851)--(4.107,2.851)--(4.114,2.856)--(4.121,2.860)--(4.128,2.863)--(4.136,2.865)%
  --(4.143,2.868)--(4.150,2.870)--(4.157,2.873)--(4.164,2.875)--(4.172,2.878)--(4.179,2.880)%
  --(4.193,2.885)--(4.200,2.885)--(4.208,2.885)--(4.215,2.888)--(4.222,2.888)--(4.229,2.888)%
  --(4.236,2.888)--(4.244,2.888)--(4.251,2.888)--(4.265,2.888)--(4.272,2.888)--(4.280,2.888)%
  --(4.287,2.888)--(4.294,2.888)--(4.301,2.888)--(4.308,2.888)--(4.316,2.888)--(4.323,2.890)%
  --(4.330,2.893)--(4.337,2.890)--(4.344,2.888)--(4.352,2.883)--(4.359,2.880)--(4.366,2.878)%
  --(4.373,2.873)--(4.380,2.870)--(4.388,2.865)--(4.395,2.863)--(4.402,2.860)--(4.409,2.856)%
  --(4.416,2.853)--(4.424,2.848)--(4.431,2.846)--(4.438,2.841)--(4.445,2.838)--(4.452,2.833)%
  --(4.460,2.831)--(4.467,2.828)--(4.474,2.823)--(4.481,2.821)--(4.488,2.818)--(4.496,2.816)%
  --(4.503,2.811)--(4.510,2.808)--(4.517,2.806)--(4.524,2.803)--(4.532,2.801)--(4.539,2.798)%
  --(4.546,2.793)--(4.553,2.791)--(4.560,2.788)--(4.568,2.786)--(4.575,2.783)--(4.582,2.781)%
  --(4.589,2.778)--(4.596,2.776)--(4.603,2.771)--(4.611,2.768)--(4.618,2.763)--(4.625,2.758)%
  --(4.632,2.756)--(4.639,2.751)--(4.647,2.748)--(4.654,2.743)--(4.661,2.741)--(4.668,2.736)%
  --(4.675,2.733)--(4.683,2.728)--(4.690,2.726)--(4.697,2.723)--(4.704,2.718)--(4.711,2.716)%
  --(4.719,2.711)--(4.726,2.708)--(4.733,2.706)--(4.740,2.703)--(4.747,2.701)--(4.755,2.698)%
  --(4.762,2.696)--(4.769,2.693)--(4.776,2.691)--(4.783,2.688)--(4.791,2.686)--(4.798,2.683)%
  --(4.805,2.683)--(4.812,2.681)--(4.819,2.678)--(4.827,2.676)--(4.834,2.671)--(4.841,2.668)%
  --(4.848,2.663)--(4.855,2.661)--(4.863,2.658)--(4.870,2.653)--(4.877,2.651)--(4.884,2.646)%
  --(4.891,2.644)--(4.899,2.641)--(4.906,2.639)--(4.913,2.636)--(4.920,2.634)--(4.927,2.634)%
  --(4.935,2.631)--(4.942,2.631)--(4.949,2.631)--(4.956,2.631)--(4.963,2.629)--(4.971,2.624)%
  --(4.978,2.624)--(4.985,2.624)--(4.992,2.629)--(4.999,2.631)--(5.007,2.644)--(5.014,2.666)%
  --(5.021,2.698)--(5.028,2.733)--(5.035,2.761)--(5.043,2.791)--(5.050,2.816)--(5.057,2.841)%
  --(5.064,2.868)--(5.071,2.895)--(5.079,2.920)--(5.086,2.948)--(5.093,2.980)--(5.100,3.018)%
  --(5.107,3.053)--(5.115,3.087)--(5.122,3.127)--(5.136,3.200)--(5.143,3.232)--(5.151,3.270)%
  --(5.158,3.304)--(5.165,3.337)--(5.172,3.372)--(5.179,3.404)--(5.187,3.437)--(5.201,3.496)%
  --(5.215,3.551)--(5.223,3.574)--(5.230,3.601)--(5.237,3.624)--(5.244,3.646)--(5.251,3.671)%
  --(5.259,3.691)--(5.266,3.718)--(5.273,3.733)--(5.280,3.761)--(5.287,3.781)--(5.295,3.801)%
  --(5.302,3.821)--(5.309,3.838)--(5.316,3.861)--(5.323,3.873)--(5.331,3.893)--(5.338,3.913)%
  --(5.345,3.925)--(5.352,3.945)--(5.359,3.960)--(5.367,3.978)--(5.374,3.998)--(5.381,4.013)%
  --(5.388,4.035)--(5.395,4.048)--(5.403,4.068)--(5.410,4.080)--(5.417,4.100)--(5.424,4.115)%
  --(5.431,4.130)--(5.439,4.147)--(5.446,4.160)--(5.453,4.175)--(5.460,4.185)--(5.467,4.197)%
  --(5.475,4.210)--(5.482,4.220)--(5.489,4.235)--(5.496,4.247)--(5.503,4.260)--(5.511,4.272)%
  --(5.518,4.282)--(5.525,4.295)--(5.532,4.302)--(5.539,4.314)--(5.547,4.322)--(5.554,4.329)%
  --(5.561,4.337)--(5.568,4.344)--(5.575,4.352)--(5.583,4.359)--(5.590,4.367)--(5.597,4.374)%
  --(5.604,4.382)--(5.611,4.387)--(5.619,4.394)--(5.626,4.399)--(5.633,4.407)--(5.640,4.412)%
  --(5.647,4.419)--(5.655,4.424)--(5.662,4.432)--(5.669,4.437)--(5.676,4.442)--(5.683,4.447)%
  --(5.691,4.452)--(5.698,4.457)--(5.705,4.464)--(5.712,4.472)--(5.719,4.479)--(5.726,4.484)%
  --(5.734,4.492)--(5.741,4.497)--(5.748,4.502)--(5.755,4.509)--(5.762,4.514)--(5.770,4.519)%
  --(5.777,4.526)--(5.784,4.531)--(5.791,4.536)--(5.798,4.544)--(5.806,4.549)--(5.813,4.554)%
  --(5.820,4.559)--(5.827,4.566)--(5.834,4.571)--(5.842,4.576)--(5.849,4.584)--(5.856,4.591)%
  --(5.863,4.599)--(5.870,4.604)--(5.878,4.611)--(5.885,4.616)--(5.892,4.624)--(5.899,4.629)%
  --(5.906,4.634)--(5.914,4.639)--(5.921,4.644)--(5.928,4.649)--(5.935,4.654)--(5.942,4.661)%
  --(5.950,4.666)--(5.957,4.671)--(5.964,4.676)--(5.971,4.684)--(5.978,4.689)--(5.986,4.696)%
  --(5.993,4.704)--(6.000,4.711)--(6.007,4.716)--(6.014,4.724)--(6.022,4.731)--(6.029,4.736)%
  --(6.036,4.741)--(6.043,4.746)--(6.050,4.751)--(6.058,4.758)--(6.065,4.763)--(6.072,4.768)%
  --(6.079,4.773)--(6.086,4.778)--(6.094,4.783)--(6.101,4.791)--(6.108,4.796)--(6.115,4.801)%
  --(6.122,4.808)--(6.130,4.813)--(6.137,4.821)--(6.144,4.826)--(6.151,4.833)--(6.158,4.838)%
  --(6.166,4.846)--(6.173,4.851)--(6.180,4.856)--(6.187,4.861)--(6.194,4.868)--(6.202,4.873)%
  --(6.209,4.878)--(6.216,4.886)--(6.223,4.891)--(6.230,4.896)--(6.238,4.901)--(6.245,4.908)%
  --(6.252,4.913)--(6.259,4.921)--(6.266,4.928)--(6.288,4.948)--(6.295,4.953)--(6.302,4.960)%
  --(6.310,4.965)--(6.317,4.973)--(6.324,4.980)--(6.331,4.985)--(6.338,4.993)--(6.346,5.000)%
  --(6.353,5.008)--(6.360,5.015)--(6.367,5.020)--(6.374,5.028)--(6.382,5.033)--(6.389,5.040)%
  --(6.396,5.045)--(6.403,5.053)--(6.410,5.058)--(6.418,5.065)--(6.425,5.070)--(6.432,5.075)%
  --(6.439,5.080)--(6.446,5.088)--(6.454,5.093)--(6.461,5.098)--(6.468,5.103)--(6.475,5.108)%
  --(6.482,5.115)--(6.490,5.120)--(6.497,5.128)--(6.511,5.140)--(6.518,5.145)--(6.526,5.152)%
  --(6.533,5.157)--(6.540,5.165)--(6.547,5.170)--(6.554,5.175)--(6.562,5.182)--(6.569,5.187)%
  --(6.576,5.192)--(6.583,5.200)--(6.590,5.205)--(6.598,5.210)--(6.605,5.217)--(6.612,5.222)%
  --(6.619,5.230)--(6.626,5.237)--(6.634,5.242)--(6.641,5.247)--(6.648,5.255)--(6.655,5.260)%
  --(6.662,5.265)--(6.670,5.270)--(6.677,5.277)--(6.684,5.282)--(6.691,5.287)--(6.706,5.300)%
  --(6.713,5.307)--(6.720,5.312)--(6.727,5.317)--(6.734,5.325)--(6.742,5.330)--(6.749,5.335)%
  --(6.756,5.340)--(6.763,5.345)--(6.770,5.350)--(6.778,5.357)--(6.785,5.362)--(6.792,5.367)%
  --(6.799,5.372)--(6.806,5.379)--(6.814,5.384)--(6.821,5.389)--(6.828,5.394)--(6.835,5.399)%
  --(6.842,5.402)--(6.850,5.407)--(6.857,5.412)--(6.864,5.417)--(6.878,5.427)--(6.885,5.432)%
  --(6.893,5.437)--(6.900,5.439)--(6.907,5.444)--(6.914,5.449)--(6.921,5.452)--(6.929,5.457)%
  --(6.936,5.462)--(6.943,5.464)--(6.950,5.469)--(6.957,5.472)--(6.965,5.474)--(6.972,5.479)%
  --(6.979,5.482)--(6.986,5.484)--(6.993,5.487)--(7.001,5.492)--(7.008,5.494)--(7.015,5.494)%
  --(7.022,5.497)--(7.037,5.502)--(7.044,5.504)--(7.051,5.504)--(7.058,5.507)--(7.065,5.509)%
  --(7.073,5.509)--(7.080,5.512)--(7.087,5.512)--(7.094,5.512)--(7.101,5.512)--(7.109,5.512)%
  --(7.116,5.512)--(7.123,5.512)--(7.130,5.512)--(7.137,5.509)--(7.145,5.509)--(7.152,5.507)%
  --(7.159,5.507)--(7.166,5.504)--(7.173,5.502)--(7.188,5.497)--(7.195,5.494)--(7.202,5.492)%
  --(7.209,5.487)--(7.217,5.484)--(7.224,5.479)--(7.231,5.477)--(7.238,5.472)--(7.245,5.464)%
  --(7.253,5.462)--(7.260,5.457)--(7.267,5.449)--(7.274,5.442)--(7.281,5.437)--(7.289,5.429)%
  --(7.296,5.422)--(7.303,5.414)--(7.310,5.404)--(7.332,5.379)--(7.339,5.369)--(7.346,5.362)%
  --(7.353,5.350)--(7.361,5.340)--(7.368,5.330)--(7.375,5.320)--(7.382,5.307)--(7.389,5.295)%
  --(7.397,5.282)--(7.404,5.270)--(7.411,5.255)--(7.418,5.242)--(7.425,5.227)--(7.433,5.212)%
  --(7.447,5.182);
\gpsetdashtype{gp dt 2}
\draw[gp path] (1.688,5.305)--(1.695,5.295)--(1.702,5.282)--(1.710,5.272)--(1.717,5.262)%
  --(1.724,5.252)--(1.731,5.242)--(1.738,5.230)--(1.746,5.220)--(1.753,5.210)--(1.760,5.025)%
  --(1.767,5.008)--(1.774,4.990)--(1.782,4.973)--(1.789,4.953)--(1.796,4.933)--(1.803,4.886)%
  --(1.810,4.863)--(1.818,4.836)--(1.825,4.773)--(1.832,4.743)--(1.839,4.709)--(1.846,4.656)%
  --(1.854,4.624)--(1.861,4.586)--(1.868,4.541)--(1.875,4.509)--(1.882,4.474)--(1.890,4.424)%
  --(1.897,4.392)--(1.904,4.357)--(1.911,4.305)--(1.918,4.267)--(1.926,4.230)--(1.933,4.180)%
  --(1.940,4.135)--(1.947,4.093)--(1.954,4.040)--(1.962,3.998)--(1.969,3.945)--(1.976,3.900)%
  --(1.983,3.853)--(1.990,3.808)--(1.998,3.763)--(2.005,3.721)--(2.012,3.681)--(2.019,3.644)%
  --(2.026,3.604)--(2.034,3.569)--(2.041,3.534)--(2.048,3.501)--(2.055,3.469)--(2.062,3.437)%
  --(2.070,3.407)--(2.077,3.377)--(2.084,3.347)--(2.091,3.319)--(2.098,3.292)--(2.106,3.267)%
  --(2.113,3.240)--(2.120,3.215)--(2.127,3.192)--(2.134,3.167)--(2.142,3.145)--(2.149,3.122)%
  --(2.156,3.102)--(2.163,3.080)--(2.170,3.060)--(2.178,3.043)--(2.185,3.023)--(2.192,3.005)%
  --(2.199,2.988)--(2.206,2.973)--(2.214,2.955)--(2.221,2.940)--(2.228,2.925)--(2.235,2.910)%
  --(2.242,2.898)--(2.250,2.885)--(2.257,2.873)--(2.264,2.860)--(2.271,2.848)--(2.278,2.838)%
  --(2.285,2.828)--(2.293,2.818)--(2.300,2.808)--(2.307,2.798)--(2.314,2.791)--(2.321,2.783)%
  --(2.329,2.776)--(2.336,2.768)--(2.343,2.763)--(2.350,2.756)--(2.357,2.751)--(2.365,2.746)%
  --(2.372,2.741)--(2.379,2.738)--(2.386,2.733)--(2.393,2.728)--(2.401,2.726)--(2.408,2.723)%
  --(2.415,2.721)--(2.422,2.718)--(2.429,2.718)--(2.437,2.716)--(2.444,2.716)--(2.451,2.716)%
  --(2.458,2.716)--(2.465,2.716)--(2.473,2.716)--(2.480,2.716)--(2.487,2.718)--(2.494,2.718)%
  --(2.501,2.721)--(2.509,2.723)--(2.516,2.723)--(2.523,2.726)--(2.530,2.731)--(2.537,2.733)%
  --(2.545,2.736)--(2.552,2.726)--(2.559,2.711)--(2.581,2.753)--(2.588,2.736)--(2.595,2.711)%
  --(2.602,2.683)--(2.609,2.653)--(2.617,2.619)--(2.624,2.576)--(2.631,2.536)--(2.638,2.496)%
  --(2.645,2.456)--(2.653,2.417)--(2.660,2.384)--(2.667,2.349)--(2.674,2.314)--(2.681,2.282)%
  --(2.689,2.284)--(2.696,2.304)--(2.703,2.319)--(2.710,2.337)--(2.717,2.349)--(2.725,2.364)%
  --(2.732,2.379)--(2.739,2.392)--(2.746,2.407)--(2.753,2.422)--(2.761,2.434)--(2.768,2.446)%
  --(2.775,2.461)--(2.782,2.476)--(2.789,2.489)--(2.797,2.504)--(2.804,2.519)--(2.811,2.534)%
  --(2.818,2.551)--(2.825,2.571)--(2.833,2.589)--(2.840,2.609)--(2.847,2.631)--(2.854,2.656)%
  --(2.861,2.676)--(2.869,2.691)--(2.876,2.701)--(2.883,2.711)--(2.890,2.721)--(2.897,2.731)%
  --(2.905,2.741)--(2.912,2.751)--(2.919,2.758)--(2.926,2.768)--(2.933,2.776)--(2.941,2.783)%
  --(2.948,2.791)--(2.955,2.798)--(2.962,2.806)--(2.969,2.813)--(2.977,2.821)--(2.984,2.828)%
  --(2.998,2.841)--(3.005,2.846)--(3.013,2.856)--(3.020,2.863)--(3.027,2.870)--(3.034,2.878)%
  --(3.041,2.885)--(3.049,2.890)--(3.056,2.895)--(3.070,2.905)--(3.077,2.910)--(3.085,2.915)%
  --(3.092,2.920)--(3.099,2.925)--(3.106,2.928)--(3.113,2.933)--(3.121,2.938)--(3.128,2.940)%
  --(3.135,2.945)--(3.142,2.950)--(3.149,2.953)--(3.157,2.955)--(3.164,2.960)--(3.171,2.963)%
  --(3.178,2.965)--(3.185,2.970)--(3.193,2.973)--(3.200,2.975)--(3.207,2.978)--(3.214,2.983)%
  --(3.221,2.985)--(3.236,2.990)--(3.243,2.993)--(3.250,2.995)--(3.257,2.998)--(3.265,3.000)%
  --(3.272,3.003)--(3.279,3.003)--(3.286,3.005)--(3.293,3.008)--(3.301,3.010)--(3.315,3.015)%
  --(3.322,3.018)--(3.329,3.020)--(3.337,3.023)--(3.344,3.025)--(3.351,3.025)--(3.358,3.028)%
  --(3.365,3.030)--(3.373,3.030)--(3.380,3.033)--(3.387,3.033)--(3.401,3.035)--(3.409,3.038)%
  --(3.416,3.038)--(3.423,3.038)--(3.430,3.040)--(3.437,3.040)--(3.444,3.040)--(3.452,3.043)%
  --(3.459,3.043)--(3.466,3.043)--(3.473,3.045)--(3.480,3.045)--(3.495,3.045)--(3.502,3.045)%
  --(3.509,3.048)--(3.516,3.048)--(3.524,3.048)--(3.531,3.048)--(3.538,3.048)--(3.545,3.048)%
  --(3.552,3.050)--(3.560,3.050)--(3.567,3.050)--(3.574,3.050)--(3.588,3.050)--(3.596,3.050)%
  --(3.603,3.050)--(3.610,3.050)--(3.617,3.050)--(3.624,3.050)--(3.632,3.050)--(3.639,3.048)%
  --(3.646,3.048)--(3.653,3.048)--(3.660,3.048)--(3.668,3.048)--(3.675,3.048)--(3.689,3.048)%
  --(3.696,3.045)--(3.704,3.045)--(3.711,3.045)--(3.718,3.045)--(3.725,3.045)--(3.732,3.043)%
  --(3.740,3.043)--(3.747,3.043)--(3.754,3.043)--(3.761,3.040)--(3.768,3.040)--(3.776,3.040)%
  --(3.790,3.038)--(3.797,3.038)--(3.804,3.035)--(3.812,3.035)--(3.819,3.035)--(3.826,3.033)%
  --(3.833,3.033)--(3.840,3.030)--(3.848,3.030)--(3.855,3.030)--(3.862,3.028)--(3.869,3.028)%
  --(3.876,3.028)--(3.884,3.025)--(3.891,3.025)--(3.898,3.023)--(3.905,3.023)--(3.912,3.020)%
  --(3.920,3.020)--(3.927,3.018)--(3.934,3.018)--(3.941,3.015)--(3.948,3.015)--(3.956,3.013)%
  --(3.963,3.013)--(3.970,3.010)--(3.977,3.010)--(3.992,3.008)--(3.999,3.005)--(4.006,3.005)%
  --(4.013,3.003)--(4.020,3.000)--(4.028,3.000)--(4.035,2.998)--(4.042,2.995)--(4.049,2.995)%
  --(4.056,2.993)--(4.064,2.990)--(4.071,2.990)--(4.078,2.988)--(4.085,2.988)--(4.100,2.983)%
  --(4.107,2.980)--(4.114,2.980)--(4.121,2.978)--(4.128,2.975)--(4.136,2.973)--(4.143,2.970)%
  --(4.150,2.968)--(4.157,2.965)--(4.179,2.960)--(4.186,2.958)--(4.193,2.955)--(4.200,2.953)%
  --(4.215,2.948)--(4.222,2.945)--(4.229,2.940)--(4.236,2.938)--(4.244,2.935)--(4.251,2.930)%
  --(4.258,2.928)--(4.272,2.923)--(4.280,2.920)--(4.287,2.915)--(4.294,2.913)--(4.301,2.910)%
  --(4.308,2.908)--(4.316,2.905)--(4.323,2.903)--(4.337,2.900)--(4.344,2.898)--(4.352,2.895)%
  --(4.359,2.893)--(4.366,2.890)--(4.373,2.888)--(4.380,2.885)--(4.388,2.883)--(4.395,2.883)%
  --(4.402,2.880)--(4.409,2.878)--(4.416,2.875)--(4.424,2.873)--(4.431,2.873)--(4.438,2.870)%
  --(4.445,2.868)--(4.452,2.865)--(4.460,2.863)--(4.474,2.860)--(4.481,2.858)--(4.488,2.856)%
  --(4.496,2.853)--(4.503,2.853)--(4.510,2.851)--(4.517,2.848)--(4.524,2.846)--(4.532,2.843)%
  --(4.539,2.841)--(4.546,2.841)--(4.553,2.838)--(4.560,2.836)--(4.568,2.836)--(4.575,2.833)%
  --(4.582,2.831)--(4.589,2.831)--(4.596,2.828)--(4.603,2.826)--(4.611,2.823)--(4.618,2.821)%
  --(4.625,2.821)--(4.632,2.818)--(4.639,2.816)--(4.647,2.813)--(4.654,2.813)--(4.661,2.811)%
  --(4.675,2.806)--(4.683,2.806)--(4.690,2.803)--(4.697,2.803)--(4.704,2.801)--(4.711,2.798)%
  --(4.719,2.796)--(4.726,2.796)--(4.733,2.793)--(4.740,2.791)--(4.747,2.791)--(4.755,2.788)%
  --(4.762,2.788)--(4.769,2.786)--(4.776,2.786)--(4.783,2.783)--(4.791,2.783)--(4.805,2.778)%
  --(4.812,2.776)--(4.819,2.776)--(4.827,2.773)--(4.834,2.771)--(4.841,2.771)--(4.848,2.768)%
  --(4.855,2.768)--(4.863,2.768)--(4.870,2.766)--(4.877,2.763)--(4.884,2.763)--(4.891,2.761)%
  --(4.899,2.761)--(4.906,2.758)--(4.913,2.758)--(4.920,2.758)--(4.935,2.756)--(4.942,2.753)%
  --(4.949,2.753)--(4.956,2.753)--(4.963,2.753)--(4.971,2.751)--(4.978,2.751)--(4.985,2.751)%
  --(4.992,2.751)--(4.999,2.751)--(5.007,2.751)--(5.014,2.753)--(5.021,2.753)--(5.028,2.756)%
  --(5.035,2.758)--(5.050,2.763)--(5.057,2.768)--(5.064,2.771)--(5.071,2.773)--(5.079,2.778)%
  --(5.086,2.781)--(5.093,2.783)--(5.100,2.788)--(5.107,2.791)--(5.115,2.793)--(5.122,2.796)%
  --(5.129,2.798)--(5.136,2.803)--(5.143,2.806)--(5.151,2.811)--(5.158,2.813)--(5.172,2.821)%
  --(5.179,2.823)--(5.194,2.833)--(5.201,2.838)--(5.208,2.843)--(5.215,2.848)--(5.223,2.853)%
  --(5.230,2.856)--(5.237,2.860)--(5.244,2.863)--(5.251,2.865)--(5.273,2.878)--(5.280,2.880)%
  --(5.287,2.883)--(5.295,2.888)--(5.302,2.890)--(5.309,2.893)--(5.316,2.898)--(5.323,2.900)%
  --(5.331,2.903)--(5.338,2.908)--(5.345,2.910)--(5.352,2.913)--(5.367,2.920)--(5.374,2.925)%
  --(5.381,2.928)--(5.388,2.933)--(5.395,2.935)--(5.403,2.938)--(5.410,2.940)--(5.417,2.945)%
  --(5.424,2.948)--(5.431,2.950)--(5.439,2.955)--(5.446,2.958)--(5.460,2.965)--(5.467,2.968)%
  --(5.475,2.970)--(5.482,2.973)--(5.489,2.978)--(5.496,2.980)--(5.503,2.985)--(5.511,2.988)%
  --(5.518,2.993)--(5.525,2.998)--(5.532,3.000)--(5.547,3.008)--(5.554,3.010)--(5.561,3.015)%
  --(5.568,3.020)--(5.575,3.023)--(5.583,3.028)--(5.590,3.033)--(5.597,3.038)--(5.604,3.043)%
  --(5.611,3.050)--(5.619,3.053)--(5.626,3.058)--(5.633,3.063)--(5.640,3.067)--(5.647,3.072)%
  --(5.655,3.080)--(5.662,3.085)--(5.669,3.092)--(5.676,3.097)--(5.683,3.105)--(5.691,3.112)%
  --(5.698,3.117)--(5.705,3.125)--(5.712,3.132)--(5.719,3.140)--(5.726,3.145)--(5.734,3.155)%
  --(5.741,3.162)--(5.748,3.172)--(5.755,3.185)--(5.762,3.200)--(5.770,3.215)--(5.777,3.227)%
  --(5.784,3.240)--(5.791,3.252)--(5.798,3.265)--(5.806,3.277)--(5.813,3.292)--(5.820,3.304)%
  --(5.827,3.317)--(5.834,3.329)--(5.842,3.342)--(5.849,3.354)--(5.856,3.369)--(5.870,3.394)%
  --(5.878,3.407)--(5.885,3.419)--(5.892,3.432)--(5.899,3.447)--(5.906,3.459)--(5.914,3.472)%
  --(5.921,3.484)--(5.928,3.499)--(5.935,3.511)--(5.942,3.524)--(5.950,3.536)--(5.957,3.546)%
  --(5.964,3.556)--(5.971,3.569)--(5.978,3.581)--(5.986,3.591)--(6.000,3.614)--(6.007,3.624)%
  --(6.014,3.636)--(6.022,3.646)--(6.029,3.659)--(6.036,3.669)--(6.043,3.681)--(6.050,3.691)%
  --(6.058,3.701)--(6.065,3.713)--(6.072,3.723)--(6.079,3.733)--(6.086,3.746)--(6.094,3.756)%
  --(6.101,3.766)--(6.108,3.776)--(6.115,3.788)--(6.122,3.798)--(6.130,3.808)--(6.137,3.818)%
  --(6.151,3.841)--(6.158,3.851)--(6.166,3.861)--(6.173,3.871)--(6.180,3.881)--(6.194,3.900)%
  --(6.202,3.913)--(6.209,3.923)--(6.216,3.933)--(6.223,3.943)--(6.230,3.953)--(6.238,3.965)%
  --(6.245,3.975)--(6.252,3.985)--(6.259,3.995)--(6.274,4.018)--(6.281,4.030)--(6.288,4.045)%
  --(6.295,4.060)--(6.302,4.073)--(6.310,4.088)--(6.317,4.100)--(6.324,4.107)--(6.331,4.115)%
  --(6.338,4.125)--(6.346,4.132)--(6.353,4.140)--(6.360,4.147)--(6.367,4.155)--(6.374,4.162)%
  --(6.382,4.170)--(6.389,4.180)--(6.396,4.187)--(6.403,4.195)--(6.410,4.202)--(6.418,4.210)%
  --(6.425,4.217)--(6.432,4.225)--(6.439,4.232)--(6.446,4.240)--(6.454,4.245)--(6.461,4.252)%
  --(6.468,4.260)--(6.475,4.267)--(6.482,4.275)--(6.490,4.280)--(6.497,4.287)--(6.504,4.295)%
  --(6.511,4.300)--(6.518,4.307)--(6.526,4.314)--(6.533,4.319)--(6.540,4.327)--(6.547,4.332)%
  --(6.554,4.337)--(6.562,4.344)--(6.569,4.349)--(6.576,4.354)--(6.583,4.362)--(6.590,4.367)%
  --(6.598,4.372)--(6.605,4.377)--(6.612,4.382)--(6.619,4.387)--(6.626,4.392)--(6.634,4.397)%
  --(6.641,4.402)--(6.648,4.407)--(6.655,4.412)--(6.662,4.414)--(6.670,4.419)--(6.677,4.424)%
  --(6.684,4.427)--(6.691,4.432)--(6.698,4.434)--(6.706,4.437)--(6.713,4.442)--(6.720,4.444)%
  --(6.727,4.447)--(6.734,4.449)--(6.742,4.454)--(6.749,4.457)--(6.756,4.459)--(6.763,4.462)%
  --(6.770,4.462)--(6.778,4.464)--(6.785,4.467)--(6.792,4.467)--(6.799,4.469)--(6.806,4.472)%
  --(6.814,4.472)--(6.821,4.472)--(6.828,4.472)--(6.835,4.474)--(6.842,4.474)--(6.850,4.474)%
  --(6.857,4.474)--(6.864,4.474)--(6.871,4.472)--(6.878,4.472)--(6.885,4.472)--(6.893,4.469)%
  --(6.900,4.469)--(6.907,4.467)--(6.914,4.467)--(6.921,4.464)--(6.929,4.462)--(6.936,4.459)%
  --(6.943,4.457)--(6.950,4.454)--(6.957,4.449)--(6.965,4.447)--(6.972,4.442)--(6.979,4.439)%
  --(6.986,4.434)--(6.993,4.429)--(7.001,4.427)--(7.008,4.419)--(7.015,4.414)--(7.022,4.409)%
  --(7.029,4.404)--(7.037,4.399)--(7.044,4.392)--(7.051,4.384)--(7.058,4.377)--(7.065,4.369)%
  --(7.073,4.362)--(7.080,4.354)--(7.087,4.347)--(7.094,4.337)--(7.101,4.329)--(7.109,4.322)%
  --(7.116,4.312)--(7.123,4.302)--(7.130,4.292)--(7.137,4.282)--(7.145,4.272)--(7.152,4.260)%
  --(7.159,4.250)--(7.166,4.240)--(7.173,4.227)--(7.181,4.215)--(7.188,4.200)--(7.195,4.187)%
  --(7.202,4.175)--(7.209,4.160)--(7.217,4.145)--(7.224,4.130)--(7.231,4.115)--(7.238,4.098)%
  --(7.245,4.083)--(7.253,4.068)--(7.260,4.048)--(7.267,4.030)--(7.274,4.013)--(7.281,3.995)%
  --(7.289,3.973)--(7.296,3.955)--(7.303,3.935)--(7.310,3.913)--(7.317,3.893)--(7.325,3.873)%
  --(7.332,3.851)--(7.339,3.828)--(7.346,3.808)--(7.353,3.783)--(7.361,3.758)--(7.368,3.736)%
  --(7.375,3.711)--(7.382,3.686)--(7.389,3.664)--(7.397,3.636)--(7.404,3.611)--(7.411,3.584)%
  --(7.418,3.556)--(7.425,3.531)--(7.433,3.504)--(7.440,3.474)--(7.447,3.449);
\gpsetdashtype{gp dt 3}
\draw[gp path] (1.688,5.866)--(1.695,5.853)--(1.702,5.843)--(1.710,5.833)--(1.717,5.821)%
  --(1.724,5.811)--(1.731,5.801)--(1.738,5.788)--(1.746,5.778)--(1.753,5.766)--(1.760,5.756)%
  --(1.767,5.746)--(1.774,5.734)--(1.782,5.724)--(1.789,5.711)--(1.796,5.701)--(1.803,5.691)%
  --(1.810,5.679)--(1.818,5.669)--(1.825,5.656)--(1.832,5.644)--(1.839,5.631)--(1.846,5.619)%
  --(1.854,5.609)--(1.875,5.571)--(1.882,5.561)--(1.890,5.552)--(1.897,5.542)--(1.904,5.532)%
  --(1.911,5.522)--(1.918,5.512)--(1.926,5.502)--(1.933,5.492)--(1.940,5.482)--(1.947,5.472)%
  --(1.954,5.462)--(1.962,5.452)--(1.969,5.442)--(1.976,5.432)--(1.983,5.424)--(1.990,5.414)%
  --(1.998,5.404)--(2.005,5.394)--(2.012,5.384)--(2.019,5.374)--(2.026,5.364)--(2.034,5.354)%
  --(2.041,5.345)--(2.048,5.335)--(2.055,5.325)--(2.062,5.315)--(2.070,5.305)--(2.077,5.295)%
  --(2.084,5.285)--(2.091,5.272)--(2.098,5.262)--(2.106,5.252)--(2.113,5.240)--(2.120,5.230)%
  --(2.127,5.217)--(2.134,5.207)--(2.142,5.195)--(2.149,5.185)--(2.156,5.175)--(2.163,5.162)%
  --(2.170,5.152)--(2.178,5.142)--(2.185,5.130)--(2.192,5.120)--(2.199,5.110)--(2.206,5.100)%
  --(2.214,5.090)--(2.221,5.080)--(2.228,5.070)--(2.235,5.058)--(2.242,5.045)--(2.250,5.033)%
  --(2.257,5.018)--(2.264,5.005)--(2.271,4.995)--(2.278,4.983)--(2.285,4.973)--(2.293,4.960)%
  --(2.300,4.950)--(2.314,4.853)--(2.321,4.836)--(2.329,4.821)--(2.336,4.803)--(2.343,4.788)%
  --(2.350,4.773)--(2.357,4.756)--(2.365,4.738)--(2.379,4.586)--(2.386,4.546)--(2.401,4.437)%
  --(2.408,4.399)--(2.415,4.364)--(2.422,4.322)--(2.429,4.270)--(2.437,4.232)--(2.444,4.192)%
  --(2.451,4.155)--(2.458,4.112)--(2.465,4.065)--(2.473,4.028)--(2.480,3.988)--(2.487,3.948)%
  --(2.494,3.908)--(2.501,3.868)--(2.509,3.826)--(2.516,3.788)--(2.523,3.748)--(2.530,3.708)%
  --(2.537,3.671)--(2.545,3.636)--(2.552,3.601)--(2.559,3.566)--(2.566,3.529)--(2.573,3.494)%
  --(2.581,3.459)--(2.588,3.424)--(2.595,3.392)--(2.602,3.362)--(2.609,3.329)--(2.617,3.299)%
  --(2.624,3.272)--(2.631,3.242)--(2.638,3.215)--(2.645,3.190)--(2.653,3.165)--(2.660,3.140)%
  --(2.667,3.115)--(2.674,3.092)--(2.681,3.070)--(2.689,3.048)--(2.696,3.028)--(2.703,3.008)%
  --(2.710,2.990)--(2.717,2.973)--(2.725,2.958)--(2.732,2.943)--(2.739,2.928)--(2.746,2.915)%
  --(2.753,2.905)--(2.761,2.895)--(2.768,2.888)--(2.775,2.880)--(2.782,2.875)--(2.789,2.873)%
  --(2.797,2.870)--(2.804,2.873)--(2.811,2.875)--(2.818,2.883)--(2.825,2.895)--(2.833,2.913)%
  --(2.840,2.910)--(2.847,2.885)--(2.854,2.868)--(2.861,2.853)--(2.869,2.875)--(2.876,2.865)%
  --(2.883,2.851)--(2.890,2.833)--(2.897,2.811)--(2.905,2.786)--(2.912,2.761)--(2.919,2.738)%
  --(2.926,2.716)--(2.933,2.693)--(2.941,2.678)--(2.948,2.673)--(2.955,2.683)--(2.962,2.696)%
  --(2.969,2.708)--(2.977,2.721)--(2.984,2.728)--(2.991,2.741)--(2.998,2.748)--(3.005,2.758)%
  --(3.013,2.766)--(3.020,2.773)--(3.027,2.781)--(3.034,2.788)--(3.041,2.796)--(3.049,2.803)%
  --(3.063,2.816)--(3.070,2.821)--(3.077,2.826)--(3.085,2.831)--(3.092,2.838)--(3.106,2.846)%
  --(3.113,2.851)--(3.121,2.858)--(3.128,2.863)--(3.135,2.868)--(3.142,2.873)--(3.149,2.878)%
  --(3.157,2.883)--(3.164,2.885)--(3.171,2.890)--(3.178,2.893)--(3.185,2.898)--(3.193,2.900)%
  --(3.200,2.905)--(3.207,2.908)--(3.214,2.910)--(3.221,2.913)--(3.229,2.915)--(3.236,2.918)%
  --(3.243,2.923)--(3.250,2.925)--(3.257,2.928)--(3.272,2.933)--(3.279,2.935)--(3.286,2.938)%
  --(3.293,2.940)--(3.301,2.943)--(3.308,2.945)--(3.315,2.948)--(3.322,2.950)--(3.337,2.953)%
  --(3.344,2.955)--(3.351,2.955)--(3.358,2.958)--(3.365,2.960)--(3.373,2.960)--(3.380,2.963)%
  --(3.387,2.965)--(3.394,2.968)--(3.401,2.970)--(3.409,2.970)--(3.416,2.973)--(3.423,2.973)%
  --(3.430,2.975)--(3.437,2.975)--(3.444,2.978)--(3.452,2.978)--(3.459,2.980)--(3.466,2.980)%
  --(3.473,2.983)--(3.480,2.983)--(3.488,2.985)--(3.495,2.985)--(3.502,2.985)--(3.509,2.985)%
  --(3.516,2.988)--(3.524,2.988)--(3.531,2.988)--(3.538,2.990)--(3.545,2.990)--(3.560,2.990)%
  --(3.567,2.993)--(3.574,2.993)--(3.581,2.993)--(3.588,2.993)--(3.596,2.993)--(3.603,2.995)%
  --(3.610,2.995)--(3.617,2.995)--(3.624,2.995)--(3.639,2.995)--(3.646,2.995)--(3.653,2.995)%
  --(3.660,2.995)--(3.668,2.995)--(3.675,2.998)--(3.682,2.998)--(3.689,2.998)--(3.696,2.998)%
  --(3.704,2.998)--(3.711,2.998)--(3.718,2.998)--(3.725,2.998)--(3.732,2.995)--(3.740,2.995)%
  --(3.747,2.995)--(3.754,2.995)--(3.761,2.995)--(3.768,2.995)--(3.776,2.995)--(3.783,2.995)%
  --(3.790,2.995)--(3.797,2.995)--(3.804,2.995)--(3.812,2.993)--(3.819,2.993)--(3.826,2.993)%
  --(3.833,2.993)--(3.840,2.993)--(3.848,2.993)--(3.855,2.993)--(3.862,2.990)--(3.869,2.990)%
  --(3.876,2.990)--(3.884,2.990)--(3.891,2.988)--(3.898,2.988)--(3.905,2.988)--(3.912,2.988)%
  --(3.920,2.985)--(3.927,2.985)--(3.934,2.985)--(3.941,2.985)--(3.948,2.985)--(3.956,2.983)%
  --(3.963,2.983)--(3.970,2.983)--(3.977,2.980)--(3.984,2.980)--(3.992,2.980)--(3.999,2.978)%
  --(4.006,2.978)--(4.013,2.978)--(4.020,2.975)--(4.028,2.975)--(4.035,2.975)--(4.042,2.973)%
  --(4.049,2.973)--(4.056,2.973)--(4.064,2.970)--(4.071,2.970)--(4.078,2.968)--(4.085,2.968)%
  --(4.092,2.968)--(4.100,2.965)--(4.107,2.965)--(4.114,2.963)--(4.121,2.963)--(4.128,2.960)%
  --(4.136,2.960)--(4.143,2.960)--(4.150,2.958)--(4.157,2.958)--(4.164,2.955)--(4.172,2.955)%
  --(4.186,2.953)--(4.193,2.950)--(4.200,2.948)--(4.208,2.948)--(4.215,2.945)--(4.222,2.943)%
  --(4.229,2.943)--(4.236,2.940)--(4.244,2.938)--(4.251,2.938)--(4.265,2.933)--(4.272,2.930)%
  --(4.280,2.930)--(4.287,2.928)--(4.294,2.925)--(4.301,2.923)--(4.308,2.920)--(4.316,2.918)%
  --(4.323,2.918)--(4.330,2.915)--(4.337,2.913)--(4.344,2.910)--(4.352,2.908)--(4.359,2.905)%
  --(4.366,2.905)--(4.373,2.903)--(4.380,2.900)--(4.388,2.900)--(4.395,2.898)--(4.402,2.895)%
  --(4.409,2.895)--(4.416,2.893)--(4.424,2.893)--(4.431,2.890)--(4.438,2.888)--(4.445,2.888)%
  --(4.452,2.885)--(4.460,2.885)--(4.467,2.883)--(4.474,2.880)--(4.481,2.880)--(4.488,2.878)%
  --(4.496,2.878)--(4.503,2.875)--(4.510,2.875)--(4.517,2.873)--(4.532,2.873)--(4.539,2.870)%
  --(4.546,2.868)--(4.553,2.868)--(4.560,2.868)--(4.568,2.865)--(4.575,2.865)--(4.582,2.863)%
  --(4.589,2.863)--(4.596,2.860)--(4.603,2.860)--(4.611,2.858)--(4.618,2.858)--(4.625,2.858)%
  --(4.632,2.856)--(4.639,2.856)--(4.647,2.853)--(4.654,2.853)--(4.661,2.853)--(4.668,2.851)%
  --(4.675,2.851)--(4.683,2.851)--(4.690,2.848)--(4.697,2.848)--(4.704,2.846)--(4.711,2.846)%
  --(4.719,2.843)--(4.726,2.843)--(4.733,2.843)--(4.740,2.843)--(4.747,2.841)--(4.755,2.841)%
  --(4.762,2.838)--(4.769,2.838)--(4.776,2.838)--(4.783,2.838)--(4.791,2.836)--(4.798,2.836)%
  --(4.805,2.836)--(4.812,2.836)--(4.827,2.833)--(4.834,2.833)--(4.841,2.833)--(4.848,2.833)%
  --(4.855,2.833)--(4.863,2.833)--(4.870,2.833)--(4.877,2.833)--(4.884,2.833)--(4.891,2.833)%
  --(4.906,2.836)--(4.913,2.836)--(4.920,2.836)--(4.927,2.836)--(4.935,2.838)--(4.942,2.838)%
  --(4.949,2.841)--(4.956,2.841)--(4.971,2.843)--(4.985,2.846)--(4.992,2.846)--(4.999,2.848)%
  --(5.007,2.851)--(5.014,2.851)--(5.021,2.853)--(5.028,2.853)--(5.035,2.856)--(5.050,2.863)%
  --(5.057,2.865)--(5.064,2.868)--(5.071,2.870)--(5.079,2.873)--(5.086,2.875)--(5.093,2.878)%
  --(5.100,2.880)--(5.107,2.883)--(5.115,2.885)--(5.122,2.885)--(5.129,2.888)--(5.136,2.890)%
  --(5.151,2.893)--(5.158,2.898)--(5.165,2.898)--(5.172,2.900)--(5.179,2.903)--(5.187,2.905)%
  --(5.194,2.908)--(5.201,2.908)--(5.208,2.910)--(5.215,2.913)--(5.223,2.913)--(5.230,2.915)%
  --(5.237,2.918)--(5.244,2.920)--(5.251,2.920)--(5.259,2.923)--(5.266,2.925)--(5.273,2.925)%
  --(5.280,2.928)--(5.287,2.930)--(5.295,2.930)--(5.302,2.933)--(5.309,2.933)--(5.316,2.935)%
  --(5.323,2.938)--(5.331,2.938)--(5.338,2.940)--(5.345,2.943)--(5.352,2.943)--(5.359,2.945)%
  --(5.374,2.948)--(5.381,2.950)--(5.388,2.953)--(5.395,2.953)--(5.403,2.955)--(5.410,2.958)%
  --(5.417,2.960)--(5.424,2.963)--(5.431,2.965)--(5.439,2.968)--(5.446,2.968)--(5.453,2.973)%
  --(5.460,2.975)--(5.467,2.978)--(5.475,2.980)--(5.482,2.983)--(5.489,2.985)--(5.496,2.988)%
  --(5.503,2.990)--(5.511,2.993)--(5.518,2.995)--(5.525,2.998)--(5.532,3.000)--(5.547,3.008)%
  --(5.554,3.010)--(5.561,3.013)--(5.568,3.018)--(5.575,3.020)--(5.583,3.023)--(5.590,3.025)%
  --(5.597,3.030)--(5.604,3.033)--(5.611,3.035)--(5.619,3.038)--(5.626,3.043)--(5.633,3.045)%
  --(5.640,3.050)--(5.647,3.053)--(5.655,3.058)--(5.662,3.060)--(5.676,3.067)--(5.683,3.072)%
  --(5.691,3.077)--(5.698,3.080)--(5.705,3.085)--(5.712,3.090)--(5.719,3.095)--(5.734,3.105)%
  --(5.741,3.110)--(5.748,3.115)--(5.755,3.120)--(5.762,3.125)--(5.770,3.132)--(5.777,3.137)%
  --(5.784,3.145)--(5.791,3.150)--(5.798,3.157)--(5.806,3.165)--(5.813,3.172)--(5.820,3.180)%
  --(5.827,3.190)--(5.834,3.195)--(5.842,3.200)--(5.849,3.207)--(5.856,3.212)--(5.863,3.220)%
  --(5.870,3.225)--(5.878,3.232)--(5.885,3.240)--(5.892,3.247)--(5.899,3.255)--(5.906,3.262)%
  --(5.914,3.270)--(5.921,3.277)--(5.928,3.284)--(5.935,3.294)--(5.942,3.302)--(5.950,3.312)%
  --(5.957,3.319)--(5.964,3.329)--(5.971,3.339)--(5.978,3.349)--(5.986,3.359)--(5.993,3.369)%
  --(6.000,3.382)--(6.007,3.392)--(6.014,3.404)--(6.022,3.417)--(6.029,3.434)--(6.036,3.457)%
  --(6.050,3.499)--(6.058,3.516)--(6.065,3.534)--(6.072,3.551)--(6.079,3.566)--(6.086,3.584)%
  --(6.094,3.601)--(6.101,3.619)--(6.108,3.634)--(6.115,3.651)--(6.122,3.666)--(6.130,3.684)%
  --(6.137,3.698)--(6.144,3.716)--(6.151,3.731)--(6.158,3.746)--(6.166,3.761)--(6.173,3.776)%
  --(6.180,3.793)--(6.187,3.808)--(6.194,3.823)--(6.202,3.838)--(6.209,3.853)--(6.216,3.866)%
  --(6.223,3.881)--(6.230,3.895)--(6.238,3.910)--(6.245,3.923)--(6.252,3.938)--(6.259,3.950)%
  --(6.266,3.965)--(6.274,3.978)--(6.281,3.990)--(6.288,4.005)--(6.295,4.018)--(6.302,4.033)%
  --(6.310,4.045)--(6.317,4.060)--(6.324,4.075)--(6.331,4.088)--(6.338,4.103)--(6.346,4.117)%
  --(6.353,4.132)--(6.360,4.145)--(6.367,4.160)--(6.374,4.175)--(6.382,4.190)--(6.389,4.205)%
  --(6.396,4.217)--(6.403,4.227)--(6.410,4.237)--(6.418,4.245)--(6.425,4.255)--(6.432,4.265)%
  --(6.439,4.275)--(6.446,4.282)--(6.454,4.292)--(6.461,4.300)--(6.468,4.310)--(6.475,4.317)%
  --(6.482,4.324)--(6.490,4.332)--(6.497,4.339)--(6.504,4.347)--(6.511,4.354)--(6.518,4.362)%
  --(6.526,4.369)--(6.533,4.374)--(6.540,4.382)--(6.547,4.387)--(6.554,4.392)--(6.562,4.397)%
  --(6.569,4.402)--(6.576,4.407)--(6.583,4.412)--(6.590,4.414)--(6.598,4.419)--(6.605,4.422)%
  --(6.612,4.424)--(6.619,4.427)--(6.626,4.429)--(6.634,4.432)--(6.641,4.432)--(6.648,4.434)%
  --(6.655,4.434)--(6.662,4.434)--(6.670,4.434)--(6.677,4.432)--(6.684,4.432)--(6.691,4.429)%
  --(6.698,4.427)--(6.706,4.424)--(6.713,4.419)--(6.720,4.417)--(6.727,4.412)--(6.734,4.407)%
  --(6.742,4.399)--(6.749,4.394)--(6.756,4.387)--(6.763,4.379)--(6.770,4.372)--(6.778,4.364)%
  --(6.785,4.354)--(6.792,4.347)--(6.799,4.334)--(6.806,4.324)--(6.814,4.314)--(6.821,4.302)%
  --(6.828,4.290)--(6.835,4.277)--(6.842,4.265)--(6.850,4.252)--(6.857,4.237)--(6.864,4.222)%
  --(6.871,4.207)--(6.878,4.192)--(6.885,4.175)--(6.893,4.160)--(6.900,4.142)--(6.907,4.125)%
  --(6.914,4.105)--(6.921,4.088)--(6.929,4.068)--(6.936,4.048)--(6.943,4.028)--(6.950,4.005)%
  --(6.957,3.985)--(6.965,3.963)--(6.972,3.940)--(6.979,3.918)--(6.986,3.895)--(6.993,3.871)%
  --(7.001,3.846)--(7.008,3.823)--(7.015,3.796)--(7.022,3.771)--(7.029,3.743)--(7.037,3.718)%
  --(7.044,3.691)--(7.051,3.664)--(7.058,3.636)--(7.065,3.609)--(7.073,3.579)--(7.080,3.551)%
  --(7.087,3.519)--(7.094,3.491)--(7.101,3.462)--(7.109,3.432)--(7.116,3.399)--(7.123,3.369)%
  --(7.130,3.337)--(7.137,3.307)--(7.145,3.274)--(7.152,3.242)--(7.159,3.210)--(7.166,3.177)%
  --(7.173,3.145)--(7.181,3.112)--(7.188,3.077)--(7.195,3.045)--(7.202,3.010)--(7.209,2.978)%
  --(7.217,2.940)--(7.224,2.908)--(7.231,2.870)--(7.238,2.836)--(7.245,2.801)--(7.253,2.763)%
  --(7.260,2.728)--(7.267,2.693)--(7.274,2.653)--(7.281,2.619)--(7.289,2.584)--(7.296,2.544)%
  --(7.303,2.509)--(7.310,2.474)--(7.317,2.434)--(7.325,2.394)--(7.332,2.359)--(7.339,2.324)%
  --(7.346,2.282)--(7.353,2.244)--(7.361,2.210)--(7.368,2.175)--(7.375,2.132)--(7.382,2.092)%
  --(7.389,2.057)--(7.397,2.025)--(7.404,1.985)--(7.411,1.938)--(7.418,1.903)--(7.425,1.868)%
  --(7.433,1.808)--(7.440,1.768)--(7.447,1.731);
\gpsetdashtype{gp dt 4}
\draw[gp path] (1.688,5.484)--(1.695,5.474)--(1.702,5.464)--(1.710,5.454)--(1.717,5.442)%
  --(1.724,5.432)--(1.731,5.422)--(1.738,5.412)--(1.746,5.402)--(1.753,5.389)--(1.760,5.379)%
  --(1.767,5.369)--(1.774,5.359)--(1.782,5.350)--(1.789,5.340)--(1.796,5.327)--(1.803,5.317)%
  --(1.810,5.307)--(1.818,5.297)--(1.825,5.287)--(1.832,5.277)--(1.839,5.267)--(1.846,5.257)%
  --(1.854,5.247)--(1.861,5.235)--(1.868,5.225)--(1.875,5.215)--(1.882,5.205)--(1.890,5.195)%
  --(1.897,5.185)--(1.904,5.175)--(1.911,5.165)--(1.918,5.152)--(1.926,5.142)--(1.933,5.133)%
  --(1.940,5.123)--(1.947,5.113)--(1.954,5.103)--(1.962,5.093)--(1.969,5.083)--(1.976,5.073)%
  --(1.983,5.063)--(1.990,5.053)--(1.998,5.043)--(2.005,5.033)--(2.012,5.023)--(2.019,5.013)%
  --(2.026,5.000)--(2.034,4.990)--(2.041,4.980)--(2.048,4.970)--(2.055,4.960)--(2.062,4.950)%
  --(2.070,4.935)--(2.077,4.928)--(2.084,4.918)--(2.091,4.908)--(2.098,4.898)--(2.106,4.891)%
  --(2.113,4.881)--(2.120,4.871)--(2.127,4.863)--(2.134,4.853)--(2.142,4.846)--(2.149,4.836)%
  --(2.156,4.828)--(2.163,4.818)--(2.170,4.811)--(2.178,4.801)--(2.185,4.793)--(2.192,4.783)%
  --(2.199,4.773)--(2.206,4.766)--(2.214,4.756)--(2.221,4.748)--(2.228,4.738)--(2.235,4.731)%
  --(2.242,4.721)--(2.250,4.711)--(2.257,4.704)--(2.264,4.694)--(2.271,4.684)--(2.278,4.674)%
  --(2.285,4.664)--(2.293,4.654)--(2.300,4.644)--(2.307,4.634)--(2.314,4.621)--(2.321,4.611)%
  --(2.329,4.601)--(2.336,4.591)--(2.343,4.581)--(2.350,4.571)--(2.357,4.561)--(2.365,4.551)%
  --(2.372,4.541)--(2.379,4.534)--(2.386,4.524)--(2.393,4.517)--(2.401,4.507)--(2.408,4.499)%
  --(2.415,4.489)--(2.422,4.482)--(2.429,4.472)--(2.437,4.464)--(2.444,4.454)--(2.451,4.444)%
  --(2.458,4.434)--(2.465,4.424)--(2.473,4.417)--(2.480,4.407)--(2.487,4.397)--(2.494,4.384)%
  --(2.501,4.374)--(2.509,4.362)--(2.516,4.352)--(2.523,4.342)--(2.530,4.332)--(2.537,4.322)%
  --(2.545,4.312)--(2.552,4.302)--(2.559,4.295)--(2.566,4.285)--(2.573,4.275)--(2.581,4.265)%
  --(2.588,4.257)--(2.595,4.247)--(2.602,4.237)--(2.609,4.227)--(2.617,4.217)--(2.624,4.150)%
  --(2.631,4.132)--(2.638,4.117)--(2.645,4.105)--(2.653,4.093)--(2.660,4.080)--(2.667,4.068)%
  --(2.674,4.055)--(2.681,4.043)--(2.689,4.030)--(2.696,4.018)--(2.703,4.005)--(2.710,3.993)%
  --(2.717,3.980)--(2.725,3.965)--(2.732,3.950)--(2.739,3.935)--(2.746,3.920)--(2.753,3.900)%
  --(2.761,3.883)--(2.768,3.858)--(2.775,3.826)--(2.782,3.781)--(2.789,3.733)--(2.797,3.676)%
  --(2.804,3.589)--(2.811,3.504)--(2.818,3.444)--(2.825,3.382)--(2.833,3.260)--(2.840,3.187)%
  --(2.847,3.137)--(2.854,3.075)--(2.861,3.008)--(2.869,2.943)--(2.890,2.873)--(2.897,2.823)%
  --(2.905,2.758)--(2.912,2.681)--(2.919,2.591)--(2.926,2.486)--(2.933,2.357)--(2.941,2.227)%
  --(2.948,2.237)--(2.955,2.274)--(2.962,2.314)--(2.969,2.347)--(2.977,2.377)--(2.984,2.404)%
  --(2.991,2.434)--(2.998,2.464)--(3.005,2.486)--(3.013,2.506)--(3.020,2.526)--(3.027,2.544)%
  --(3.034,2.559)--(3.041,2.574)--(3.049,2.586)--(3.056,2.601)--(3.063,2.614)--(3.070,2.624)%
  --(3.077,2.636)--(3.085,2.646)--(3.092,2.656)--(3.099,2.666)--(3.106,2.676)--(3.113,2.683)%
  --(3.121,2.693)--(3.128,2.701)--(3.135,2.708)--(3.142,2.716)--(3.149,2.723)--(3.157,2.731)%
  --(3.164,2.736)--(3.171,2.743)--(3.185,2.756)--(3.193,2.761)--(3.200,2.766)--(3.207,2.768)%
  --(3.214,2.773)--(3.221,2.778)--(3.229,2.791)--(3.236,2.798)--(3.243,2.803)--(3.257,2.813)%
  --(3.265,2.818)--(3.272,2.823)--(3.279,2.826)--(3.286,2.828)--(3.293,2.833)--(3.301,2.836)%
  --(3.308,2.841)--(3.315,2.843)--(3.322,2.846)--(3.329,2.851)--(3.337,2.853)--(3.344,2.856)%
  --(3.351,2.858)--(3.358,2.860)--(3.365,2.863)--(3.373,2.865)--(3.380,2.868)--(3.387,2.870)%
  --(3.394,2.873)--(3.401,2.878)--(3.409,2.880)--(3.416,2.880)--(3.423,2.883)--(3.430,2.885)%
  --(3.437,2.888)--(3.444,2.888)--(3.452,2.890)--(3.459,2.893)--(3.466,2.895)--(3.473,2.895)%
  --(3.480,2.898)--(3.488,2.900)--(3.495,2.900)--(3.502,2.903)--(3.509,2.903)--(3.516,2.908)%
  --(3.524,2.910)--(3.531,2.913)--(3.538,2.915)--(3.545,2.915)--(3.552,2.918)--(3.560,2.918)%
  --(3.567,2.920)--(3.574,2.920)--(3.581,2.920)--(3.588,2.923)--(3.596,2.923)--(3.603,2.923)%
  --(3.610,2.923)--(3.617,2.923)--(3.624,2.925)--(3.632,2.925)--(3.639,2.925)--(3.646,2.925)%
  --(3.653,2.928)--(3.660,2.928)--(3.668,2.928)--(3.675,2.928)--(3.682,2.928)--(3.689,2.928)%
  --(3.696,2.928)--(3.704,2.928)--(3.711,2.928)--(3.718,2.928)--(3.725,2.928)--(3.732,2.928)%
  --(3.740,2.928)--(3.747,2.930)--(3.754,2.930)--(3.761,2.930)--(3.768,2.928)--(3.776,2.930)%
  --(3.783,2.930)--(3.790,2.930)--(3.797,2.930)--(3.804,2.930)--(3.812,2.928)--(3.819,2.928)%
  --(3.826,2.928)--(3.833,2.928)--(3.840,2.928)--(3.848,2.928)--(3.862,2.928)--(3.869,2.925)%
  --(3.876,2.925)--(3.884,2.925)--(3.891,2.923)--(3.898,2.923)--(3.905,2.923)--(3.912,2.920)%
  --(3.920,2.920)--(3.927,2.920)--(3.934,2.918)--(3.941,2.918)--(3.948,2.918)--(3.956,2.918)%
  --(3.970,2.915)--(3.977,2.913)--(3.984,2.913)--(3.992,2.913)--(3.999,2.910)--(4.006,2.910)%
  --(4.013,2.908)--(4.020,2.908)--(4.028,2.908)--(4.035,2.905)--(4.042,2.905)--(4.049,2.903)%
  --(4.056,2.903)--(4.064,2.903)--(4.078,2.900)--(4.085,2.898)--(4.092,2.898)--(4.100,2.895)%
  --(4.107,2.895)--(4.114,2.893)--(4.121,2.890)--(4.128,2.890)--(4.136,2.888)--(4.143,2.888)%
  --(4.150,2.885)--(4.157,2.885)--(4.164,2.885)--(4.172,2.883)--(4.179,2.883)--(4.193,2.878)%
  --(4.200,2.878)--(4.208,2.875)--(4.215,2.873)--(4.222,2.873)--(4.229,2.870)--(4.236,2.870)%
  --(4.244,2.868)--(4.251,2.865)--(4.258,2.865)--(4.265,2.863)--(4.272,2.863)--(4.280,2.860)%
  --(4.287,2.858)--(4.294,2.858)--(4.308,2.853)--(4.316,2.853)--(4.323,2.851)--(4.330,2.848)%
  --(4.337,2.846)--(4.344,2.843)--(4.352,2.841)--(4.359,2.838)--(4.366,2.836)--(4.373,2.831)%
  --(4.380,2.828)--(4.388,2.826)--(4.395,2.823)--(4.402,2.821)--(4.409,2.816)--(4.416,2.813)%
  --(4.424,2.811)--(4.438,2.803)--(4.445,2.801)--(4.452,2.798)--(4.460,2.793)--(4.467,2.791)%
  --(4.474,2.791)--(4.481,2.788)--(4.488,2.786)--(4.496,2.783)--(4.503,2.783)--(4.510,2.781)%
  --(4.517,2.778)--(4.524,2.778)--(4.532,2.776)--(4.539,2.773)--(4.546,2.771)--(4.553,2.771)%
  --(4.560,2.768)--(4.568,2.766)--(4.575,2.766)--(4.582,2.763)--(4.589,2.761)--(4.596,2.761)%
  --(4.603,2.758)--(4.611,2.758)--(4.618,2.756)--(4.625,2.753)--(4.632,2.751)--(4.639,2.751)%
  --(4.647,2.748)--(4.661,2.746)--(4.668,2.743)--(4.675,2.743)--(4.683,2.741)--(4.690,2.741)%
  --(4.697,2.738)--(4.704,2.738)--(4.711,2.736)--(4.719,2.736)--(4.726,2.736)--(4.733,2.733)%
  --(4.747,2.733)--(4.755,2.733)--(4.762,2.731)--(4.769,2.731)--(4.776,2.731)--(4.783,2.731)%
  --(4.791,2.728)--(4.798,2.728)--(4.805,2.728)--(4.812,2.728)--(4.819,2.728)--(4.827,2.728)%
  --(4.834,2.728)--(4.841,2.731)--(4.848,2.731)--(4.863,2.736)--(4.870,2.741)--(4.877,2.746)%
  --(4.884,2.751)--(4.891,2.758)--(4.899,2.763)--(4.906,2.771)--(4.913,2.776)--(4.920,2.783)%
  --(4.927,2.788)--(4.935,2.793)--(4.942,2.798)--(4.949,2.801)--(4.956,2.803)--(4.971,2.811)%
  --(4.978,2.813)--(4.985,2.818)--(4.992,2.821)--(4.999,2.826)--(5.007,2.828)--(5.014,2.833)%
  --(5.021,2.838)--(5.028,2.843)--(5.035,2.846)--(5.043,2.848)--(5.050,2.851)--(5.057,2.856)%
  --(5.064,2.858)--(5.079,2.863)--(5.086,2.865)--(5.093,2.868)--(5.100,2.868)--(5.107,2.868)%
  --(5.115,2.870)--(5.122,2.870)--(5.129,2.873)--(5.136,2.873)--(5.143,2.873)--(5.151,2.875)%
  --(5.158,2.875)--(5.165,2.878)--(5.172,2.878)--(5.179,2.880)--(5.187,2.880)--(5.194,2.883)%
  --(5.201,2.883)--(5.208,2.885)--(5.215,2.885)--(5.223,2.888)--(5.230,2.888)--(5.237,2.890)%
  --(5.244,2.890)--(5.251,2.893)--(5.259,2.895)--(5.266,2.898)--(5.280,2.900)--(5.287,2.903)%
  --(5.295,2.905)--(5.302,2.905)--(5.309,2.908)--(5.316,2.910)--(5.323,2.913)--(5.331,2.915)%
  --(5.338,2.918)--(5.345,2.920)--(5.352,2.923)--(5.359,2.925)--(5.374,2.930)--(5.381,2.933)%
  --(5.388,2.935)--(5.395,2.938)--(5.403,2.940)--(5.410,2.943)--(5.417,2.945)--(5.424,2.948)%
  --(5.431,2.950)--(5.439,2.953)--(5.446,2.958)--(5.460,2.963)--(5.467,2.965)--(5.475,2.970)%
  --(5.482,2.973)--(5.489,2.978)--(5.496,2.980)--(5.503,2.985)--(5.511,2.988)--(5.518,2.993)%
  --(5.525,2.998)--(5.539,3.008)--(5.547,3.013)--(5.554,3.018)--(5.561,3.023)--(5.568,3.030)%
  --(5.575,3.038)--(5.583,3.045)--(5.590,3.053)--(5.597,3.058)--(5.604,3.063)--(5.611,3.067)%
  --(5.619,3.072)--(5.626,3.077)--(5.633,3.082)--(5.640,3.087)--(5.647,3.095)--(5.655,3.100)%
  --(5.662,3.107)--(5.676,3.117)--(5.683,3.125)--(5.691,3.132)--(5.698,3.137)--(5.705,3.145)%
  --(5.712,3.152)--(5.719,3.160)--(5.726,3.165)--(5.734,3.172)--(5.741,3.180)--(5.748,3.187)%
  --(5.755,3.195)--(5.762,3.202)--(5.770,3.212)--(5.777,3.220)--(5.784,3.230)--(5.791,3.240)%
  --(5.798,3.250)--(5.806,3.260)--(5.813,3.274)--(5.820,3.287)--(5.827,3.302)--(5.834,3.314)%
  --(5.842,3.329)--(5.849,3.344)--(5.856,3.359)--(5.863,3.372)--(5.870,3.387)--(5.878,3.402)%
  --(5.885,3.417)--(5.892,3.429)--(5.899,3.444)--(5.906,3.459)--(5.914,3.472)--(5.921,3.486)%
  --(5.928,3.501)--(5.935,3.514)--(5.942,3.529)--(5.950,3.544)--(5.957,3.559)--(5.964,3.571)%
  --(5.971,3.586)--(5.978,3.601)--(5.986,3.616)--(5.993,3.634)--(6.000,3.649)--(6.007,3.664)%
  --(6.014,3.681)--(6.022,3.698)--(6.029,3.713)--(6.036,3.733)--(6.043,3.751)--(6.050,3.773)%
  --(6.058,3.796)--(6.086,3.861)--(6.094,3.878)--(6.101,3.898)--(6.108,3.918)--(6.115,3.940)%
  --(6.122,3.958)--(6.130,3.970)--(6.137,3.980)--(6.144,3.993)--(6.151,4.003)--(6.158,4.013)%
  --(6.166,4.025)--(6.173,4.035)--(6.180,4.045)--(6.187,4.058)--(6.194,4.068)--(6.202,4.080)%
  --(6.209,4.093)--(6.216,4.105)--(6.223,4.117)--(6.230,4.127)--(6.238,4.140)--(6.245,4.152)%
  --(6.252,4.165)--(6.259,4.177)--(6.266,4.190)--(6.274,4.200)--(6.281,4.212)--(6.288,4.225)%
  --(6.295,4.237)--(6.302,4.250)--(6.310,4.260)--(6.317,4.272)--(6.324,4.285)--(6.331,4.297)%
  --(6.338,4.307)--(6.346,4.319)--(6.353,4.332)--(6.360,4.342)--(6.367,4.354)--(6.374,4.367)%
  --(6.382,4.377)--(6.389,4.387)--(6.396,4.399)--(6.403,4.409)--(6.410,4.422)--(6.418,4.432)%
  --(6.425,4.442)--(6.432,4.452)--(6.439,4.462)--(6.446,4.472)--(6.454,4.482)--(6.461,4.489)%
  --(6.468,4.499)--(6.475,4.507)--(6.482,4.514)--(6.490,4.524)--(6.497,4.531)--(6.504,4.536)%
  --(6.511,4.544)--(6.518,4.549)--(6.526,4.556)--(6.533,4.561)--(6.540,4.566)--(6.547,4.569)%
  --(6.554,4.574)--(6.562,4.576)--(6.569,4.579)--(6.576,4.581)--(6.583,4.584)--(6.590,4.584)%
  --(6.598,4.584)--(6.605,4.584)--(6.612,4.584)--(6.619,4.584)--(6.626,4.581)--(6.634,4.579)%
  --(6.641,4.576)--(6.648,4.571)--(6.655,4.569)--(6.662,4.564)--(6.670,4.559)--(6.677,4.554)%
  --(6.684,4.546)--(6.691,4.539)--(6.698,4.531)--(6.706,4.524)--(6.713,4.514)--(6.720,4.504)%
  --(6.727,4.494)--(6.734,4.484)--(6.742,4.472)--(6.749,4.459)--(6.756,4.447)--(6.763,4.432)%
  --(6.770,4.417)--(6.778,4.402)--(6.785,4.387)--(6.792,4.369)--(6.799,4.352)--(6.806,4.332)%
  --(6.814,4.314)--(6.821,4.295)--(6.828,4.272)--(6.835,4.252)--(6.842,4.227)--(6.850,4.205)%
  --(6.857,4.182)--(6.864,4.155)--(6.871,4.130)--(6.878,4.105)--(6.885,4.078)--(6.893,4.048)%
  --(6.900,4.020)--(6.907,3.990)--(6.914,3.960)--(6.921,3.930)--(6.929,3.898)--(6.936,3.866)%
  --(6.943,3.833)--(6.950,3.801)--(6.957,3.766)--(6.965,3.731)--(6.972,3.696)--(6.979,3.661)%
  --(6.986,3.624)--(6.993,3.586)--(7.001,3.549)--(7.008,3.509)--(7.015,3.467)--(7.022,3.422)%
  --(7.029,3.377)--(7.037,3.324)--(7.044,3.267)--(7.051,3.207)--(7.058,3.140)--(7.065,3.067)%
  --(7.073,2.978)--(7.087,2.594)--(7.094,2.549)--(7.101,2.516)--(7.109,2.489)--(7.116,2.466)%
  --(7.123,2.446)--(7.130,2.429)--(7.137,2.412)--(7.145,2.394)--(7.152,2.377)--(7.159,2.359)%
  --(7.166,2.342)--(7.173,2.324)--(7.181,2.307)--(7.188,2.284)--(7.195,2.182)--(7.202,2.167)%
  --(7.209,2.150)--(7.217,2.132)--(7.224,2.115)--(7.231,2.095)--(7.238,2.072)--(7.245,2.052)%
  --(7.253,2.035)--(7.260,2.018)--(7.267,2.003)--(7.274,1.988)--(7.281,1.973)--(7.289,1.958)%
  --(7.296,1.945)--(7.303,1.930)--(7.310,1.918)--(7.317,1.903)--(7.325,1.890)--(7.332,1.878)%
  --(7.339,1.863)--(7.346,1.850)--(7.353,1.838)--(7.361,1.825)--(7.368,1.813)--(7.375,1.798)%
  --(7.382,1.786)--(7.389,1.773)--(7.397,1.761)--(7.404,1.748)--(7.411,1.736)--(7.418,1.723)%
  --(7.425,1.711)--(7.433,1.698)--(7.440,1.686)--(7.447,1.671);
\gpsetdashtype{gp dt solid}
\draw[gp path] (1.688,5.973)--(1.688,0.985)--(7.447,0.985)--(7.447,5.973)--cycle;
%% coordinates of the plot area
\gpdefrectangularnode{gp plot 3}{\pgfpoint{1.688cm}{0.985cm}}{\pgfpoint{7.447cm}{5.973cm}}
\node[gp node center] at (11.953,20.229) {Legenda};
\node[gp node right] at (12.047,17.426) {66\textsubscript{3}-418};
\draw[gp path] (12.231,17.426)--(13.147,17.426);
\node[gp node right] at (12.047,17.118) {64\textsubscript{2}-415};
\gpsetdashtype{gp dt 2}
\draw[gp path] (12.231,17.118)--(13.147,17.118);
\node[gp node right] at (12.047,16.810) {63\textsubscript{1}-412};
\gpsetdashtype{gp dt 3}
\draw[gp path] (12.231,16.810)--(13.147,16.810);
\node[gp node right] at (12.047,16.502) {65\textsubscript{1}-412};
\gpsetdashtype{gp dt 4}
\draw[gp path] (12.231,16.502)--(13.147,16.502);
%% coordinates of the plot area
\gpdefrectangularnode{gp plot 4}{\pgfpoint{8.460cm}{14.162cm}}{\pgfpoint{15.447cm}{19.767cm}}
\gpcolor{color=gp lt color axes}
\gpsetlinetype{gp lt axes}
\gpsetdashtype{gp dt axes}
\gpsetlinewidth{0.50}
\draw[gp path] (9.504,7.882)--(15.447,7.882);
\gpcolor{color=gp lt color border}
\gpsetlinetype{gp lt border}
\gpsetdashtype{gp dt solid}
\gpsetlinewidth{1.00}
\draw[gp path] (9.504,7.882)--(9.684,7.882);
\draw[gp path] (15.447,7.882)--(15.267,7.882);
\node[gp node right] at (9.320,7.882) {$0$};
\gpcolor{color=gp lt color axes}
\gpsetlinetype{gp lt axes}
\gpsetdashtype{gp dt axes}
\gpsetlinewidth{0.50}
\draw[gp path] (9.504,8.880)--(15.447,8.880);
\gpcolor{color=gp lt color border}
\gpsetlinetype{gp lt border}
\gpsetdashtype{gp dt solid}
\gpsetlinewidth{1.00}
\draw[gp path] (9.504,8.880)--(9.684,8.880);
\draw[gp path] (15.447,8.880)--(15.267,8.880);
\node[gp node right] at (9.320,8.880) {$200$};
\gpcolor{color=gp lt color axes}
\gpsetlinetype{gp lt axes}
\gpsetdashtype{gp dt axes}
\gpsetlinewidth{0.50}
\draw[gp path] (9.504,9.877)--(15.447,9.877);
\gpcolor{color=gp lt color border}
\gpsetlinetype{gp lt border}
\gpsetdashtype{gp dt solid}
\gpsetlinewidth{1.00}
\draw[gp path] (9.504,9.877)--(9.684,9.877);
\draw[gp path] (15.447,9.877)--(15.267,9.877);
\node[gp node right] at (9.320,9.877) {$400$};
\gpcolor{color=gp lt color axes}
\gpsetlinetype{gp lt axes}
\gpsetdashtype{gp dt axes}
\gpsetlinewidth{0.50}
\draw[gp path] (9.504,10.875)--(15.447,10.875);
\gpcolor{color=gp lt color border}
\gpsetlinetype{gp lt border}
\gpsetdashtype{gp dt solid}
\gpsetlinewidth{1.00}
\draw[gp path] (9.504,10.875)--(9.684,10.875);
\draw[gp path] (15.447,10.875)--(15.267,10.875);
\node[gp node right] at (9.320,10.875) {$600$};
\gpcolor{color=gp lt color axes}
\gpsetlinetype{gp lt axes}
\gpsetdashtype{gp dt axes}
\gpsetlinewidth{0.50}
\draw[gp path] (9.504,11.872)--(15.447,11.872);
\gpcolor{color=gp lt color border}
\gpsetlinetype{gp lt border}
\gpsetdashtype{gp dt solid}
\gpsetlinewidth{1.00}
\draw[gp path] (9.504,11.872)--(9.684,11.872);
\draw[gp path] (15.447,11.872)--(15.267,11.872);
\node[gp node right] at (9.320,11.872) {$800$};
\gpcolor{color=gp lt color axes}
\gpsetlinetype{gp lt axes}
\gpsetdashtype{gp dt axes}
\gpsetlinewidth{0.50}
\draw[gp path] (9.504,12.870)--(15.447,12.870);
\gpcolor{color=gp lt color border}
\gpsetlinetype{gp lt border}
\gpsetdashtype{gp dt solid}
\gpsetlinewidth{1.00}
\draw[gp path] (9.504,12.870)--(9.684,12.870);
\draw[gp path] (15.447,12.870)--(15.267,12.870);
\node[gp node right] at (9.320,12.870) {$1000$};
\draw[gp path] (9.695,7.882)--(9.695,7.972);
\draw[gp path] (9.695,12.870)--(9.695,12.780);
\gpcolor{color=gp lt color axes}
\gpsetlinetype{gp lt axes}
\gpsetdashtype{gp dt axes}
\gpsetlinewidth{0.50}
\draw[gp path] (9.909,7.882)--(9.909,12.870);
\gpcolor{color=gp lt color border}
\gpsetlinetype{gp lt border}
\gpsetdashtype{gp dt solid}
\gpsetlinewidth{1.00}
\draw[gp path] (9.909,7.882)--(9.909,8.062);
\draw[gp path] (9.909,12.870)--(9.909,12.690);
\node[gp node center] at (9.909,7.574) {$-1$};
\draw[gp path] (10.123,7.882)--(10.123,7.972);
\draw[gp path] (10.123,12.870)--(10.123,12.780);
\draw[gp path] (10.337,7.882)--(10.337,7.972);
\draw[gp path] (10.337,12.870)--(10.337,12.780);
\draw[gp path] (10.551,7.882)--(10.551,7.972);
\draw[gp path] (10.551,12.870)--(10.551,12.780);
\draw[gp path] (10.765,7.882)--(10.765,7.972);
\draw[gp path] (10.765,12.870)--(10.765,12.780);
\gpcolor{color=gp lt color axes}
\gpsetlinetype{gp lt axes}
\gpsetdashtype{gp dt axes}
\gpsetlinewidth{0.50}
\draw[gp path] (10.979,7.882)--(10.979,12.870);
\gpcolor{color=gp lt color border}
\gpsetlinetype{gp lt border}
\gpsetdashtype{gp dt solid}
\gpsetlinewidth{1.00}
\draw[gp path] (10.979,7.882)--(10.979,8.062);
\draw[gp path] (10.979,12.870)--(10.979,12.690);
\node[gp node center] at (10.979,7.574) {$-0.5$};
\draw[gp path] (11.192,7.882)--(11.192,7.972);
\draw[gp path] (11.192,12.870)--(11.192,12.780);
\draw[gp path] (11.406,7.882)--(11.406,7.972);
\draw[gp path] (11.406,12.870)--(11.406,12.780);
\draw[gp path] (11.620,7.882)--(11.620,7.972);
\draw[gp path] (11.620,12.870)--(11.620,12.780);
\draw[gp path] (11.834,7.882)--(11.834,7.972);
\draw[gp path] (11.834,12.870)--(11.834,12.780);
\gpcolor{color=gp lt color axes}
\gpsetlinetype{gp lt axes}
\gpsetdashtype{gp dt axes}
\gpsetlinewidth{0.50}
\draw[gp path] (12.048,7.882)--(12.048,12.870);
\gpcolor{color=gp lt color border}
\gpsetlinetype{gp lt border}
\gpsetdashtype{gp dt solid}
\gpsetlinewidth{1.00}
\draw[gp path] (12.048,7.882)--(12.048,8.062);
\draw[gp path] (12.048,12.870)--(12.048,12.690);
\node[gp node center] at (12.048,7.574) {$0$};
\draw[gp path] (12.262,7.882)--(12.262,7.972);
\draw[gp path] (12.262,12.870)--(12.262,12.780);
\draw[gp path] (12.476,7.882)--(12.476,7.972);
\draw[gp path] (12.476,12.870)--(12.476,12.780);
\draw[gp path] (12.690,7.882)--(12.690,7.972);
\draw[gp path] (12.690,12.870)--(12.690,12.780);
\draw[gp path] (12.904,7.882)--(12.904,7.972);
\draw[gp path] (12.904,12.870)--(12.904,12.780);
\gpcolor{color=gp lt color axes}
\gpsetlinetype{gp lt axes}
\gpsetdashtype{gp dt axes}
\gpsetlinewidth{0.50}
\draw[gp path] (13.118,7.882)--(13.118,12.870);
\gpcolor{color=gp lt color border}
\gpsetlinetype{gp lt border}
\gpsetdashtype{gp dt solid}
\gpsetlinewidth{1.00}
\draw[gp path] (13.118,7.882)--(13.118,8.062);
\draw[gp path] (13.118,12.870)--(13.118,12.690);
\node[gp node center] at (13.118,7.574) {$0.5$};
\draw[gp path] (13.332,7.882)--(13.332,7.972);
\draw[gp path] (13.332,12.870)--(13.332,12.780);
\draw[gp path] (13.546,7.882)--(13.546,7.972);
\draw[gp path] (13.546,12.870)--(13.546,12.780);
\draw[gp path] (13.760,7.882)--(13.760,7.972);
\draw[gp path] (13.760,12.870)--(13.760,12.780);
\draw[gp path] (13.973,7.882)--(13.973,7.972);
\draw[gp path] (13.973,12.870)--(13.973,12.780);
\gpcolor{color=gp lt color axes}
\gpsetlinetype{gp lt axes}
\gpsetdashtype{gp dt axes}
\gpsetlinewidth{0.50}
\draw[gp path] (14.187,7.882)--(14.187,12.870);
\gpcolor{color=gp lt color border}
\gpsetlinetype{gp lt border}
\gpsetdashtype{gp dt solid}
\gpsetlinewidth{1.00}
\draw[gp path] (14.187,7.882)--(14.187,8.062);
\draw[gp path] (14.187,12.870)--(14.187,12.690);
\node[gp node center] at (14.187,7.574) {$1$};
\draw[gp path] (14.401,7.882)--(14.401,7.972);
\draw[gp path] (14.401,12.870)--(14.401,12.780);
\draw[gp path] (14.615,7.882)--(14.615,7.972);
\draw[gp path] (14.615,12.870)--(14.615,12.780);
\draw[gp path] (14.829,7.882)--(14.829,7.972);
\draw[gp path] (14.829,12.870)--(14.829,12.780);
\draw[gp path] (15.043,7.882)--(15.043,7.972);
\draw[gp path] (15.043,12.870)--(15.043,12.780);
\gpcolor{color=gp lt color axes}
\gpsetlinetype{gp lt axes}
\gpsetdashtype{gp dt axes}
\gpsetlinewidth{0.50}
\draw[gp path] (15.257,7.882)--(15.257,12.870);
\gpcolor{color=gp lt color border}
\gpsetlinetype{gp lt border}
\gpsetdashtype{gp dt solid}
\gpsetlinewidth{1.00}
\draw[gp path] (15.257,7.882)--(15.257,8.062);
\draw[gp path] (15.257,12.870)--(15.257,12.690);
\node[gp node center] at (15.257,7.574) {$1.5$};
\draw[gp path] (9.504,12.870)--(9.504,7.882)--(15.447,7.882)--(15.447,12.870)--cycle;
\node[gp node center,rotate=-270] at (8.246,10.376) {$C_d \cdot 10^4$};
\node[gp node center] at (12.475,7.112) {$C_l$};
\node[gp node center] at (12.475,13.332) {Polar de Arrasto};
\draw[gp path] (10.519,12.870)--(10.514,12.842)--(10.499,12.752)--(10.482,12.657)--(10.467,12.570)%
  --(10.452,12.481)--(10.438,12.395)--(10.424,12.314)--(10.410,12.230)--(10.398,12.152)--(10.386,12.072)%
  --(10.373,11.995)--(10.362,11.921)--(10.350,11.845)--(10.329,11.703)--(10.319,11.633)--(10.310,11.567)%
  --(10.300,11.499)--(10.291,11.436)--(10.282,11.372)--(10.273,11.311)--(10.265,11.250)--(10.257,11.190)%
  --(10.249,11.133)--(10.242,11.075)--(10.235,11.020)--(10.227,10.966)--(10.221,10.914)--(10.214,10.862)%
  --(10.202,10.764)--(10.196,10.716)--(10.191,10.670)--(10.186,10.623)--(10.181,10.580)--(10.176,10.535)%
  --(10.171,10.494)--(10.167,10.452)--(10.163,10.411)--(10.159,10.373)--(10.155,10.334)--(10.152,10.297)%
  --(10.149,10.261)--(10.145,10.225)--(10.142,10.191)--(10.139,10.158)--(10.134,10.092)--(10.132,10.061)%
  --(10.129,10.029)--(10.128,9.999)--(10.126,9.970)--(10.124,9.941)--(10.122,9.912)--(10.122,9.885)%
  --(10.121,9.859)--(10.119,9.832)--(10.118,9.806)--(10.118,9.782)--(10.117,9.758)--(10.116,9.734)%
  --(10.115,9.710)--(10.115,9.688)--(10.116,9.666)--(10.116,9.644)--(10.116,9.600)--(10.116,9.579)%
  --(10.118,9.560)--(10.119,9.541)--(10.121,9.522)--(10.122,9.504)--(10.123,9.485)--(10.125,9.467)%
  --(10.127,9.451)--(10.130,9.435)--(10.133,9.419)--(10.135,9.404)--(10.137,9.387)--(10.139,9.371)%
  --(10.141,9.355)--(10.144,9.340)--(10.147,9.326)--(10.149,9.311)--(10.152,9.296)--(10.154,9.280)%
  --(10.158,9.250)--(10.160,9.235)--(10.164,9.221)--(10.167,9.208)--(10.170,9.194)--(10.174,9.181)%
  --(10.177,9.168)--(10.180,9.155)--(10.183,9.142)--(10.186,9.129)--(10.190,9.117)--(10.194,9.106)%
  --(10.198,9.094)--(10.202,9.083)--(10.206,9.072)--(10.210,9.060)--(10.214,9.049)--(10.218,9.038)%
  --(10.222,9.027)--(10.226,9.017)--(10.231,9.007)--(10.236,8.997)--(10.246,8.978)--(10.251,8.969)%
  --(10.257,8.960)--(10.262,8.950)--(10.267,8.941)--(10.272,8.931)--(10.277,8.922)--(10.282,8.914)%
  --(10.288,8.906)--(10.294,8.897)--(10.301,8.890)--(10.307,8.882)--(10.313,8.874)--(10.319,8.866)%
  --(10.325,8.859)--(10.331,8.851)--(10.337,8.844)--(10.343,8.837)--(10.349,8.829)--(10.355,8.822)%
  --(10.362,8.816)--(10.369,8.809)--(10.376,8.803)--(10.382,8.797)--(10.389,8.790)--(10.396,8.784)%
  --(10.403,8.778)--(10.417,8.766)--(10.425,8.760)--(10.432,8.754)--(10.439,8.747)--(10.446,8.741)%
  --(10.453,8.735)--(10.460,8.730)--(10.468,8.724)--(10.473,8.716)--(10.477,8.707)--(10.481,8.698)%
  --(10.486,8.690)--(10.491,8.683)--(10.497,8.676)--(10.503,8.669)--(10.509,8.662)--(10.515,8.656)%
  --(10.521,8.649)--(10.528,8.644)--(10.535,8.638)--(10.542,8.633)--(10.550,8.628)--(10.557,8.623)%
  --(10.565,8.618)--(10.573,8.614)--(10.581,8.609)--(10.589,8.605)--(10.597,8.601)--(10.604,8.596)%
  --(10.612,8.592)--(10.628,8.584)--(10.636,8.579)--(10.644,8.575)--(10.652,8.570)--(10.660,8.566)%
  --(10.669,8.562)--(10.678,8.558)--(10.686,8.554)--(10.694,8.550)--(10.703,8.546)--(10.711,8.542)%
  --(10.720,8.538)--(10.728,8.534)--(10.736,8.530)--(10.745,8.526)--(10.753,8.522)--(10.761,8.518)%
  --(10.769,8.515)--(10.777,8.511)--(10.786,8.507)--(10.794,8.504)--(10.803,8.501)--(10.811,8.497)%
  --(10.820,8.494)--(10.829,8.490)--(10.837,8.487)--(10.846,8.484)--(10.854,8.480)--(10.863,8.477)%
  --(10.871,8.474)--(10.880,8.470)--(10.888,8.467)--(10.896,8.464)--(10.905,8.461)--(10.913,8.458)%
  --(10.922,8.455)--(10.931,8.452)--(10.939,8.449)--(10.957,8.444)--(10.966,8.441)--(10.975,8.438)%
  --(10.984,8.435)--(10.993,8.432)--(11.002,8.429)--(11.011,8.426)--(11.020,8.423)--(11.029,8.420)%
  --(11.038,8.418)--(11.046,8.414)--(11.054,8.410)--(11.062,8.406)--(11.069,8.402)--(11.078,8.399)%
  --(11.086,8.396)--(11.094,8.392)--(11.102,8.389)--(11.111,8.386)--(11.119,8.383)--(11.127,8.380)%
  --(11.136,8.378)--(11.145,8.375)--(11.154,8.372)--(11.163,8.370)--(11.172,8.367)--(11.180,8.365)%
  --(11.189,8.362)--(11.198,8.360)--(11.206,8.357)--(11.215,8.355)--(11.224,8.352)--(11.232,8.350)%
  --(11.241,8.348)--(11.250,8.346)--(11.259,8.343)--(11.268,8.341)--(11.276,8.339)--(11.285,8.337)%
  --(11.293,8.335)--(11.302,8.333)--(11.310,8.330)--(11.319,8.328)--(11.328,8.326)--(11.336,8.324)%
  --(11.345,8.322)--(11.353,8.320)--(11.361,8.318)--(11.369,8.316)--(11.386,8.313)--(11.395,8.311)%
  --(11.402,8.308)--(11.409,8.305)--(11.415,8.303)--(11.422,8.300)--(11.428,8.298)--(11.434,8.296)%
  --(11.438,8.295)--(11.443,8.293)--(11.449,8.292)--(11.455,8.291)--(11.463,8.290)--(11.472,8.288)%
  --(11.482,8.287)--(11.492,8.285)--(11.502,8.284)--(11.512,8.282)--(11.522,8.281)--(11.532,8.279)%
  --(11.543,8.278)--(11.553,8.277)--(11.564,8.275)--(11.574,8.274)--(11.585,8.273)--(11.596,8.272)%
  --(11.607,8.271)--(11.617,8.269)--(11.627,8.267)--(11.638,8.266)--(11.649,8.264)--(11.659,8.263)%
  --(11.670,8.262)--(11.681,8.260)--(11.692,8.259)--(11.703,8.258)--(11.714,8.257)--(11.726,8.255)%
  --(11.737,8.254)--(11.748,8.253)--(11.760,8.252)--(11.771,8.251)--(11.782,8.249)--(11.793,8.248)%
  --(11.804,8.247)--(11.815,8.245)--(11.827,8.244)--(11.838,8.243)--(11.850,8.242)--(11.862,8.241)%
  --(11.873,8.240)--(11.884,8.239)--(11.895,8.237)--(11.907,8.236)--(11.918,8.235)--(11.930,8.234)%
  --(11.941,8.233)--(11.953,8.231)--(11.964,8.230)--(11.975,8.228)--(11.986,8.226)--(11.997,8.225)%
  --(12.008,8.223)--(12.019,8.221)--(12.029,8.219)--(12.040,8.217)--(12.051,8.214)--(12.062,8.212)%
  --(12.072,8.210)--(12.083,8.208)--(12.093,8.204)--(12.103,8.201)--(12.113,8.198)--(12.123,8.195)%
  --(12.133,8.191)--(12.144,8.188)--(12.154,8.184)--(12.165,8.181)--(12.175,8.177)--(12.185,8.173)%
  --(12.206,8.165)--(12.218,8.162)--(12.229,8.159)--(12.240,8.155)--(12.251,8.151)--(12.262,8.148)%
  --(12.274,8.145)--(12.285,8.141)--(12.296,8.137)--(12.319,8.130)--(12.331,8.126)--(12.342,8.122)%
  --(12.353,8.118)--(12.365,8.113)--(12.376,8.109)--(12.387,8.103)--(12.398,8.096)--(12.408,8.088)%
  --(12.417,8.077)--(12.430,8.073)--(12.443,8.072)--(12.456,8.072)--(12.469,8.071)--(12.483,8.071)%
  --(12.497,8.071)--(12.510,8.071)--(12.524,8.070)--(12.537,8.070)--(12.550,8.070)--(12.564,8.069)%
  --(12.578,8.069)--(12.591,8.069)--(12.605,8.069)--(12.619,8.070)--(12.632,8.070)--(12.646,8.070)%
  --(12.659,8.070)--(12.673,8.070)--(12.687,8.071)--(12.700,8.071)--(12.713,8.071)--(12.727,8.071)%
  --(12.740,8.071)--(12.753,8.071)--(12.766,8.071)--(12.780,8.071)--(12.793,8.071)--(12.806,8.071)%
  --(12.819,8.071)--(12.833,8.071)--(12.846,8.072)--(12.859,8.072)--(12.872,8.072)--(12.885,8.073)%
  --(12.899,8.073)--(12.912,8.074)--(12.925,8.074)--(12.939,8.074)--(12.953,8.074)--(12.967,8.074)%
  --(12.980,8.074)--(12.994,8.074)--(13.008,8.074)--(13.021,8.074)--(13.035,8.074)--(13.049,8.075)%
  --(13.062,8.075)--(13.076,8.075)--(13.090,8.075)--(13.103,8.075)--(13.117,8.075)--(13.130,8.075)%
  --(13.143,8.074)--(13.157,8.074)--(13.170,8.074)--(13.183,8.075)--(13.197,8.075)--(13.210,8.075)%
  --(13.223,8.075)--(13.236,8.075)--(13.249,8.076)--(13.262,8.076)--(13.275,8.076)--(13.288,8.077)%
  --(13.301,8.077)--(13.314,8.078)--(13.327,8.078)--(13.340,8.078)--(13.354,8.078)--(13.367,8.078)%
  --(13.381,8.078)--(13.394,8.078)--(13.408,8.078)--(13.422,8.078)--(13.435,8.078)--(13.449,8.078)%
  --(13.462,8.078)--(13.475,8.078)--(13.489,8.079)--(13.502,8.079)--(13.515,8.079)--(13.527,8.080)%
  --(13.540,8.080)--(13.553,8.081)--(13.565,8.082)--(13.577,8.082)--(13.591,8.083)--(13.604,8.083)%
  --(13.617,8.083)--(13.629,8.084)--(13.639,8.086)--(13.650,8.088)--(13.657,8.091)--(13.660,8.097)%
  --(13.658,8.106)--(13.656,8.115)--(13.656,8.123)--(13.656,8.130)--(13.657,8.137)--(13.658,8.142)%
  --(13.659,8.148)--(13.659,8.154)--(13.662,8.159)--(13.662,8.163)--(13.660,8.167)--(13.657,8.170)%
  --(13.655,8.175)--(13.652,8.180)--(13.649,8.185)--(13.641,8.198)--(13.639,8.203)--(13.635,8.210)%
  --(13.633,8.216)--(13.630,8.223)--(13.627,8.230)--(13.625,8.236)--(13.621,8.243)--(13.618,8.257)%
  --(13.617,8.270)--(13.618,8.276)--(13.618,8.283)--(13.619,8.290)--(13.620,8.296)--(13.620,8.303)%
  --(13.622,8.309)--(13.622,8.317)--(13.625,8.322)--(13.625,8.330)--(13.627,8.337)--(13.628,8.344)%
  --(13.630,8.351)--(13.634,8.357)--(13.635,8.365)--(13.639,8.370)--(13.641,8.377)--(13.643,8.385)%
  --(13.648,8.390)--(13.650,8.398)--(13.654,8.404)--(13.657,8.412)--(13.658,8.420)--(13.662,8.427)%
  --(13.663,8.436)--(13.668,8.442)--(13.669,8.451)--(13.674,8.457)--(13.676,8.465)--(13.679,8.472)%
  --(13.682,8.480)--(13.685,8.488)--(13.690,8.495)--(13.693,8.503)--(13.699,8.508)--(13.704,8.515)%
  --(13.709,8.521)--(13.716,8.527)--(13.719,8.535)--(13.725,8.541)--(13.729,8.548)--(13.734,8.555)%
  --(13.740,8.562)--(13.745,8.569)--(13.752,8.574)--(13.757,8.581)--(13.765,8.586)--(13.772,8.591)%
  --(13.779,8.596)--(13.787,8.601)--(13.795,8.606)--(13.802,8.611)--(13.810,8.616)--(13.817,8.622)%
  --(13.824,8.627)--(13.832,8.632)--(13.841,8.636)--(13.849,8.641)--(13.857,8.645)--(13.865,8.650)%
  --(13.873,8.655)--(13.881,8.660)--(13.890,8.665)--(13.898,8.669)--(13.907,8.673)--(13.916,8.677)%
  --(13.925,8.681)--(13.933,8.686)--(13.941,8.691)--(13.948,8.696)--(13.956,8.702)--(13.963,8.707)%
  --(13.971,8.713)--(13.979,8.717)--(13.987,8.721)--(13.996,8.726)--(14.004,8.731)--(14.013,8.736)%
  --(14.021,8.741)--(14.029,8.746)--(14.037,8.751)--(14.045,8.756)--(14.053,8.761)--(14.062,8.766)%
  --(14.070,8.771)--(14.079,8.776)--(14.087,8.781)--(14.094,8.787)--(14.102,8.793)--(14.109,8.799)%
  --(14.117,8.805)--(14.124,8.811)--(14.132,8.817)--(14.140,8.823)--(14.148,8.828)--(14.156,8.833)%
  --(14.165,8.839)--(14.173,8.844)--(14.181,8.850)--(14.190,8.855)--(14.198,8.861)--(14.206,8.866)%
  --(14.214,8.872)--(14.222,8.877)--(14.230,8.883)--(14.238,8.890)--(14.245,8.896)--(14.253,8.903)%
  --(14.260,8.910)--(14.267,8.917)--(14.274,8.924)--(14.281,8.931)--(14.288,8.938)--(14.297,8.944)%
  --(14.305,8.949)--(14.313,8.955)--(14.321,8.961)--(14.329,8.967)--(14.337,8.973)--(14.346,8.979)%
  --(14.353,8.985)--(14.362,8.991)--(14.370,8.997)--(14.378,9.003)--(14.386,9.010)--(14.393,9.017)%
  --(14.400,9.024)--(14.407,9.032)--(14.414,9.039)--(14.422,9.047)--(14.429,9.054)--(14.437,9.061)%
  --(14.444,9.068)--(14.452,9.076)--(14.459,9.083)--(14.467,9.090)--(14.475,9.097)--(14.482,9.104)%
  --(14.490,9.111)--(14.497,9.119)--(14.505,9.126)--(14.513,9.133)--(14.520,9.141)--(14.527,9.149)%
  --(14.534,9.157)--(14.541,9.166)--(14.547,9.175)--(14.567,9.202)--(14.574,9.210)--(14.580,9.219)%
  --(14.587,9.228)--(14.594,9.237)--(14.600,9.246)--(14.607,9.255)--(14.613,9.265)--(14.619,9.275)%
  --(14.625,9.285)--(14.630,9.296)--(14.637,9.305)--(14.644,9.315)--(14.650,9.325)--(14.656,9.335)%
  --(14.663,9.344)--(14.669,9.354)--(14.676,9.363)--(14.683,9.373)--(14.689,9.382)--(14.696,9.392)%
  --(14.703,9.401)--(14.710,9.410)--(14.717,9.420)--(14.724,9.430)--(14.731,9.439)--(14.737,9.449)%
  --(14.744,9.459)--(14.750,9.470)--(14.756,9.481)--(14.767,9.505)--(14.773,9.517)--(14.778,9.528)%
  --(14.784,9.539)--(14.790,9.550)--(14.796,9.561)--(14.803,9.572)--(14.809,9.584)--(14.815,9.595)%
  --(14.821,9.607)--(14.827,9.618)--(14.833,9.630)--(14.838,9.643)--(14.843,9.656)--(14.848,9.671)%
  --(14.853,9.685)--(14.857,9.699)--(14.863,9.712)--(14.868,9.726)--(14.873,9.739)--(14.878,9.753)%
  --(14.884,9.765)--(14.889,9.779)--(14.894,9.792)--(14.900,9.806)--(14.905,9.820)--(14.913,9.852)%
  --(14.916,9.869)--(14.921,9.885)--(14.925,9.901)--(14.929,9.917)--(14.933,9.932)--(14.938,9.948)%
  --(14.943,9.963)--(14.947,9.979)--(14.952,9.994)--(14.956,10.011)--(14.959,10.029)--(14.962,10.048)%
  --(14.965,10.067)--(14.969,10.086)--(14.972,10.104)--(14.976,10.123)--(14.979,10.141)--(14.983,10.159)%
  --(14.987,10.176)--(14.991,10.195)--(14.994,10.214)--(14.997,10.235)--(15.001,10.279)--(15.004,10.300)%
  --(15.006,10.320)--(15.009,10.341)--(15.012,10.362)--(15.015,10.382)--(15.018,10.403)--(15.020,10.426)%
  --(15.022,10.451)--(15.023,10.475)--(15.025,10.499)--(15.027,10.523)--(15.029,10.546)--(15.031,10.570)%
  --(15.033,10.593)--(15.035,10.618)--(15.036,10.646)--(15.036,10.673)--(15.037,10.700)--(15.039,10.726)%
  --(15.040,10.752)--(15.043,10.805)--(15.043,10.836)--(15.043,10.866)--(15.044,10.895)--(15.044,10.923)%
  --(15.045,10.952)--(15.046,10.982)--(15.045,11.014)--(15.044,11.047)--(15.044,11.079)--(15.045,11.110)%
  --(15.045,11.141)--(15.045,11.175)--(15.043,11.210)--(15.042,11.246)--(15.042,11.280)--(15.042,11.313)%
  --(15.041,11.348)--(15.039,11.386)--(15.038,11.424)--(15.036,11.497)--(15.035,11.535)--(15.032,11.576)%
  --(15.030,11.617)--(15.028,11.656)--(15.027,11.695)--(15.025,11.736)--(15.022,11.781)--(15.019,11.824)%
  --(15.017,11.865)--(15.015,11.907)--(15.012,11.953)--(15.008,12.000)--(15.005,12.044)--(15.003,12.088)%
  --(15.000,12.136)--(14.996,12.186)--(14.992,12.234)--(14.981,12.383)--(14.977,12.433)--(14.974,12.481)%
  --(14.969,12.534)--(14.964,12.590)--(14.960,12.642)--(14.956,12.693)--(14.951,12.750)--(14.946,12.807)%
  --(14.941,12.861)--(14.940,12.870);
\gpsetdashtype{gp dt 2}
\draw[gp path] (9.947,12.870)--(9.925,12.754)--(9.904,12.638)--(9.885,12.528)--(9.867,12.424)%
  --(9.850,12.322)--(9.834,12.223)--(9.818,12.127)--(9.804,12.036)--(9.791,11.948)--(9.778,11.862)%
  --(9.766,11.778)--(9.754,11.698)--(9.743,11.622)--(9.732,11.544)--(9.722,11.471)--(9.713,11.401)%
  --(9.703,11.329)--(9.694,11.262)--(9.685,11.194)--(9.677,11.130)--(9.668,11.067)--(9.660,11.005)%
  --(9.652,10.945)--(9.645,10.887)--(9.637,10.829)--(9.631,10.775)--(9.624,10.718)--(9.618,10.666)%
  --(9.611,10.613)--(9.605,10.562)--(9.599,10.512)--(9.593,10.462)--(9.587,10.415)--(9.582,10.368)%
  --(9.576,10.320)--(9.572,10.276)--(9.567,10.231)--(9.562,10.187)--(9.558,10.146)--(9.553,10.104)%
  --(9.549,10.063)--(9.545,10.024)--(9.541,9.984)--(9.537,9.947)--(9.534,9.910)--(9.530,9.873)%
  --(9.527,9.837)--(9.524,9.803)--(9.521,9.767)--(9.519,9.735)--(9.516,9.702)--(9.514,9.669)%
  --(9.512,9.638)--(9.511,9.607)--(9.509,9.576)--(9.508,9.547)--(9.507,9.519)--(9.505,9.490)%
  --(9.505,9.463)--(9.504,9.436)--(9.504,9.409)--(9.504,9.383)--(9.504,9.358)--(9.504,9.333)%
  --(9.505,9.310)--(9.506,9.287)--(9.507,9.264)--(9.508,9.242)--(9.510,9.221)--(9.512,9.200)%
  --(9.514,9.180)--(9.517,9.161)--(9.519,9.142)--(9.528,9.121)--(9.539,9.099)--(9.537,9.051)%
  --(9.550,9.031)--(9.567,9.011)--(9.585,8.989)--(9.604,8.969)--(9.620,8.938)--(9.640,8.906)%
  --(9.661,8.880)--(9.685,8.856)--(9.709,8.836)--(9.734,8.817)--(9.757,8.800)--(9.781,8.785)%
  --(9.807,8.770)--(9.830,8.757)--(9.839,8.749)--(9.840,8.742)--(9.842,8.735)--(9.844,8.728)%
  --(9.848,8.720)--(9.851,8.713)--(9.854,8.706)--(9.858,8.698)--(9.862,8.691)--(9.865,8.684)%
  --(9.869,8.677)--(9.874,8.669)--(9.878,8.662)--(9.881,8.655)--(9.885,8.649)--(9.889,8.642)%
  --(9.892,8.635)--(9.896,8.628)--(9.898,8.622)--(9.901,8.616)--(9.903,8.609)--(9.904,8.603)%
  --(9.905,8.598)--(9.904,8.593)--(9.907,8.589)--(9.913,8.585)--(9.920,8.579)--(9.928,8.574)%
  --(9.936,8.569)--(9.944,8.565)--(9.952,8.560)--(9.960,8.555)--(9.969,8.551)--(9.977,8.547)%
  --(9.987,8.542)--(9.996,8.538)--(10.006,8.533)--(10.015,8.529)--(10.025,8.525)--(10.035,8.521)%
  --(10.045,8.517)--(10.054,8.514)--(10.075,8.506)--(10.086,8.503)--(10.094,8.495)--(10.102,8.489)%
  --(10.111,8.483)--(10.121,8.478)--(10.131,8.473)--(10.141,8.469)--(10.152,8.465)--(10.174,8.457)%
  --(10.185,8.454)--(10.196,8.450)--(10.207,8.446)--(10.219,8.443)--(10.230,8.440)--(10.242,8.436)%
  --(10.253,8.433)--(10.264,8.430)--(10.276,8.427)--(10.288,8.424)--(10.299,8.421)--(10.311,8.418)%
  --(10.323,8.415)--(10.335,8.412)--(10.347,8.409)--(10.359,8.406)--(10.371,8.403)--(10.383,8.400)%
  --(10.395,8.398)--(10.407,8.395)--(10.419,8.392)--(10.443,8.387)--(10.455,8.385)--(10.468,8.382)%
  --(10.480,8.380)--(10.492,8.377)--(10.505,8.375)--(10.517,8.373)--(10.529,8.370)--(10.542,8.368)%
  --(10.554,8.366)--(10.577,8.358)--(10.589,8.355)--(10.602,8.351)--(10.614,8.349)--(10.626,8.346)%
  --(10.639,8.343)--(10.651,8.341)--(10.664,8.339)--(10.676,8.337)--(10.689,8.334)--(10.701,8.332)%
  --(10.727,8.328)--(10.739,8.326)--(10.752,8.324)--(10.765,8.322)--(10.778,8.320)--(10.791,8.318)%
  --(10.804,8.316)--(10.816,8.314)--(10.829,8.312)--(10.842,8.311)--(10.855,8.309)--(10.868,8.307)%
  --(10.893,8.304)--(10.906,8.302)--(10.919,8.301)--(10.932,8.299)--(10.945,8.298)--(10.958,8.295)%
  --(10.971,8.293)--(10.983,8.291)--(10.996,8.289)--(11.009,8.287)--(11.022,8.285)--(11.035,8.283)%
  --(11.061,8.280)--(11.074,8.278)--(11.087,8.277)--(11.100,8.275)--(11.113,8.274)--(11.126,8.272)%
  --(11.139,8.271)--(11.152,8.269)--(11.166,8.268)--(11.179,8.267)--(11.192,8.266)--(11.205,8.265)%
  --(11.218,8.264)--(11.244,8.261)--(11.257,8.259)--(11.270,8.257)--(11.283,8.255)--(11.296,8.254)%
  --(11.309,8.252)--(11.323,8.251)--(11.336,8.249)--(11.349,8.248)--(11.362,8.247)--(11.375,8.246)%
  --(11.388,8.245)--(11.401,8.244)--(11.427,8.242)--(11.441,8.241)--(11.454,8.240)--(11.467,8.238)%
  --(11.480,8.237)--(11.493,8.236)--(11.507,8.234)--(11.520,8.233)--(11.533,8.232)--(11.546,8.231)%
  --(11.559,8.230)--(11.572,8.229)--(11.585,8.229)--(11.599,8.228)--(11.612,8.227)--(11.625,8.226)%
  --(11.639,8.224)--(11.652,8.223)--(11.665,8.222)--(11.678,8.220)--(11.691,8.219)--(11.705,8.218)%
  --(11.718,8.218)--(11.731,8.217)--(11.744,8.216)--(11.757,8.215)--(11.770,8.214)--(11.797,8.211)%
  --(11.810,8.210)--(11.823,8.208)--(11.836,8.207)--(11.850,8.206)--(11.863,8.204)--(11.876,8.202)%
  --(11.889,8.201)--(11.903,8.200)--(11.916,8.198)--(11.929,8.197)--(11.942,8.196)--(11.955,8.194)%
  --(11.968,8.193)--(11.994,8.189)--(12.008,8.188)--(12.021,8.186)--(12.034,8.184)--(12.047,8.182)%
  --(12.061,8.179)--(12.074,8.177)--(12.087,8.175)--(12.100,8.172)--(12.140,8.166)--(12.153,8.163)%
  --(12.166,8.161)--(12.179,8.158)--(12.205,8.153)--(12.219,8.149)--(12.232,8.145)--(12.245,8.142)%
  --(12.258,8.138)--(12.272,8.134)--(12.285,8.131)--(12.312,8.125)--(12.325,8.122)--(12.338,8.119)%
  --(12.351,8.116)--(12.365,8.113)--(12.378,8.112)--(12.391,8.111)--(12.405,8.110)--(12.431,8.110)%
  --(12.445,8.109)--(12.458,8.108)--(12.472,8.108)--(12.485,8.108)--(12.499,8.107)--(12.512,8.107)%
  --(12.526,8.106)--(12.539,8.106)--(12.552,8.106)--(12.566,8.105)--(12.579,8.105)--(12.593,8.105)%
  --(12.606,8.105)--(12.619,8.105)--(12.633,8.105)--(12.646,8.105)--(12.659,8.105)--(12.686,8.105)%
  --(12.700,8.105)--(12.713,8.105)--(12.727,8.105)--(12.740,8.105)--(12.753,8.104)--(12.767,8.104)%
  --(12.780,8.104)--(12.794,8.104)--(12.807,8.104)--(12.820,8.104)--(12.834,8.105)--(12.847,8.105)%
  --(12.860,8.105)--(12.874,8.105)--(12.887,8.106)--(12.900,8.106)--(12.914,8.106)--(12.927,8.106)%
  --(12.940,8.106)--(12.954,8.106)--(12.967,8.106)--(12.981,8.106)--(12.994,8.106)--(13.008,8.107)%
  --(13.021,8.107)--(13.034,8.107)--(13.061,8.108)--(13.074,8.108)--(13.087,8.108)--(13.101,8.109)%
  --(13.114,8.109)--(13.127,8.110)--(13.140,8.110)--(13.154,8.110)--(13.167,8.110)--(13.181,8.111)%
  --(13.194,8.111)--(13.207,8.112)--(13.220,8.112)--(13.233,8.113)--(13.247,8.114)--(13.260,8.115)%
  --(13.273,8.115)--(13.300,8.115)--(13.313,8.115)--(13.327,8.116)--(13.340,8.116)--(13.353,8.116)%
  --(13.366,8.117)--(13.379,8.118)--(13.392,8.119)--(13.405,8.120)--(13.419,8.120)--(13.432,8.121)%
  --(13.445,8.121)--(13.458,8.122)--(13.472,8.122)--(13.485,8.123)--(13.498,8.124)--(13.511,8.126)%
  --(13.537,8.127)--(13.550,8.127)--(13.563,8.128)--(13.576,8.129)--(13.588,8.131)--(13.601,8.132)%
  --(13.615,8.133)--(13.627,8.134)--(13.640,8.136)--(13.652,8.138)--(13.665,8.139)--(13.677,8.142)%
  --(13.689,8.145)--(13.700,8.149)--(13.711,8.153)--(13.733,8.162)--(13.744,8.167)--(13.755,8.171)%
  --(13.766,8.176)--(13.776,8.181)--(13.787,8.185)--(13.798,8.190)--(13.809,8.195)--(13.820,8.198)%
  --(13.831,8.203)--(13.841,8.208)--(13.853,8.212)--(13.863,8.217)--(13.874,8.221)--(13.885,8.226)%
  --(13.896,8.231)--(13.917,8.239)--(13.928,8.244)--(13.948,8.255)--(13.957,8.261)--(13.968,8.266)%
  --(13.978,8.271)--(13.988,8.276)--(14.000,8.280)--(14.010,8.285)--(14.021,8.288)--(14.032,8.292)%
  --(14.064,8.304)--(14.076,8.306)--(14.087,8.310)--(14.097,8.314)--(14.109,8.316)--(14.120,8.320)%
  --(14.131,8.324)--(14.142,8.326)--(14.154,8.329)--(14.164,8.334)--(14.176,8.336)--(14.187,8.339)%
  --(14.209,8.346)--(14.220,8.349)--(14.231,8.352)--(14.242,8.356)--(14.254,8.358)--(14.265,8.360)%
  --(14.276,8.363)--(14.287,8.367)--(14.298,8.369)--(14.310,8.371)--(14.321,8.374)--(14.332,8.377)%
  --(14.355,8.382)--(14.366,8.384)--(14.378,8.386)--(14.389,8.388)--(14.401,8.391)--(14.412,8.393)%
  --(14.423,8.396)--(14.434,8.398)--(14.445,8.401)--(14.456,8.404)--(14.467,8.407)--(14.490,8.411)%
  --(14.501,8.413)--(14.512,8.415)--(14.523,8.418)--(14.534,8.420)--(14.545,8.423)--(14.556,8.426)%
  --(14.566,8.429)--(14.576,8.432)--(14.587,8.435)--(14.598,8.437)--(14.609,8.440)--(14.620,8.442)%
  --(14.630,8.444)--(14.641,8.447)--(14.651,8.450)--(14.661,8.453)--(14.671,8.456)--(14.681,8.459)%
  --(14.690,8.463)--(14.700,8.466)--(14.710,8.469)--(14.720,8.471)--(14.730,8.474)--(14.740,8.477)%
  --(14.749,8.480)--(14.759,8.483)--(14.768,8.487)--(14.776,8.490)--(14.783,8.493)--(14.790,8.496)%
  --(14.796,8.500)--(14.803,8.503)--(14.810,8.506)--(14.817,8.509)--(14.824,8.512)--(14.832,8.515)%
  --(14.839,8.518)--(14.846,8.522)--(14.852,8.525)--(14.859,8.529)--(14.866,8.533)--(14.873,8.537)%
  --(14.879,8.541)--(14.892,8.550)--(14.899,8.554)--(14.906,8.559)--(14.912,8.563)--(14.918,8.568)%
  --(14.924,8.573)--(14.930,8.578)--(14.936,8.584)--(14.942,8.589)--(14.947,8.595)--(14.954,8.600)%
  --(14.961,8.605)--(14.968,8.610)--(14.975,8.615)--(14.981,8.620)--(14.988,8.625)--(14.994,8.631)%
  --(15.008,8.642)--(15.014,8.648)--(15.020,8.654)--(15.027,8.660)--(15.033,8.666)--(15.040,8.672)%
  --(15.046,8.679)--(15.052,8.686)--(15.058,8.692)--(15.065,8.699)--(15.071,8.706)--(15.077,8.713)%
  --(15.082,8.720)--(15.088,8.727)--(15.095,8.734)--(15.101,8.742)--(15.107,8.749)--(15.112,8.757)%
  --(15.118,8.765)--(15.124,8.774)--(15.136,8.790)--(15.141,8.799)--(15.146,8.807)--(15.152,8.816)%
  --(15.158,8.825)--(15.168,8.844)--(15.173,8.854)--(15.178,8.864)--(15.183,8.873)--(15.188,8.883)%
  --(15.193,8.894)--(15.198,8.904)--(15.202,8.916)--(15.206,8.926)--(15.211,8.937)--(15.218,8.960)%
  --(15.221,8.971)--(15.224,8.983)--(15.226,8.994)--(15.229,9.002)--(15.229,9.014)--(15.232,9.030)%
  --(15.236,9.043)--(15.242,9.055)--(15.247,9.067)--(15.252,9.079)--(15.257,9.092)--(15.262,9.105)%
  --(15.266,9.118)--(15.272,9.130)--(15.277,9.143)--(15.281,9.157)--(15.286,9.171)--(15.290,9.185)%
  --(15.295,9.198)--(15.300,9.212)--(15.304,9.226)--(15.308,9.241)--(15.312,9.257)--(15.316,9.272)%
  --(15.321,9.287)--(15.325,9.302)--(15.329,9.317)--(15.333,9.334)--(15.337,9.350)--(15.340,9.367)%
  --(15.344,9.383)--(15.348,9.399)--(15.352,9.416)--(15.355,9.434)--(15.358,9.452)--(15.362,9.470)%
  --(15.365,9.488)--(15.369,9.505)--(15.372,9.523)--(15.375,9.543)--(15.378,9.562)--(15.381,9.582)%
  --(15.384,9.601)--(15.388,9.620)--(15.391,9.640)--(15.393,9.661)--(15.395,9.683)--(15.398,9.704)%
  --(15.401,9.724)--(15.404,9.745)--(15.406,9.766)--(15.408,9.789)--(15.410,9.812)--(15.413,9.835)%
  --(15.415,9.857)--(15.418,9.879)--(15.420,9.903)--(15.421,9.928)--(15.423,9.953)--(15.425,9.977)%
  --(15.427,10.001)--(15.429,10.025)--(15.431,10.051)--(15.432,10.079)--(15.433,10.105)--(15.435,10.131)%
  --(15.436,10.157)--(15.438,10.183)--(15.439,10.212)--(15.439,10.241)--(15.440,10.270)--(15.441,10.298)%
  --(15.443,10.325)--(15.444,10.354)--(15.444,10.385)--(15.444,10.417)--(15.445,10.447)--(15.446,10.477)%
  --(15.447,10.507)--(15.447,10.540)--(15.446,10.574)--(15.446,10.607)--(15.446,10.639)--(15.447,10.671)%
  --(15.447,10.704)--(15.446,10.741)--(15.445,10.777)--(15.445,10.812)--(15.445,10.846)--(15.445,10.881)%
  --(15.444,10.919)--(15.442,10.959)--(15.441,10.997)--(15.441,11.034)--(15.440,11.071)--(15.439,11.110)%
  --(15.437,11.152)--(15.435,11.194)--(15.434,11.233)--(15.433,11.273)--(15.432,11.314)--(15.429,11.360)%
  --(15.426,11.404)--(15.425,11.447)--(15.423,11.489)--(15.421,11.533)--(15.418,11.582)--(15.415,11.629)%
  --(15.413,11.675)--(15.411,11.720)--(15.408,11.767)--(15.405,11.818)--(15.402,11.867)--(15.400,11.914)%
  --(15.398,11.960)--(15.395,12.010)--(15.391,12.064)--(15.388,12.115)--(15.385,12.164)--(15.383,12.213)%
  --(15.379,12.267)--(15.374,12.324)--(15.371,12.378)--(15.368,12.430)--(15.365,12.483)--(15.360,12.544)%
  --(15.355,12.602)--(15.351,12.658)--(15.347,12.714)--(15.342,12.778)--(15.336,12.841)--(15.334,12.870);
\gpsetdashtype{gp dt 3}
\draw[gp path] (10.396,12.870)--(10.379,12.761)--(10.359,12.643)--(10.340,12.524)--(10.319,12.403)%
  --(10.298,12.277)--(10.278,12.157)--(10.259,12.039)--(10.238,11.921)--(10.217,11.800)--(10.196,11.676)%
  --(10.172,11.548)--(10.149,11.405)--(10.133,11.283)--(10.121,11.177)--(10.110,11.080)--(10.101,10.990)%
  --(10.092,10.905)--(10.084,10.824)--(10.076,10.747)--(10.068,10.673)--(10.060,10.601)--(10.053,10.531)%
  --(10.045,10.464)--(10.038,10.398)--(10.031,10.334)--(10.024,10.272)--(10.016,10.211)--(10.009,10.152)%
  --(10.002,10.093)--(9.995,10.037)--(9.988,9.982)--(9.981,9.928)--(9.974,9.875)--(9.968,9.824)%
  --(9.961,9.774)--(9.954,9.726)--(9.947,9.678)--(9.941,9.632)--(9.934,9.588)--(9.928,9.545)%
  --(9.921,9.504)--(9.915,9.463)--(9.909,9.425)--(9.903,9.387)--(9.896,9.352)--(9.890,9.318)%
  --(9.884,9.286)--(9.878,9.256)--(9.871,9.229)--(9.863,9.205)--(9.854,9.185)--(9.850,9.154)%
  --(9.854,9.106)--(9.858,9.064)--(9.863,9.025)--(9.859,9.005)--(9.865,8.971)--(9.873,8.936)%
  --(9.885,8.901)--(9.898,8.867)--(9.913,8.834)--(9.929,8.802)--(9.946,8.774)--(9.964,8.748)%
  --(9.981,8.724)--(9.998,8.703)--(10.010,8.687)--(10.017,8.675)--(10.023,8.665)--(10.030,8.654)%
  --(10.036,8.644)--(10.044,8.634)--(10.051,8.624)--(10.060,8.615)--(10.068,8.605)--(10.077,8.596)%
  --(10.086,8.588)--(10.095,8.579)--(10.106,8.571)--(10.115,8.563)--(10.125,8.556)--(10.146,8.541)%
  --(10.157,8.534)--(10.168,8.527)--(10.179,8.521)--(10.190,8.515)--(10.213,8.505)--(10.223,8.497)%
  --(10.233,8.484)--(10.243,8.475)--(10.254,8.467)--(10.265,8.460)--(10.276,8.454)--(10.288,8.449)%
  --(10.300,8.443)--(10.312,8.438)--(10.324,8.433)--(10.336,8.429)--(10.349,8.425)--(10.361,8.421)%
  --(10.373,8.417)--(10.386,8.412)--(10.398,8.408)--(10.411,8.404)--(10.423,8.401)--(10.435,8.397)%
  --(10.448,8.393)--(10.460,8.390)--(10.485,8.383)--(10.498,8.379)--(10.511,8.376)--(10.524,8.372)%
  --(10.536,8.369)--(10.549,8.366)--(10.562,8.363)--(10.575,8.360)--(10.600,8.354)--(10.613,8.351)%
  --(10.626,8.349)--(10.639,8.346)--(10.652,8.343)--(10.665,8.341)--(10.677,8.336)--(10.690,8.330)%
  --(10.702,8.326)--(10.715,8.323)--(10.727,8.319)--(10.740,8.316)--(10.754,8.313)--(10.766,8.311)%
  --(10.779,8.308)--(10.792,8.306)--(10.806,8.303)--(10.818,8.301)--(10.831,8.299)--(10.844,8.298)%
  --(10.857,8.295)--(10.871,8.293)--(10.884,8.291)--(10.897,8.289)--(10.910,8.287)--(10.923,8.285)%
  --(10.936,8.283)--(10.949,8.282)--(10.962,8.280)--(10.975,8.279)--(11.002,8.276)--(11.015,8.274)%
  --(11.028,8.272)--(11.041,8.270)--(11.054,8.266)--(11.067,8.263)--(11.080,8.261)--(11.093,8.258)%
  --(11.106,8.256)--(11.119,8.254)--(11.145,8.251)--(11.158,8.249)--(11.172,8.247)--(11.185,8.246)%
  --(11.198,8.244)--(11.211,8.242)--(11.224,8.241)--(11.237,8.240)--(11.250,8.238)--(11.264,8.237)%
  --(11.277,8.236)--(11.290,8.234)--(11.303,8.233)--(11.316,8.232)--(11.330,8.231)--(11.343,8.230)%
  --(11.356,8.229)--(11.369,8.227)--(11.382,8.225)--(11.395,8.223)--(11.408,8.222)--(11.421,8.220)%
  --(11.434,8.218)--(11.447,8.217)--(11.461,8.215)--(11.474,8.214)--(11.487,8.213)--(11.500,8.212)%
  --(11.513,8.211)--(11.526,8.210)--(11.540,8.209)--(11.553,8.208)--(11.566,8.207)--(11.579,8.206)%
  --(11.592,8.205)--(11.606,8.204)--(11.619,8.203)--(11.632,8.201)--(11.645,8.200)--(11.658,8.199)%
  --(11.671,8.197)--(11.684,8.196)--(11.697,8.196)--(11.710,8.195)--(11.724,8.193)--(11.737,8.193)%
  --(11.750,8.192)--(11.763,8.191)--(11.777,8.190)--(11.790,8.189)--(11.803,8.188)--(11.816,8.187)%
  --(11.829,8.186)--(11.842,8.185)--(11.855,8.184)--(11.868,8.183)--(11.882,8.182)--(11.895,8.182)%
  --(11.908,8.181)--(11.921,8.180)--(11.934,8.179)--(11.947,8.178)--(11.960,8.177)--(11.973,8.176)%
  --(11.986,8.175)--(12.000,8.174)--(12.013,8.173)--(12.026,8.172)--(12.039,8.171)--(12.052,8.170)%
  --(12.065,8.168)--(12.078,8.167)--(12.091,8.166)--(12.104,8.164)--(12.117,8.163)--(12.144,8.159)%
  --(12.157,8.158)--(12.170,8.156)--(12.183,8.154)--(12.196,8.152)--(12.209,8.150)--(12.222,8.148)%
  --(12.235,8.146)--(12.248,8.144)--(12.261,8.142)--(12.288,8.138)--(12.301,8.135)--(12.314,8.133)%
  --(12.327,8.131)--(12.340,8.128)--(12.353,8.125)--(12.366,8.123)--(12.379,8.120)--(12.392,8.117)%
  --(12.405,8.115)--(12.418,8.113)--(12.432,8.111)--(12.445,8.109)--(12.458,8.106)--(12.471,8.104)%
  --(12.484,8.103)--(12.497,8.101)--(12.510,8.101)--(12.523,8.100)--(12.536,8.099)--(12.549,8.099)%
  --(12.563,8.098)--(12.576,8.097)--(12.589,8.096)--(12.602,8.095)--(12.615,8.094)--(12.628,8.094)%
  --(12.641,8.093)--(12.654,8.092)--(12.667,8.091)--(12.681,8.091)--(12.694,8.090)--(12.707,8.090)%
  --(12.720,8.090)--(12.733,8.091)--(12.746,8.091)--(12.772,8.092)--(12.785,8.091)--(12.798,8.091)%
  --(12.811,8.091)--(12.825,8.091)--(12.838,8.091)--(12.851,8.091)--(12.864,8.092)--(12.877,8.092)%
  --(12.890,8.092)--(12.903,8.091)--(12.916,8.092)--(12.929,8.092)--(12.942,8.093)--(12.955,8.093)%
  --(12.968,8.094)--(12.981,8.094)--(12.994,8.094)--(13.007,8.094)--(13.020,8.095)--(13.033,8.096)%
  --(13.046,8.096)--(13.059,8.097)--(13.072,8.097)--(13.085,8.097)--(13.099,8.097)--(13.112,8.098)%
  --(13.124,8.098)--(13.137,8.099)--(13.150,8.100)--(13.163,8.101)--(13.176,8.101)--(13.189,8.101)%
  --(13.203,8.102)--(13.215,8.102)--(13.228,8.104)--(13.241,8.104)--(13.254,8.105)--(13.267,8.105)%
  --(13.280,8.106)--(13.306,8.108)--(13.319,8.109)--(13.332,8.109)--(13.344,8.111)--(13.357,8.112)%
  --(13.370,8.114)--(13.382,8.116)--(13.394,8.117)--(13.407,8.119)--(13.419,8.122)--(13.443,8.127)%
  --(13.456,8.128)--(13.468,8.131)--(13.480,8.133)--(13.492,8.136)--(13.505,8.138)--(13.516,8.142)%
  --(13.528,8.145)--(13.552,8.151)--(13.576,8.157)--(13.588,8.159)--(13.600,8.162)--(13.612,8.166)%
  --(13.624,8.169)--(13.635,8.173)--(13.647,8.176)--(13.659,8.180)--(13.681,8.191)--(13.692,8.196)%
  --(13.702,8.201)--(13.714,8.207)--(13.725,8.211)--(13.736,8.215)--(13.747,8.220)--(13.759,8.224)%
  --(13.771,8.227)--(13.782,8.231)--(13.794,8.234)--(13.806,8.239)--(13.818,8.241)--(13.841,8.249)%
  --(13.852,8.253)--(13.864,8.256)--(13.876,8.260)--(13.888,8.263)--(13.899,8.267)--(13.911,8.270)%
  --(13.923,8.272)--(13.935,8.276)--(13.947,8.279)--(13.959,8.281)--(13.971,8.285)--(13.983,8.287)%
  --(13.995,8.290)--(14.007,8.293)--(14.019,8.296)--(14.031,8.298)--(14.043,8.300)--(14.055,8.302)%
  --(14.068,8.304)--(14.080,8.307)--(14.092,8.309)--(14.104,8.311)--(14.117,8.312)--(14.129,8.314)%
  --(14.141,8.316)--(14.154,8.319)--(14.166,8.321)--(14.178,8.324)--(14.190,8.326)--(14.215,8.330)%
  --(14.227,8.331)--(14.239,8.333)--(14.252,8.335)--(14.264,8.338)--(14.276,8.340)--(14.288,8.343)%
  --(14.300,8.346)--(14.312,8.348)--(14.324,8.350)--(14.336,8.352)--(14.348,8.355)--(14.360,8.358)%
  --(14.372,8.361)--(14.384,8.363)--(14.396,8.365)--(14.408,8.368)--(14.420,8.370)--(14.432,8.372)%
  --(14.444,8.375)--(14.456,8.377)--(14.468,8.380)--(14.480,8.383)--(14.503,8.388)--(14.515,8.391)%
  --(14.527,8.394)--(14.538,8.397)--(14.550,8.400)--(14.562,8.402)--(14.574,8.405)--(14.586,8.407)%
  --(14.598,8.410)--(14.610,8.412)--(14.622,8.415)--(14.633,8.418)--(14.645,8.420)--(14.657,8.423)%
  --(14.669,8.426)--(14.680,8.429)--(14.692,8.432)--(14.715,8.438)--(14.727,8.441)--(14.738,8.444)%
  --(14.749,8.447)--(14.761,8.450)--(14.772,8.453)--(14.783,8.457)--(14.806,8.464)--(14.817,8.467)%
  --(14.828,8.471)--(14.839,8.474)--(14.850,8.478)--(14.860,8.482)--(14.871,8.487)--(14.882,8.491)%
  --(14.892,8.495)--(14.902,8.500)--(14.912,8.505)--(14.922,8.510)--(14.931,8.516)--(14.941,8.521)%
  --(14.952,8.525)--(14.963,8.528)--(14.974,8.531)--(14.984,8.535)--(14.995,8.539)--(15.005,8.542)%
  --(15.016,8.546)--(15.026,8.550)--(15.036,8.554)--(15.046,8.559)--(15.056,8.563)--(15.066,8.567)%
  --(15.076,8.571)--(15.085,8.576)--(15.095,8.580)--(15.105,8.585)--(15.114,8.589)--(15.123,8.594)%
  --(15.132,8.599)--(15.141,8.604)--(15.150,8.608)--(15.159,8.613)--(15.167,8.618)--(15.176,8.623)%
  --(15.184,8.628)--(15.192,8.633)--(15.199,8.638)--(15.205,8.642)--(15.208,8.645)--(15.215,8.651)%
  --(15.220,8.656)--(15.226,8.662)--(15.231,8.668)--(15.236,8.674)--(15.241,8.680)--(15.246,8.687)%
  --(15.251,8.693)--(15.256,8.700)--(15.261,8.707)--(15.265,8.713)--(15.270,8.721)--(15.275,8.728)%
  --(15.280,8.736)--(15.284,8.743)--(15.289,8.751)--(15.294,8.759)--(15.298,8.768)--(15.303,8.777)%
  --(15.307,8.785)--(15.312,8.794)--(15.316,8.804)--(15.320,8.813)--(15.324,8.823)--(15.328,8.833)%
  --(15.332,8.843)--(15.336,8.853)--(15.340,8.864)--(15.345,8.874)--(15.349,8.885)--(15.353,8.896)%
  --(15.356,8.908)--(15.360,8.920)--(15.363,8.932)--(15.366,8.945)--(15.369,8.959)--(15.371,8.973)%
  --(15.373,8.988)--(15.375,9.005)--(15.376,9.021)--(15.377,9.039)--(15.377,9.057)--(15.377,9.077)%
  --(15.376,9.098)--(15.374,9.120)--(15.372,9.143)--(15.369,9.169)--(15.365,9.197)--(15.366,9.217)%
  --(15.368,9.235)--(15.371,9.253)--(15.373,9.271)--(15.375,9.289)--(15.377,9.309)--(15.379,9.328)%
  --(15.380,9.347)--(15.382,9.368)--(15.383,9.389)--(15.385,9.410)--(15.386,9.432)--(15.387,9.455)%
  --(15.388,9.477)--(15.389,9.500)--(15.389,9.525)--(15.389,9.549)--(15.390,9.573)--(15.390,9.599)%
  --(15.389,9.626)--(15.389,9.652)--(15.389,9.680)--(15.388,9.708)--(15.387,9.737)--(15.387,9.765)%
  --(15.385,9.796)--(15.384,9.826)--(15.383,9.857)--(15.381,9.890)--(15.379,9.923)--(15.377,9.956)%
  --(15.375,9.991)--(15.372,10.026)--(15.370,10.061)--(15.367,10.098)--(15.364,10.136)--(15.362,10.173)%
  --(15.358,10.213)--(15.355,10.253)--(15.352,10.293)--(15.348,10.336)--(15.344,10.378)--(15.341,10.421)%
  --(15.336,10.466)--(15.332,10.511)--(15.328,10.557)--(15.323,10.604)--(15.319,10.652)--(15.314,10.701)%
  --(15.309,10.751)--(15.305,10.801)--(15.299,10.853)--(15.294,10.906)--(15.289,10.959)--(15.284,11.014)%
  --(15.278,11.069)--(15.272,11.126)--(15.267,11.183)--(15.261,11.239)--(15.255,11.299)--(15.250,11.358)%
  --(15.244,11.417)--(15.238,11.479)--(15.232,11.540)--(15.226,11.604)--(15.220,11.666)--(15.214,11.730)%
  --(15.208,11.796)--(15.202,11.860)--(15.195,11.928)--(15.189,11.996)--(15.182,12.063)--(15.176,12.134)%
  --(15.169,12.203)--(15.162,12.276)--(15.156,12.348)--(15.149,12.421)--(15.142,12.496)--(15.135,12.569)%
  --(15.128,12.647)--(15.121,12.722)--(15.113,12.801)--(15.107,12.870);
\gpsetdashtype{gp dt 4}
\draw[gp path] (10.889,12.870)--(10.878,12.785)--(10.853,12.624)--(10.829,12.467)--(10.797,12.278)%
  --(10.749,12.017)--(10.705,11.770)--(10.681,11.611)--(10.659,11.446)--(10.632,11.171)--(10.616,11.014)%
  --(10.604,10.906)--(10.592,10.788)--(10.580,10.669)--(10.569,10.559)--(10.535,10.359)--(10.524,10.269)%
  --(10.513,10.165)--(10.501,10.046)--(10.487,9.915)--(10.473,9.761)--(10.458,9.570)--(10.438,9.356)%
  --(10.422,9.302)--(10.407,9.283)--(10.392,9.261)--(10.374,9.220)--(10.356,9.171)--(10.340,9.122)%
  --(10.322,9.067)--(10.305,9.007)--(10.295,8.963)--(10.289,8.925)--(10.284,8.892)--(10.282,8.860)%
  --(10.281,8.830)--(10.282,8.803)--(10.284,8.778)--(10.286,8.755)--(10.290,8.734)--(10.294,8.713)%
  --(10.299,8.695)--(10.305,8.677)--(10.312,8.660)--(10.318,8.645)--(10.325,8.630)--(10.333,8.617)%
  --(10.341,8.604)--(10.349,8.591)--(10.358,8.580)--(10.367,8.569)--(10.377,8.558)--(10.386,8.549)%
  --(10.396,8.540)--(10.406,8.532)--(10.427,8.517)--(10.438,8.510)--(10.449,8.505)--(10.460,8.499)%
  --(10.472,8.493)--(10.483,8.487)--(10.489,8.467)--(10.498,8.455)--(10.508,8.446)--(10.530,8.432)%
  --(10.541,8.426)--(10.553,8.421)--(10.565,8.416)--(10.577,8.412)--(10.589,8.408)--(10.601,8.404)%
  --(10.613,8.400)--(10.625,8.397)--(10.637,8.393)--(10.649,8.390)--(10.661,8.386)--(10.674,8.383)%
  --(10.686,8.379)--(10.698,8.376)--(10.711,8.372)--(10.723,8.369)--(10.735,8.365)--(10.747,8.362)%
  --(10.760,8.359)--(10.772,8.356)--(10.784,8.353)--(10.797,8.350)--(10.809,8.347)--(10.822,8.344)%
  --(10.834,8.341)--(10.847,8.338)--(10.859,8.335)--(10.872,8.333)--(10.884,8.330)--(10.897,8.328)%
  --(10.909,8.325)--(10.922,8.323)--(10.935,8.321)--(10.947,8.318)--(10.960,8.316)--(10.971,8.307)%
  --(10.982,8.301)--(10.994,8.297)--(11.007,8.293)--(11.019,8.290)--(11.032,8.288)--(11.044,8.286)%
  --(11.057,8.284)--(11.070,8.283)--(11.083,8.281)--(11.096,8.279)--(11.109,8.277)--(11.122,8.276)%
  --(11.135,8.274)--(11.147,8.272)--(11.160,8.270)--(11.173,8.268)--(11.186,8.267)--(11.198,8.265)%
  --(11.211,8.263)--(11.224,8.262)--(11.237,8.260)--(11.250,8.258)--(11.262,8.256)--(11.275,8.254)%
  --(11.288,8.253)--(11.301,8.251)--(11.314,8.250)--(11.327,8.248)--(11.340,8.247)--(11.353,8.245)%
  --(11.366,8.244)--(11.379,8.243)--(11.391,8.241)--(11.404,8.240)--(11.417,8.239)--(11.430,8.235)%
  --(11.442,8.231)--(11.455,8.228)--(11.468,8.226)--(11.481,8.224)--(11.493,8.222)--(11.506,8.221)%
  --(11.519,8.219)--(11.532,8.218)--(11.545,8.217)--(11.558,8.216)--(11.584,8.213)--(11.597,8.212)%
  --(11.610,8.211)--(11.623,8.209)--(11.636,8.208)--(11.649,8.207)--(11.662,8.206)--(11.675,8.205)%
  --(11.688,8.204)--(11.701,8.203)--(11.714,8.202)--(11.727,8.201)--(11.740,8.200)--(11.753,8.199)%
  --(11.778,8.198)--(11.792,8.197)--(11.805,8.196)--(11.818,8.194)--(11.831,8.192)--(11.844,8.190)%
  --(11.857,8.189)--(11.870,8.188)--(11.883,8.187)--(11.895,8.186)--(11.908,8.185)--(11.921,8.184)%
  --(11.934,8.183)--(11.947,8.183)--(11.973,8.181)--(11.986,8.180)--(11.999,8.179)--(12.012,8.179)%
  --(12.026,8.178)--(12.039,8.177)--(12.052,8.177)--(12.065,8.176)--(12.078,8.175)--(12.091,8.174)%
  --(12.104,8.173)--(12.117,8.172)--(12.129,8.171)--(12.143,8.171)--(12.155,8.170)--(12.181,8.169)%
  --(12.195,8.168)--(12.208,8.167)--(12.221,8.167)--(12.234,8.166)--(12.247,8.165)--(12.260,8.164)%
  --(12.273,8.163)--(12.286,8.162)--(12.299,8.162)--(12.312,8.161)--(12.325,8.160)--(12.338,8.159)%
  --(12.351,8.158)--(12.364,8.156)--(12.389,8.152)--(12.402,8.150)--(12.415,8.147)--(12.428,8.144)%
  --(12.441,8.141)--(12.454,8.138)--(12.467,8.135)--(12.480,8.132)--(12.493,8.129)--(12.506,8.125)%
  --(12.519,8.121)--(12.532,8.117)--(12.544,8.112)--(12.557,8.108)--(12.570,8.105)--(12.583,8.100)%
  --(12.596,8.096)--(12.622,8.087)--(12.635,8.084)--(12.648,8.081)--(12.661,8.079)--(12.674,8.077)%
  --(12.687,8.076)--(12.700,8.075)--(12.713,8.074)--(12.727,8.074)--(12.740,8.074)--(12.753,8.073)%
  --(12.765,8.073)--(12.779,8.073)--(12.791,8.072)--(12.805,8.072)--(12.818,8.071)--(12.831,8.070)%
  --(12.844,8.071)--(12.857,8.071)--(12.870,8.071)--(12.883,8.070)--(12.896,8.071)--(12.909,8.071)%
  --(12.922,8.071)--(12.935,8.071)--(12.948,8.071)--(12.961,8.071)--(12.974,8.071)--(12.988,8.071)%
  --(13.001,8.071)--(13.027,8.072)--(13.040,8.072)--(13.053,8.072)--(13.066,8.073)--(13.078,8.073)%
  --(13.091,8.074)--(13.105,8.074)--(13.117,8.075)--(13.130,8.076)--(13.142,8.077)--(13.155,8.078)%
  --(13.181,8.080)--(13.193,8.081)--(13.206,8.081)--(13.219,8.082)--(13.231,8.084)--(13.244,8.085)%
  --(13.257,8.086)--(13.269,8.087)--(13.281,8.089)--(13.294,8.089)--(13.307,8.090)--(13.319,8.092)%
  --(13.331,8.094)--(13.343,8.097)--(13.355,8.099)--(13.377,8.107)--(13.387,8.112)--(13.397,8.119)%
  --(13.406,8.125)--(13.415,8.133)--(13.425,8.140)--(13.434,8.148)--(13.443,8.156)--(13.452,8.163)%
  --(13.461,8.171)--(13.471,8.177)--(13.481,8.182)--(13.492,8.187)--(13.502,8.192)--(13.524,8.200)%
  --(13.535,8.205)--(13.545,8.210)--(13.556,8.214)--(13.566,8.219)--(13.576,8.224)--(13.587,8.229)%
  --(13.597,8.235)--(13.607,8.240)--(13.618,8.244)--(13.629,8.248)--(13.640,8.252)--(13.651,8.255)%
  --(13.662,8.259)--(13.684,8.266)--(13.696,8.269)--(13.708,8.271)--(13.720,8.273)--(13.732,8.275)%
  --(13.744,8.277)--(13.756,8.279)--(13.768,8.280)--(13.780,8.282)--(13.793,8.283)--(13.805,8.285)%
  --(13.817,8.286)--(13.829,8.288)--(13.841,8.290)--(13.853,8.292)--(13.865,8.294)--(13.877,8.296)%
  --(13.889,8.298)--(13.901,8.298)--(13.914,8.300)--(13.926,8.301)--(13.938,8.303)--(13.950,8.305)%
  --(13.962,8.307)--(13.973,8.309)--(13.985,8.311)--(13.997,8.313)--(14.021,8.317)--(14.033,8.319)%
  --(14.045,8.320)--(14.057,8.322)--(14.069,8.324)--(14.081,8.326)--(14.092,8.328)--(14.104,8.331)%
  --(14.116,8.333)--(14.127,8.335)--(14.139,8.338)--(14.150,8.340)--(14.174,8.345)--(14.186,8.346)%
  --(14.198,8.348)--(14.209,8.350)--(14.221,8.352)--(14.233,8.354)--(14.245,8.356)--(14.256,8.358)%
  --(14.268,8.361)--(14.280,8.363)--(14.291,8.365)--(14.314,8.370)--(14.326,8.372)--(14.337,8.374)%
  --(14.349,8.377)--(14.360,8.379)--(14.371,8.382)--(14.382,8.385)--(14.393,8.387)--(14.405,8.390)%
  --(14.416,8.393)--(14.437,8.400)--(14.448,8.403)--(14.459,8.406)--(14.469,8.410)--(14.479,8.414)%
  --(14.489,8.418)--(14.499,8.423)--(14.509,8.428)--(14.519,8.431)--(14.530,8.434)--(14.541,8.437)%
  --(14.552,8.439)--(14.562,8.442)--(14.573,8.445)--(14.584,8.449)--(14.594,8.452)--(14.605,8.455)%
  --(14.615,8.458)--(14.636,8.465)--(14.646,8.468)--(14.656,8.472)--(14.666,8.475)--(14.676,8.479)%
  --(14.687,8.482)--(14.697,8.485)--(14.707,8.489)--(14.717,8.492)--(14.727,8.496)--(14.737,8.499)%
  --(14.746,8.503)--(14.756,8.506)--(14.765,8.510)--(14.774,8.514)--(14.783,8.518)--(14.792,8.522)%
  --(14.801,8.526)--(14.809,8.530)--(14.816,8.534)--(14.823,8.538)--(14.830,8.542)--(14.837,8.546)%
  --(14.844,8.550)--(14.851,8.554)--(14.857,8.559)--(14.864,8.563)--(14.871,8.568)--(14.878,8.573)%
  --(14.884,8.577)--(14.891,8.582)--(14.897,8.587)--(14.903,8.592)--(14.910,8.597)--(14.917,8.602)%
  --(14.923,8.608)--(14.929,8.613)--(14.936,8.619)--(14.941,8.624)--(14.948,8.630)--(14.954,8.636)%
  --(14.960,8.642)--(14.965,8.649)--(14.971,8.655)--(14.976,8.662)--(14.981,8.669)--(14.986,8.676)%
  --(14.991,8.683)--(14.995,8.691)--(14.999,8.699)--(15.003,8.706)--(15.006,8.714)--(15.008,8.721)%
  --(15.010,8.728)--(15.023,8.775)--(15.025,8.790)--(15.026,8.806)--(15.025,8.824)--(15.023,8.845)%
  --(15.025,8.862)--(15.030,8.872)--(15.036,8.881)--(15.042,8.891)--(15.048,8.900)--(15.054,8.909)%
  --(15.060,8.918)--(15.067,8.927)--(15.073,8.937)--(15.078,8.948)--(15.083,8.959)--(15.088,8.971)%
  --(15.093,8.984)--(15.097,8.996)--(15.102,9.009)--(15.106,9.023)--(15.110,9.036)--(15.114,9.050)%
  --(15.118,9.065)--(15.121,9.079)--(15.125,9.094)--(15.128,9.110)--(15.131,9.126)--(15.135,9.141)%
  --(15.138,9.158)--(15.140,9.175)--(15.143,9.193)--(15.145,9.211)--(15.147,9.229)--(15.149,9.248)%
  --(15.151,9.268)--(15.152,9.288)--(15.154,9.309)--(15.155,9.330)--(15.155,9.351)--(15.156,9.374)%
  --(15.156,9.397)--(15.157,9.420)--(15.156,9.445)--(15.156,9.470)--(15.156,9.496)--(15.155,9.522)%
  --(15.154,9.548)--(15.152,9.576)--(15.151,9.604)--(15.149,9.633)--(15.147,9.663)--(15.145,9.693)%
  --(15.143,9.724)--(15.140,9.756)--(15.137,9.789)--(15.134,9.822)--(15.130,9.856)--(15.127,9.891)%
  --(15.123,9.926)--(15.119,9.962)--(15.115,9.999)--(15.110,10.037)--(15.106,10.076)--(15.101,10.116)%
  --(15.096,10.156)--(15.091,10.197)--(15.086,10.239)--(15.081,10.282)--(15.075,10.325)--(15.070,10.370)%
  --(15.065,10.413)--(15.059,10.459)--(15.054,10.504)--(15.049,10.550)--(15.043,10.597)--(15.038,10.644)%
  --(15.033,10.692)--(15.027,10.741)--(15.022,10.790)--(15.016,10.840)--(15.011,10.891)--(15.005,10.942)%
  --(14.999,10.995)--(14.994,11.048)--(14.988,11.103)--(14.982,11.158)--(14.976,11.214)--(14.970,11.271)%
  --(14.964,11.329)--(14.958,11.388)--(14.952,11.448)--(14.945,11.509)--(14.939,11.572)--(14.933,11.635)%
  --(14.926,11.700)--(14.919,11.765)--(14.912,11.833)--(14.906,11.901)--(14.899,11.971)--(14.892,12.041)%
  --(14.885,12.113)--(14.878,12.186)--(14.871,12.261)--(14.864,12.336)--(14.856,12.413)--(14.849,12.491)%
  --(14.842,12.571)--(14.835,12.652)--(14.827,12.734)--(14.820,12.818)--(14.815,12.870);
\gpsetdashtype{gp dt solid}
\draw[gp path] (9.504,12.870)--(9.504,7.882)--(15.447,7.882)--(15.447,12.870)--cycle;
%% coordinates of the plot area
\gpdefrectangularnode{gp plot 5}{\pgfpoint{9.504cm}{7.882cm}}{\pgfpoint{15.447cm}{12.870cm}}
\gpcolor{color=gp lt color axes}
\gpsetlinetype{gp lt axes}
\gpsetdashtype{gp dt axes}
\gpsetlinewidth{0.50}
\draw[gp path] (9.136,0.985)--(15.447,0.985);
\gpcolor{color=gp lt color border}
\gpsetlinetype{gp lt border}
\gpsetdashtype{gp dt solid}
\gpsetlinewidth{1.00}
\draw[gp path] (9.136,0.985)--(9.316,0.985);
\draw[gp path] (15.447,0.985)--(15.267,0.985);
\node[gp node right] at (8.952,0.985) {$35$};
\gpcolor{color=gp lt color axes}
\gpsetlinetype{gp lt axes}
\gpsetdashtype{gp dt axes}
\gpsetlinewidth{0.50}
\draw[gp path] (9.136,1.654)--(15.447,1.654);
\gpcolor{color=gp lt color border}
\gpsetlinetype{gp lt border}
\gpsetdashtype{gp dt solid}
\gpsetlinewidth{1.00}
\draw[gp path] (9.136,1.654)--(9.316,1.654);
\draw[gp path] (15.447,1.654)--(15.267,1.654);
\node[gp node right] at (8.952,1.654) {$40$};
\gpcolor{color=gp lt color axes}
\gpsetlinetype{gp lt axes}
\gpsetdashtype{gp dt axes}
\gpsetlinewidth{0.50}
\draw[gp path] (9.136,2.322)--(15.447,2.322);
\gpcolor{color=gp lt color border}
\gpsetlinetype{gp lt border}
\gpsetdashtype{gp dt solid}
\gpsetlinewidth{1.00}
\draw[gp path] (9.136,2.322)--(9.316,2.322);
\draw[gp path] (15.447,2.322)--(15.267,2.322);
\node[gp node right] at (8.952,2.322) {$45$};
\gpcolor{color=gp lt color axes}
\gpsetlinetype{gp lt axes}
\gpsetdashtype{gp dt axes}
\gpsetlinewidth{0.50}
\draw[gp path] (9.136,2.991)--(15.447,2.991);
\gpcolor{color=gp lt color border}
\gpsetlinetype{gp lt border}
\gpsetdashtype{gp dt solid}
\gpsetlinewidth{1.00}
\draw[gp path] (9.136,2.991)--(9.316,2.991);
\draw[gp path] (15.447,2.991)--(15.267,2.991);
\node[gp node right] at (8.952,2.991) {$50$};
\gpcolor{color=gp lt color axes}
\gpsetlinetype{gp lt axes}
\gpsetdashtype{gp dt axes}
\gpsetlinewidth{0.50}
\draw[gp path] (9.136,3.659)--(15.447,3.659);
\gpcolor{color=gp lt color border}
\gpsetlinetype{gp lt border}
\gpsetdashtype{gp dt solid}
\gpsetlinewidth{1.00}
\draw[gp path] (9.136,3.659)--(9.316,3.659);
\draw[gp path] (15.447,3.659)--(15.267,3.659);
\node[gp node right] at (8.952,3.659) {$55$};
\gpcolor{color=gp lt color axes}
\gpsetlinetype{gp lt axes}
\gpsetdashtype{gp dt axes}
\gpsetlinewidth{0.50}
\draw[gp path] (9.136,4.328)--(15.447,4.328);
\gpcolor{color=gp lt color border}
\gpsetlinetype{gp lt border}
\gpsetdashtype{gp dt solid}
\gpsetlinewidth{1.00}
\draw[gp path] (9.136,4.328)--(9.316,4.328);
\draw[gp path] (15.447,4.328)--(15.267,4.328);
\node[gp node right] at (8.952,4.328) {$60$};
\gpcolor{color=gp lt color axes}
\gpsetlinetype{gp lt axes}
\gpsetdashtype{gp dt axes}
\gpsetlinewidth{0.50}
\draw[gp path] (9.136,4.996)--(15.447,4.996);
\gpcolor{color=gp lt color border}
\gpsetlinetype{gp lt border}
\gpsetdashtype{gp dt solid}
\gpsetlinewidth{1.00}
\draw[gp path] (9.136,4.996)--(9.316,4.996);
\draw[gp path] (15.447,4.996)--(15.267,4.996);
\node[gp node right] at (8.952,4.996) {$65$};
\gpcolor{color=gp lt color axes}
\gpsetlinetype{gp lt axes}
\gpsetdashtype{gp dt axes}
\gpsetlinewidth{0.50}
\draw[gp path] (9.136,5.665)--(15.447,5.665);
\gpcolor{color=gp lt color border}
\gpsetlinetype{gp lt border}
\gpsetdashtype{gp dt solid}
\gpsetlinewidth{1.00}
\draw[gp path] (9.136,5.665)--(9.316,5.665);
\draw[gp path] (15.447,5.665)--(15.267,5.665);
\node[gp node right] at (8.952,5.665) {$70$};
\gpcolor{color=gp lt color axes}
\gpsetlinetype{gp lt axes}
\gpsetdashtype{gp dt axes}
\gpsetlinewidth{0.50}
\draw[gp path] (9.136,0.985)--(9.136,5.665);
\gpcolor{color=gp lt color border}
\gpsetlinetype{gp lt border}
\gpsetdashtype{gp dt solid}
\gpsetlinewidth{1.00}
\draw[gp path] (9.136,0.985)--(9.136,1.165);
\node[gp node center] at (9.136,0.677) {$0$};
\draw[gp path] (9.767,0.985)--(9.767,1.075);
\draw[gp path] (10.398,0.985)--(10.398,1.075);
\draw[gp path] (11.029,0.985)--(11.029,1.075);
\draw[gp path] (11.660,0.985)--(11.660,1.075);
\gpcolor{color=gp lt color axes}
\gpsetlinetype{gp lt axes}
\gpsetdashtype{gp dt axes}
\gpsetlinewidth{0.50}
\draw[gp path] (12.292,0.985)--(12.292,5.665);
\gpcolor{color=gp lt color border}
\gpsetlinetype{gp lt border}
\gpsetdashtype{gp dt solid}
\gpsetlinewidth{1.00}
\draw[gp path] (12.292,0.985)--(12.292,1.165);
\node[gp node center] at (12.292,0.677) {$0.5$};
\draw[gp path] (12.923,0.985)--(12.923,1.075);
\draw[gp path] (13.554,0.985)--(13.554,1.075);
\draw[gp path] (14.185,0.985)--(14.185,1.075);
\draw[gp path] (14.816,0.985)--(14.816,1.075);
\gpcolor{color=gp lt color axes}
\gpsetlinetype{gp lt axes}
\gpsetdashtype{gp dt axes}
\gpsetlinewidth{0.50}
\draw[gp path] (15.447,0.985)--(15.447,5.665);
\gpcolor{color=gp lt color border}
\gpsetlinetype{gp lt border}
\gpsetdashtype{gp dt solid}
\gpsetlinewidth{1.00}
\draw[gp path] (15.447,0.985)--(15.447,1.165);
\node[gp node center] at (15.447,0.677) {$1$};
\draw[gp path] (13.301,5.665)--(13.301,5.485);
\node[gp node center] at (13.301,5.973) {\footnotesize cruise};
\draw[gp path] (9.136,5.665)--(9.136,0.985)--(15.447,0.985)--(15.447,5.665)--cycle;
\node[gp node center,rotate=-270] at (8.246,3.325) {$C_d \cdot 10^4$};
\node[gp node center] at (12.291,0.215) {$C_l$};
\node[gp node center] at (12.291,6.743) {Polar de Arrasto no Balde Laminar};
\draw[gp path] (9.136,5.226)--(9.143,5.210)--(9.176,5.157)--(9.205,5.090)--(9.238,5.037)%
  --(9.267,4.943)--(9.296,4.863)--(9.327,4.782)--(9.358,4.702)--(9.387,4.595)--(9.417,4.502)%
  --(9.449,4.408)--(9.480,4.314)--(9.510,4.208)--(9.541,4.101)--(9.603,3.887)--(9.636,3.806)%
  --(9.669,3.726)--(9.701,3.619)--(9.734,3.526)--(9.768,3.432)--(9.801,3.352)--(9.835,3.258)%
  --(9.868,3.151)--(9.936,2.951)--(9.969,2.857)--(10.003,2.750)--(10.036,2.630)--(10.069,2.509)%
  --(10.103,2.389)--(10.137,2.242)--(10.168,2.055)--(10.198,1.827)--(10.225,1.520)--(10.261,1.426)%
  --(10.300,1.400)--(10.340,1.386)--(10.379,1.359)--(10.418,1.359)--(10.459,1.359)--(10.499,1.359)%
  --(10.539,1.346)--(10.578,1.333)--(10.618,1.333)--(10.658,1.319)--(10.698,1.319)--(10.738,1.319)%
  --(10.779,1.319)--(10.819,1.333)--(10.859,1.333)--(10.899,1.333)--(10.939,1.346)--(10.979,1.346)%
  --(11.019,1.359)--(11.058,1.359)--(11.098,1.359)--(11.137,1.359)--(11.176,1.359)--(11.215,1.359)%
  --(11.255,1.359)--(11.294,1.359)--(11.333,1.359)--(11.372,1.359)--(11.411,1.373)--(11.450,1.373)%
  --(11.489,1.386)--(11.528,1.400)--(11.567,1.400)--(11.606,1.413)--(11.645,1.426)--(11.684,1.440)%
  --(11.724,1.440)--(11.765,1.440)--(11.805,1.440)--(11.845,1.453)--(11.886,1.453)--(11.926,1.453)%
  --(11.966,1.453)--(12.007,1.453)--(12.047,1.453)--(12.088,1.466)--(12.127,1.466)--(12.168,1.466)%
  --(12.208,1.480)--(12.248,1.466)--(12.288,1.466)--(12.327,1.466)--(12.367,1.453)--(12.406,1.453)%
  --(12.446,1.453)--(12.485,1.466)--(12.524,1.466)--(12.563,1.466)--(12.602,1.480)--(12.640,1.480)%
  --(12.678,1.493)--(12.717,1.493)--(12.755,1.506)--(12.793,1.520)--(12.831,1.533)--(12.870,1.547)%
  --(12.907,1.547)--(12.947,1.560)--(12.987,1.560)--(13.027,1.560)--(13.067,1.560)--(13.108,1.560)%
  --(13.147,1.560)--(13.188,1.560)--(13.227,1.560)--(13.268,1.560)--(13.308,1.560)--(13.347,1.560)%
  --(13.385,1.573)--(13.424,1.573)--(13.462,1.587)--(13.500,1.600)--(13.538,1.613)--(13.575,1.627)%
  --(13.611,1.654)--(13.647,1.667)--(13.686,1.680)--(13.726,1.680)--(13.764,1.694)--(13.799,1.720)%
  --(13.828,1.774)--(13.862,1.814)--(13.883,1.908)--(13.891,2.081)--(13.885,2.322)--(13.879,2.563)%
  --(13.878,2.777)--(13.879,2.964)--(13.881,3.138)--(13.886,3.285)--(13.888,3.445)--(13.889,3.606)%
  --(13.896,3.726)--(13.896,3.846)--(13.892,3.940)--(13.883,4.034)--(13.877,4.154)--(13.869,4.288)%
  --(13.857,4.435)--(13.836,4.769)--(13.830,4.916)--(13.817,5.103)--(13.810,5.264)--(13.802,5.438)%
  --(13.792,5.625)--(13.791,5.665);
\gpsetdashtype{gp dt 2}
\draw[gp path] (9.136,4.349)--(9.173,4.274)--(9.212,4.221)--(9.250,4.154)--(9.289,4.087)%
  --(9.405,3.913)--(9.444,3.846)--(9.483,3.780)--(9.522,3.713)--(9.600,3.566)--(9.639,3.472)%
  --(9.678,3.365)--(9.717,3.272)--(9.756,3.165)--(9.795,3.071)--(9.835,2.991)--(9.913,2.830)%
  --(9.952,2.750)--(9.991,2.670)--(10.030,2.576)--(10.069,2.509)--(10.109,2.483)--(10.148,2.442)%
  --(10.187,2.429)--(10.266,2.416)--(10.306,2.389)--(10.346,2.376)--(10.386,2.376)--(10.425,2.376)%
  --(10.465,2.349)--(10.505,2.336)--(10.545,2.322)--(10.584,2.322)--(10.624,2.309)--(10.663,2.295)%
  --(10.702,2.282)--(10.742,2.282)--(10.781,2.282)--(10.821,2.295)--(10.860,2.295)--(10.899,2.295)%
  --(10.939,2.295)--(11.018,2.282)--(11.058,2.282)--(11.097,2.282)--(11.137,2.282)--(11.177,2.282)%
  --(11.217,2.269)--(11.256,2.269)--(11.296,2.255)--(11.335,2.255)--(11.375,2.255)--(11.414,2.269)%
  --(11.453,2.282)--(11.493,2.295)--(11.532,2.295)--(11.571,2.295)--(11.610,2.309)--(11.649,2.309)%
  --(11.689,2.309)--(11.729,2.309)--(11.768,2.309)--(11.808,2.309)--(11.848,2.309)--(11.887,2.322)%
  --(11.927,2.322)--(11.966,2.336)--(12.006,2.336)--(12.045,2.349)--(12.124,2.362)--(12.163,2.376)%
  --(12.202,2.376)--(12.241,2.389)--(12.280,2.402)--(12.319,2.416)--(12.358,2.416)--(12.398,2.429)%
  --(12.438,2.429)--(12.477,2.442)--(12.516,2.456)--(12.555,2.469)--(12.594,2.483)--(12.632,2.509)%
  --(12.671,2.523)--(12.710,2.549)--(12.749,2.563)--(12.829,2.563)--(12.868,2.563)--(12.907,2.576)%
  --(12.947,2.576)--(12.986,2.590)--(13.025,2.603)--(13.063,2.630)--(13.101,2.656)--(13.139,2.697)%
  --(13.179,2.697)--(13.218,2.710)--(13.257,2.723)--(13.296,2.737)--(13.335,2.750)--(13.374,2.777)%
  --(13.412,2.804)--(13.450,2.844)--(13.528,2.870)--(13.566,2.884)--(13.605,2.910)--(13.643,2.937)%
  --(13.679,2.991)--(13.718,3.017)--(13.757,3.031)--(13.794,3.058)--(13.831,3.124)--(13.868,3.165)%
  --(13.906,3.205)--(13.940,3.285)--(13.976,3.352)--(14.009,3.472)--(14.043,3.566)--(14.107,3.820)%
  --(14.138,3.940)--(14.172,4.047)--(14.202,4.181)--(14.234,4.314)--(14.267,4.421)--(14.298,4.555)%
  --(14.329,4.689)--(14.363,4.782)--(14.394,4.916)--(14.426,5.050)--(14.459,5.157)--(14.490,5.291)%
  --(14.522,5.398)--(14.555,5.518)--(14.586,5.652)--(14.590,5.665);
\gpsetdashtype{gp dt 3}
\draw[gp path] (9.136,4.029)--(9.148,4.020)--(9.186,3.967)--(9.225,3.940)--(9.263,3.913)%
  --(9.302,3.873)--(9.340,3.833)--(9.417,3.740)--(9.457,3.699)--(9.495,3.646)--(9.534,3.606)%
  --(9.573,3.539)--(9.611,3.485)--(9.650,3.445)--(9.688,3.378)--(9.727,3.325)--(9.765,3.285)%
  --(9.842,3.165)--(9.881,3.098)--(9.920,3.044)--(9.958,2.977)--(9.997,2.910)--(10.035,2.830)%
  --(10.074,2.763)--(10.112,2.683)--(10.151,2.616)--(10.189,2.563)--(10.228,2.496)--(10.267,2.442)%
  --(10.305,2.389)--(10.345,2.322)--(10.383,2.269)--(10.422,2.229)--(10.460,2.188)--(10.499,2.175)%
  --(10.537,2.148)--(10.576,2.135)--(10.615,2.122)--(10.654,2.095)--(10.692,2.081)--(10.731,2.041)%
  --(10.770,2.028)--(10.808,2.001)--(10.847,1.988)--(10.885,1.961)--(10.923,1.948)--(10.962,1.921)%
  --(11.002,1.908)--(11.040,1.894)--(11.079,1.894)--(11.118,1.894)--(11.156,1.908)--(11.195,1.921)%
  --(11.272,1.934)--(11.310,1.921)--(11.349,1.921)--(11.388,1.921)--(11.426,1.921)--(11.465,1.921)%
  --(11.503,1.921)--(11.542,1.934)--(11.580,1.934)--(11.619,1.934)--(11.658,1.921)--(11.696,1.934)%
  --(11.734,1.948)--(11.773,1.961)--(11.811,1.974)--(11.849,1.988)--(11.888,1.988)--(11.927,1.988)%
  --(11.965,2.001)--(12.004,2.015)--(12.042,2.041)--(12.080,2.055)--(12.118,2.081)--(12.157,2.081)%
  --(12.196,2.081)--(12.235,2.081)--(12.273,2.095)--(12.311,2.108)--(12.350,2.135)--(12.387,2.148)%
  --(12.426,2.175)--(12.464,2.175)--(12.503,2.188)--(12.541,2.202)--(12.579,2.215)--(12.617,2.255)%
  --(12.655,2.269)--(12.694,2.282)--(12.732,2.295)--(12.771,2.322)--(12.846,2.376)--(12.884,2.389)%
  --(12.923,2.402)--(12.960,2.442)--(12.996,2.483)--(13.034,2.523)--(13.071,2.576)--(13.108,2.616)%
  --(13.144,2.670)--(13.180,2.737)--(13.252,2.870)--(13.289,2.910)--(13.325,2.977)--(13.361,3.031)%
  --(13.397,3.111)--(13.433,3.178)--(13.467,3.272)--(13.502,3.365)--(13.572,3.526)--(13.643,3.673)%
  --(13.679,3.740)--(13.714,3.820)--(13.749,3.913)--(13.785,3.994)--(13.819,4.101)--(13.853,4.194)%
  --(13.887,4.301)--(13.952,4.582)--(13.984,4.716)--(14.016,4.863)--(14.049,5.010)--(14.082,5.130)%
  --(14.116,5.237)--(14.149,5.357)--(14.183,5.464)--(14.218,5.545)--(14.252,5.665);
\gpsetdashtype{gp dt 4}
\draw[gp path] (9.136,4.211)--(9.146,4.208)--(9.185,4.181)--(9.223,4.154)--(9.262,4.127)%
  --(9.299,4.101)--(9.338,4.087)--(9.376,4.060)--(9.414,4.047)--(9.452,4.020)--(9.529,3.994)%
  --(9.568,3.980)--(9.606,3.953)--(9.645,3.940)--(9.684,3.913)--(9.722,3.887)--(9.761,3.860)%
  --(9.799,3.846)--(9.837,3.820)--(9.876,3.806)--(9.914,3.793)--(9.952,3.766)--(9.991,3.726)%
  --(10.028,3.713)--(10.066,3.646)--(10.142,3.539)--(10.180,3.485)--(10.219,3.419)--(10.257,3.338)%
  --(10.295,3.258)--(10.334,3.178)--(10.372,3.098)--(10.410,3.017)--(10.448,2.937)--(10.486,2.830)%
  --(10.524,2.723)--(10.562,2.603)--(10.600,2.483)--(10.638,2.376)--(10.676,2.282)--(10.714,2.162)%
  --(10.752,2.041)--(10.828,1.801)--(10.867,1.707)--(10.906,1.640)--(10.944,1.573)--(10.983,1.533)%
  --(11.022,1.493)--(11.060,1.466)--(11.099,1.440)--(11.137,1.440)--(11.176,1.440)--(11.214,1.426)%
  --(11.252,1.413)--(11.291,1.413)--(11.328,1.400)--(11.368,1.386)--(11.406,1.359)--(11.445,1.346)%
  --(11.484,1.359)--(11.522,1.359)--(11.561,1.359)--(11.599,1.346)--(11.637,1.359)--(11.675,1.373)%
  --(11.713,1.373)--(11.752,1.373)--(11.791,1.359)--(11.830,1.359)--(11.869,1.373)--(11.907,1.373)%
  --(11.946,1.373)--(12.023,1.386)--(12.061,1.386)--(12.100,1.400)--(12.138,1.413)--(12.175,1.426)%
  --(12.214,1.440)--(12.252,1.440)--(12.290,1.466)--(12.327,1.506)--(12.364,1.533)--(12.403,1.547)%
  --(12.477,1.600)--(12.514,1.627)--(12.552,1.640)--(12.590,1.667)--(12.626,1.707)--(12.663,1.734)%
  --(12.701,1.761)--(12.738,1.787)--(12.774,1.841)--(12.812,1.854)--(12.849,1.894)--(12.884,1.948)%
  --(12.921,1.988)--(12.955,2.068)--(12.990,2.135)--(13.056,2.349)--(13.086,2.483)--(13.114,2.656)%
  --(13.143,2.830)--(13.169,3.031)--(13.196,3.231)--(13.223,3.445)--(13.250,3.646)--(13.277,3.846)%
  --(13.303,4.047)--(13.333,4.208)--(13.364,4.341)--(13.395,4.475)--(13.426,4.609)--(13.489,4.836)%
  --(13.521,4.970)--(13.552,5.090)--(13.583,5.210)--(13.614,5.344)--(13.645,5.478)--(13.675,5.612)%
  --(13.686,5.665);
\gpsetdashtype{gp dt solid}
\draw[gp path] (9.136,5.665)--(9.136,0.985)--(15.447,0.985)--(15.447,5.665)--cycle;
%% coordinates of the plot area
\gpdefrectangularnode{gp plot 6}{\pgfpoint{9.136cm}{0.985cm}}{\pgfpoint{15.447cm}{5.665cm}}
\end{tikzpicture}
%% gnuplot variables

    \caption[Coeficientes Aerodinâmicos dos Perfis Selecionados]{Os dados foram
    obtidos através do software de CFD XFOIL\cite{xfoil}, configurado
    para $\text{Re} = 5\cdot10^6$ e $\text{Ma} =0,3$.}
    \label{fig:aero_coef}
\end{figure}

O perfil escolhido para o projeto foi o 66\textsubscript{3}-418, devido ao seu baixo ${c_d}_{\min}$ e grande balde laminar. Além disso ele é bem comportado no estol,que ocorre com relativamente alto ângulo de ataque e em um $c_l$ na mesma faixa dos outros perfis da série 6. Sua alta espessura é vantajosa para reduzir o peso estrutural da aeronave e permite amplo espaço para tanques de combustível.

A escolha de um perfil de escoamento laminar só é vantajosa se a operação do avião ocorrer dentro do balde laminar. Para verificar o atendimento a esse critério, é necessário calcular o coeficiente de sustentação para a situação de cruzeiro e subida.
\begin{equation}
  (C_L)_{\text{cruise}} =
    \left.
      \frac {W} {\frac{1}{2} \rho V^2 S}
    \right|_{\text{cruise}} =
    \frac
      {14549\si{kg} \cdot 9,8\si{N/kg} }
      {\frac{1}{2} \cdot 0,4135 \si{kg/m^3} \cdot (140\si{m/s})^2 \cdot 53 \si{m^2}}
    = 0,66
\end{equation}

\begin{equation}
  (C_L)_{\text{climb}} =
    \left.\frac
      {W}
      {\frac{1}{2} \rho V^2 S}
    \right|_{\text{climb}} =
    \frac
      {16957\si{kg} \cdot 9,8\si{N/kg}}
      {\frac{1}{2} \cdot 1,225 \si{kg/m^3}
    (58\si{m/s})^2 \cdot 53 \si{m^2}}
    = 1,52
\end{equation}

Considerando o peso de cruzeiro como o peso do avião com o máximo de passageiros e metade do combustível máximo, a 30000ft, e o início da subida, ao nível do mar 30.000 \si{ft}, com o máximo de passageiros e 96,5\% do combustível, conforme sugerido por \cite{roskam} e velocidade ideal de subida.

A \autoref{fig:aero_coef} mostra que, no ponto de projeto atual, o valor de
$C_l$ para o cruzeiro está no balde laminar. O $C_l$ de subida está preocupantemente próximo do estol, então a velocidade de subida deve ser maior que $58\si{m/s}$. Com $62\si{m/s}$ o $C_l$ diminui para 1,33.

\section{Forma em planta da asa}
\label{formaemplanta_asa}

Para uma primeira aproximação, a asa foi definida como trapezoidal com um afilamento de $\lambda = 0,45$, conforme sugerido por \cite{raymer2012aircraft} para aproximar a distribuição de sustentação da ideal, elíptica. O alongamento foi estimado com base na comparação entre aeronaves da categoria como $A = 14$, ligeiramente maior que a dos concorrentes, para maior eficiência aerodinâmica. Veja a \autoref{fig:alongamento}. Além disso, conforme definido na \autoref{sec:ponto_de_projeto}, a área alar é de $S=53\si{m^2}$. O bordo de ataque será reto para evitar interferência da asa com a hélice, e também para melhorar as características de estol. O efeito estrutural do enflexamento negativo pode ser compensando com o uso de compósitos. Com esses parâmetros, é fácil derivar os outros parâmetros da asa, descritos na \autoref{tbl:asa_trapezoidal}

\begin{table} [H]
\caption{Parâmetros da asa trapezoidal}
\label{tbl:asa_trapezoidal}
\begin{tabular}{llccrl}
\toprule
					  & Nome        & Símbolo   & Fórmula & Valor & \\ \midrule
Parâmetros de projeto & Alongamento & $A$       &         & 14    & \\
                      & Afilamento  & $\lambda$ &         & 0,45  & \\
                      & Área alar   & $S$		&         & 53    & \si{m^2}\\ \midrule
Parâmetros calculados & Envergadura & $b$       & $\sqrt{AS}$ & 27,240&\si{m} \\
                      & Corda na raiz & $c_r$ & $\frac{2b}{(1+\lambda)A}$ & 2,684 & \si{m} \\
                      & Corda média aerodinâmica & CMA & $\frac{2}{3} c_r \frac{1+\lambda +\lambda^2}{1+\lambda}$ & 2,039 & \si{m} \\
                      & Corda na ponta & $c_t$ & $\lambda c_r$ & 1,207 & \si{m} \\
                      & Enflexamento & $\gamma$ & $-\arctan\left(\frac{c_r-c_t}{2b}\right)$ & -1,55 & deg\\
\bottomrule
\end{tabular}
\end{table}

\section{Dimensionamento Inicial das Empenagens}
\label{formaemplanta_empenagens}
% --------------------------
% DIMENSIONAMENTO EMPENAGENS
% --------------------------

Definido o ponto de projeto em termos de P, W e S, a próxima etapa é a determinação das áreas das empenagens. A metodologia aplicada consiste em atingir o volume de cauda típico para a classe de aeronaves turboélice como sugerido por \cite{raymer2012aircraft}. A definição do volume de cauda para empenagens horizontal e vertical são apresentadas abaixo.

%%%%%%%%%
% DEIXAR EQUAÇÕES BONITAS
%%%%%%%%%

%% Equação volume de cauda
\begin{equation}
\overline{V}_{eh} = \frac{S \overline{c}_w}{S_{eh} l_t} 
\end{equation}

\begin{equation}
\overline{V}_{ev} = \frac{S b_w}{S_{ev} l_t} 
\end{equation}

Os valores típicos para o volume de cauda utilizados foram baseados em \cite{raymer2012aircraft} e estão apresentados na tabela abaixo para aeronaves turboélice.

\begin{table}[H]
\centering
\caption{Volume de cauda típicos para aeronaves turboélice}
\label{tbl:vbar}
\begin{tabular}{cc}
\toprule
$\overline{V}_{eh}$ & 0.9  \\
$\overline{V}_{ev}$  & 0.08  \\
\bottomrule
\end{tabular}
\end{table}

Para dimensionamento das empenagens, é necessário o conhecimento da corda média aerodinâmica ($\overline{c}_w$) e envergadura da asa e distância do centro aerodinâmico da asa ao centro aerodinâmico das empenagens que, por simplicidade, é considerado o mesmo para a empenagem horizontal e vertical ($l_t$). A estimativa considerou o ponto de projeto escolhido e o comprimento de fuselagem definido na Parte I deste trabalho ($l_{fus} = 29.87$). A tabela abaixo apresenta os valores considerados

\begin{table}[H]
\centering
\caption{Corda média aerodinâmica da asa e distância do centro aerodinâmico da asa ao centro aerodinâmico das empenagens}
\label{tbl:lt}
\begin{tabular}{cc}
\toprule
$\overline{c}_w$ & 2.039 m \\
$b_w$  & 27.240 m \\
$l_t$  & 13.5 m \\
\bottomrule
\end{tabular}
\end{table}

A partir do volume de cauda típico e estimativas referentes a geometria externas da aeronave, tem-se abaixo as áreas necessárias para a empenagem horizontal e vertical.

\begin{table}[H]
\centering
\caption[Área das Empenagens]{Área das empenagens}
\label{tbl:s_eh}
\begin{tabular}{cc}
\toprule
$S_{eh}$ & 7.1 $m^2$ \\
$S_{ev}$  & 8.8 $m^2$ \\
\bottomrule
\end{tabular}
\end{table}

Por fim, a \autoref{tbl:Dimensoes_iniciais} resume os parâmetros da aeronave definidos a partir do ponto de projeto escolhido e do dimensionamento das empenagens.

\begin{table}[H]
\centering
\caption[Dimensionamento inicial da aeronave]{Dimensionamento inicial da aeronave}
\label{tbl:Dimensoes_iniciais}
\begin{tabular}{cc}
\toprule
$P$ & 1846 $kW$ \\
$S$ & 53 $m^2$ \\
$S_{eh}$ & 7.1 $m^2$ \\
$S_{ev}$  & 8.8 $m^2$ \\
$l_t$ & 13.5 $m$ \\
$\overline{c}_w$ & 2 $m$ \\
\bottomrule
\end{tabular}
\end{table}



