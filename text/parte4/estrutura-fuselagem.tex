\section{Dimensionamento Estrutural da Fuselagem}

\subsection{Espessura do revestimento}
A espessura do revestimento foi dimensionada com base na pressão de pressurização que ele deve suportar.
Os demais esforços serão suportados por esse revestimento somado aos reforçadores e cavernas.

O teto de serviço da aeronave é de 25000ft,
enquanto a cabine é pressurizada a 6000ft,
conforme definido na \autoref{sec:sys_press}.
Considerando atmosfera padrão (ISA) a diferença de pressão é de 44kPa.

O material do revestimento foi escolhido como alumínio 2024-T3, devido à sua boa resistência a corrosão, a impactos e estrutural.
Por isso ele é de emprego comum na indústria aeronáutica para revestimentos.
O limite de elasticidade desse material é de 290MPa.
Considerando um fator de segurança para o material de 1,15, o limite de escoamento fica $\sigma_y = 252\text{MPa}$.

Para um vaso de pressão cilindrico, temos
\begin{align}
\sigma_\text{long} &= \frac{pr}{2t} \\
\sigma_\theta      &= \frac{pr}{t}
\end{align}

O esforço na direção $\theta$ é claramente o dimensionante, portanto, considerando um fator de projeto de 1,5:
\begin{equation}
    t = \frac{pd}{2\sigma_y} = \frac{1,5 \cdot 44\si{kPa} \cdot 2070\si{mm}}{2 \cdot 252\si{MPa}} = 0,27\si{mm}
\end{equation}

Para aumentar a resistência a impactos e a facilidade de fabricação, a espessura do revestimento foi aumentada para $2\si{mm}$.
